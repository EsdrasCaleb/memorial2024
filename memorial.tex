%%%%%%%%%%%%%%%%%%%%%%%%%%%%%%%%%%%%%%%%%%%%%%%%%%%%%%%%%%%%%%%%%%%%%%%%%%%%%%%
% Memorial para concurso público de Professor Doutor na USP.
%
% Formatação inspirada em:
% * https://tug.org/pracjourn/2008-1/mori/mori.pdf
% * https://github.com/santisoler/phd-thesis
% * https://github.com/compgeolab/dissertation-template
%%%%%%%%%%%%%%%%%%%%%%%%%%%%%%%%%%%%%%%%%%%%%%%%%%%%%%%%%%%%%%%%%%%%%%%%%%%%%%%

%%%%%%%%%%%%%%%%%%%%%%%%%%%%%%%%%%%%%%%%%%%%%%%%%%%%%%%%%%%%%%%%%%%%%%%%%%%%%%%
% Set a class and import packages
\documentclass[10pt,a4paper,oneside]{book}

% Variables
\newcommand{\Year}{2024}
\newcommand{\Author}{Esdras Caleb Oliveira Silva}
\newcommand{\Title}{Memorial e Projeto de Atuação Profissional de \Author{} para concurso público - Professor do Magistério Superior Adjunto-A - IMD/UFRN}
\newcommand{\Email}{Leonardo.Uieda@liverpool.ac.uk}
\newcommand{\EmailPersonal}{leouieda@gmail.com}
\newcommand{\ORCID}{0000-0001-6123-9515}
\newcommand{\ResearcherID}{G-3258-2012}
\newcommand{\GoogleScholar}{qfmPrUEAAAAJ}
\newcommand{\Lattes}{8939551682050504}

% Variables for easier typing of some names
\newcommand{\UFF}{Universidade Federal Fluminense}
\newcommand{\UFRN}{Universidade Federal do Rio Grande do Norte}

% Names for citing coauthors
\newcommand{\Me}{\textbf{Uieda, L}}
\newcommand{\Val}{Barbosa, VCF}
\newcommand{\Bi}{Oliveira Jr, VC}
\newcommand{\Paul}{Wessel, P}
\newcommand{\Joaquim}{Luis, J}
\newcommand{\Remko}{Scharroo, R}
\newcommand{\Florian}{Wobbe, F}
\newcommand{\Walter}{Smith, WHF}
\newcommand{\Dongdong}{Tian, D}
\newcommand{\Bridget}{Smith-Konter, B}
\newcommand{\Eric}{Xu, X}
\newcommand{\David}{Sandwell, DT}
\newcommand{\Carla}{Braitenberg, C}
\newcommand{\Naomi}{Ussami, N}
\newcommand{\Manoel}{D'Agrella-Filho, MS}
\newcommand{\JB}{Silva, JBC}
\newcommand{\Dai}{Sales, DP}
\newcommand{\Figura}{Melo, FF}
\newcommand{\Dio}{Carlos, DU}
\newcommand{\BragaVale}{Braga, MA}
\newcommand{\YLi}{Li, Y}
\newcommand{\Angeli}{Angeli, G}
\newcommand{\Peres}{Peres, G}
\newcommand{\Everton}{Bomfim, EP}
\newcommand{\Eder}{Molina, E}
\newcommand{\Gomes}{Gomes, AAS}
\newcommand{\Santiago}{Soler, SR}
\newcommand{\Agustina}{Pesce, A}
\newcommand{\Gimenez}{Gimenez, ME}
\newcommand{\Kristoffer}{Hallam, KAT}
\newcommand{\Guangdong}{Zhao, G}
\newcommand{\Bo}{Chen, B}
\newcommand{\JLiu}{Liu, J}
\newcommand{\LChen}{Chen, L}
\newcommand{\RGuo}{Guo, R}
\newcommand{\MKaban}{Kaban, MK}
\newcommand{\Lindsey}{Heagy, LJ}
\newcommand{\Lion}{Krischer, L}
\newcommand{\Rene}{Gassmoeller, R}
\newcommand{\Bane}{Sullivan, CB}
\newcommand{\Jens}{Klump, JF}
\newcommand{\LBarba}{Barba, LA}
\newcommand{\JBazan}{Bazan, J}
\newcommand{\JBrown}{Brown, J}
\newcommand{\RGuimera}{Guimera, RV}
\newcommand{\MGymrek}{Gymrek, M}
\newcommand{\AHanna}{Alex Hanna}
\newcommand{\KHuff}{Huff, KD}
\newcommand{\DKatz}{Katz, DS}
\newcommand{\CMadan}{Madan, CR}
\newcommand{\KMoerman}{Moerman, KM}
\newcommand{\KNiemeyer}{Niemeyer, KE}
\newcommand{\JPoulson}{Poulson, JL}
\newcommand{\PPrins}{Prins, P}
\newcommand{\KRam}{Ram, K}
\newcommand{\ARokem}{Rokem, A}
\newcommand{\Arfon}{Smith, AM}
\newcommand{\GThiruvathukal}{Thiruvathukal, GK}
\newcommand{\KThyng}{Thyng, KM}
\newcommand{\BWilson}{Wilson, BE}
\newcommand{\Yehudi}{Yehudi, Y}
\newcommand{\Remi}{Rampin, R}
\newcommand{\Hugo}{van Kemenade, H}
\newcommand{\MattTurk}{Turk, M}
\newcommand{\Shapero}{Shapero, D}
\newcommand{\Anderson}{Banihirwe, A}
\newcommand{\Leeman}{Leeman, J}
\newcommand{\JEbbing}{Ebbing, J}
\newcommand{\AGuy}{Guy, A}
\newcommand{\JFarquharson}{Farquharson, J}
\newcommand{\AKushnir}{Kushnir, A}
\newcommand{\FWadsworth}{Wadsworth, F}
\newcommand{\LPerozzi}{Perozzi, L}
\newcommand{\MWieczorek}{Wieczorek, MA}
\newcommand{\LLi}{Li, L}
\newcommand{\Ricardo}{Trindade, RIF}

% Links to webpages I use often
\newcommand{\SantiagoLink}{\href{https://www.santisoler.com/}{Santiago R. Soler}}
\newcommand{\VanderleiLink}{\href{https://www.pinga-lab.org/people/oliveira-jr.html}{Vanderlei C. Oliveira Jr.}}
\newcommand{\SandwellLink}{\href{https://topex.ucsd.edu/sandwell/}{David Sandwell}}
\newcommand{\ValeriaLink}{\href{https://www.pinga-lab.org/people/barbosa.html}{Valéria C. F. Barbosa}}
\newcommand{\PaulLink}{\href{https://www.soest.hawaii.edu/pwessel/}{Paul Wessel}}
\newcommand{\IndiaLink}{\href{https://www.compgeolab.org/team/\#indiauppal}{India Uppal}}
\newcommand{\GelsonLink}{\href{https://www.compgeolab.org/team/\#Souza-junior}{Gelson Ferreira de Souza Junior}}

\newcommand{\GMTLink}{\href{https://www.generic-mapping-tools.org}{Generic Mapping Tools}}
\newcommand{\CompGeoLabLink}{\href{https://www.compgeolab.org/}{Computer-Oriented Geoscience Lab}}
\newcommand{\SwunngLink}{\href{https://softwareunderground.org/}{Software Underground}}
\newcommand{\FatiandoLink}{\href{https://www.fatiando.org}{Fatiando a Terra}}
\newcommand{\PyGMTLink}{\href{https://www.pygmt.org}{PyGMT}}
\newcommand{\SSILink}{\href{https://software.ac.uk/}{Software Sustainability Institute}}

% Import packages
\usepackage[utf8]{inputenc}
\usepackage[T1]{fontenc}
\usepackage[brazil]{babel}
\usepackage{geometry}
\usepackage{graphicx}
\usepackage{amssymb}
\usepackage{amsmath}
\usepackage{mathpazo}
\usepackage{hyperref}
% create fancy headers
\usepackage{fancyhdr}
% commands for managing dates and its formats
\usepackage{datetime2}
% improved urls with proper hyphenation
\usepackage{xurl}
% Control over enumerate and itemize
\usepackage{enumitem}
% Tweak the look of captions
\usepackage{caption}
% To control the style of section titles
\usepackage{titlesec}
% Add the bibliography to the table of contents
\usepackage[nottoc,chapter]{tocbibind}
\usepackage[round,authoryear,sort]{natbib}
% show dois as links on references
\usepackage{doi}
% Icon and fonts (requires using xelatex or luatex)
\usepackage{fontawesome5}
\usepackage{academicons}
\usepackage{fontspec}
\usepackage[mono]{notomath}
% To make everything neater
\usepackage{microtype}
% To make fancy text boxes
\usepackage{xcolor}
\usepackage[framemethod=default]{mdframed}
% For fancy and multipage tables
\usepackage{tabularx}
\usepackage{ltablex}
% To define custom environments
\usepackage{environ}
\usepackage{setspace}
% Reference sections by name
\usepackage{nameref}
% Better handling of footnotes inside summary boxes
\usepackage{footmisc}
%%%%%%%%%%%%%%%%%%%%%%%%%%%%%%%%%%%%%%%%%%%%%%%%%%%%%%%%%%%%%%%%%%%%%%%%%%%%%%%

%%%%%%%%%%%%%%%%%%%%%%%%%%%%%%%%%%%%%%%%%%%%%%%%%%%%%%%%%%%%%%%%%%%%%%%%%%%%%%%
% Configuration of the document

\geometry{%
  left=30mm,
  right=30mm,
  top=20mm,
  bottom=15mm,
  headsep=5mm,
  headheight=5mm,
  footskip=10mm,
  includehead=true,
  includefoot=true
}

% Increase the line spacing
\SetSinglespace{1.2}
\onehalfspacing

% Remove spacing between enumerate/itemize items
\setlist{nosep}

% Padding between the first figure and the chapter title
\newcommand{\HeroFigPad}{\vspace{-1cm}}

% Padding before the software logo figures
\newcommand{\SoftwareFigPad}{\vspace{-0.3cm}}

% Add a link to a DOI
\newcommand{\DOI}[1]{\url{https://doi.org/#1}}

% Add a link to a GitHub repository
\newcommand{\GitHub}[1]{\faGithub{} Código: \url{https://github.com/#1}}

% Add a link to a YouTube video
\newcommand{\YouTube}[1]{\faYoutube{} Vídeo: \url{https://youtu.be/#1}}

% Add a link to a supplementary data
\newcommand{\Data}[1]{\faChartBar{} Dados: \url{https://doi.org/#1}}

% Add a link to a preprint
\newcommand{\Preprint}[1]{\faLockOpen{} Preprint: \url{https://doi.org/#1}}

% Make a Unicode bullet symbol
\newcommand{\Bullet}{•\enspace}

% Define custom colors
\definecolor{lu_gray}{gray}{0.98}
\definecolor{lu_darkgray}{gray}{0.3}
\definecolor{lu_blue}{RGB}{32, 96, 194}
\definecolor{lu_lightblue}{RGB}{238, 245, 250}
\definecolor{lu_yellow}{RGB}{255, 193, 7}
\definecolor{lu_lightyellow}{RGB}{255, 249, 230}

% Customize how Chapter headings are displayed
\titleformat{\chapter}[display]{\normalfont}{\large Capítulo \thechapter}{0pt}{\huge}[\titlerule]
\titlespacing*{\chapter}{0pt}{-40pt}{40pt}

% Set the spacing between bibliography entries (requires natbib)
\setlength{\bibsep}{0pt}

% Configure captions
\captionsetup{labelfont=bf,font={small,color=lu_darkgray},skip=0pt}

% Define a fancy text box
\mdfdefinestyle{summarybox}{%
  leftline=true,
  rightline=false,
  topline=false,
  bottomline=false,
  linewidth=4pt,
  linecolor=lu_blue,
  frametitlefont=\bfseries\color{black}\small,
  frametitlebackgroundcolor=lu_lightblue,
  frametitleaboveskip=7pt,
  frametitlebelowskip=7pt,
  frametitlerule=true,
  frametitlerulewidth=1pt,
  backgroundcolor=lu_gray,
  innertopmargin=7pt,
  innerbottommargin=10pt,
  innerleftmargin=15pt,
  innerrightmargin=15pt,
  skipbelow=5pt,
  skipabove=0pt,
}
\newmdenv[style=summarybox]{summarybox}
\mdfdefinestyle{subsummarybox}{%
  leftline=true,
  rightline=false,
  topline=false,
  bottomline=false,
  linewidth=4pt,
  linecolor=lu_yellow,
  frametitlefont=\bfseries\color{black}\small,
  frametitlebackgroundcolor=lu_lightyellow,
  frametitleaboveskip=7pt,
  frametitlebelowskip=7pt,
  frametitlerule=true,
  frametitlerulewidth=1pt,
  backgroundcolor=lu_gray,
  innertopmargin=7pt,
  innerbottommargin=10pt,
  innerleftmargin=15pt,
  innerrightmargin=15pt,
  skipbelow=5pt,
  skipabove=0pt,
}
\newmdenv[style=subsummarybox]{subsummarybox}

% Define something like an fa-ul and a date list
\NewEnviron{fa-ul}{%
  \vspace{-0.4cm}
  \small
  \renewcommand{\arraystretch}{1.25}
  \begin{tabularx}{\linewidth}{@{}p{0.05\linewidth}@{}@{}p{0.95\linewidth}@{}}
    \BODY
  \end{tabularx}%
}
\NewEnviron{datelist}{%
  \vspace{-0.4cm}
  \small
  \renewcommand{\arraystretch}{1.25}
  \begin{tabularx}{\linewidth}{@{}p{0.15\linewidth}@{}@{}p{0.85\linewidth}@{}}
    \BODY
  \end{tabularx}%
}
\NewEnviron{paperlist}{%
  \vspace{-0.4cm}
  \small
  \renewcommand{\arraystretch}{1.25}
  \begin{tabularx}{\linewidth}{@{}p{0.08\linewidth}@{}@{}p{0.92\linewidth}@{}}
    \BODY
  \end{tabularx}%
}
\NewEnviron{courselist}{%
  \vspace{-0.4cm}
  \small
  \renewcommand{\arraystretch}{1.25}
  \begin{tabularx}{\linewidth}{@{}p{0.15\linewidth}@{}@{}p{0.85\linewidth}@{}}
    \BODY
  \end{tabularx}
}

% Define a fancy enumerate that has a title
\NewEnviron{fancyenum}[2]{%
  \vspace{0.25cm}
  \noindent#1\quad\textbf{#2}:
  \vspace{0.25cm}
  \begin{enumerate}
    \BODY
  \end{enumerate}
}

% Configure hyperref and add PDF metadata
\hypersetup{
    colorlinks,
    allcolors=lu_blue,
    pdftitle={\Title},
    pdfauthor={\Author},
    pdftex,
    breaklinks=true,
}

% make urls use the same font as every other text
\urlstyle{same}

% Prevent footnotes from being broken into multiple pages
\interfootnotelinepenalty=10000

% Configure headers and footers
\fancyhf{}
\lhead{\fontsize{9pt}{0}\selectfont\itshape \nouppercase\leftmark}
\chead{}
\rhead{\fontsize{9pt}{0}\selectfont \thepage}
\cfoot{}
\renewcommand{\headrulewidth}{0.3pt}
%%%%%%%%%%%%%%%%%%%%%%%%%%%%%%%%%%%%%%%%%%%%%%%%%%%%%%%%%%%%%%%%%%%%%%%%%%%%%%%

%%%%%%%%%%%%%%%%%%%%%%%%%%%%%%%%%%%%%%%%%%%%%%%%%%%%%%%%%%%%%%%%%%%%%%%%%%%%%%%
\begin{document}

\pagestyle{plain}
\frontmatter

\begin{titlepage}
  \begin{center}
    \includegraphics[height=2cm]{images/logo.pdf}
    \vspace{1cm}

    CONCURSO PÚBLICO

    PROFESSOR ADJUNTO A EM MLOPS

    INSTITUTO METROPOLE DIGITAL
    \vspace{5cm}

    \textbf{\LARGE MEMORIAL E PROJETO DE ATUAÇÂO PROFISSIONAL}
    \vspace{1cm}

    \textbf{\LARGE \MakeUppercase{\Author{}}}
    \vspace{5cm}

    {\small
      Apresentado para concurso público de títulos e provas para cargo de

      Professor Adjunto junto ao Instituto Metropole Digital

      Universidade Federal do Rio Grande do Norte.
      \vspace{1cm}

      Edital 069/2024-PROGESP
    }
    \vfill

    \Year{}
  \end{center}
\end{titlepage}

%==============================================================================
\chapter*{Resumo}

Possuo Bacharelado em Engenharia de Telecomunicações e Mestrado em Ciência da Computação pela \UFF{}.
Atualmente, estou cursando o Doutorado em Ciência da Computação na \UFRN{}.

Atuei como professor no programa PRONATEC e monitor da disciplina de Microprocessadores na \UFF{}. Desde a graduação,
venho me dedicando ao desenvolvimento e implementação de sistemas, com destaque para minha atuação no Laboratório de
Difração de Raios X da UFF (\textbf{LDRX}) durante uma bolsa do CNPq. Posteriormente, ampliei essa experiência em meu
estágio na GO2WEB e no desenvolvimento de sistemas de ensino a distância na UFF, além de meu trabalho atual na SEDIS-UFRN.
Essa trajetória também me proporcionou sólida expertise em extração e tratamento de dados.

Meu interesse em Inteligência Artificial começou na graduação, mas foi com o advento das LLMs que tive a oportunidade
de trabalhar mais intensamente com essa tecnologia, utilizando modelos abertos como o LLaMA~\cite{touvron2023llama}.
Minha atuação acadêmica inclui prêmios por trabalhos em Televisão Digital, com menções honrosas e publicações em eventos
relevantes.

Este memorial apresenta minha formação e trajetória profissional, incluindo reflexões sobre os fatores que me trouxeram
até aqui, as lições aprendidas ao longo do caminho e minha motivação para retornar à carreira acadêmica. Além disso, o
documento relata meus planos futuros para minha atuação no IMD e os projetos que pretendo desenvolver.



% usar linha do tempo?

%==============================================================================
\tableofcontents

\mainmatter
\pagestyle{fancy}

%==============================================================================
\chapter{Introdução}

\begin{figure}[h]
  \HeroFigPad
  \begin{center}
    \includegraphics[width=\textwidth]{images/1997-06-ithaca-creek.jpg}
  \end{center}
  \caption{
    Minha mãe mostrando para mim e minha irmã caçula o lado inferior de uma
    pedra em um riacho, provavelmente contendo invertebrados aquáticos.
    Foto de Junho de 1997, tirada no estado de Nova York, E.U.A., durante o
    pós-doutorado de meus pais na
    \href{https://www.cornell.edu/}{Cornell University}.
  }
  \label{fig_riacho}
\end{figure}
\begin{summarybox}[frametitle=\faInfoCircle{}\quad Informações para contato]
  \begin{fa-ul}
    \faEnvelope & email profissional: \href{mailto:\Email}{\Email} \\
    \faEnvelope & email pessoal: \href{mailto:\EmailPersonal}{\EmailPersonal} \\
    \aiOrcid & ORCID: \href{https://orcid.org/\ORCID}{\ORCID} \\
    \aiLattes & Currículo Lattes: \url{https://lattes.cnpq.br/\Lattes} \\
    \aiPublonsSquare & ResearcherID: \href{https://www.webofscience.com/wos/author/rid/\ResearcherID}{\ResearcherID} \\
    \faUser & Página pessoal: \url{https://www.leouieda.com} \\
    \faUsers & Grupo de pesquisa: \url{https://www.compgeolab.org}
  \end{fa-ul}
\end{summarybox}

Este memorial apresenta uma análise reflexiva sobre os principais temas de
minha carreira acadêmica: minha formação, minhas linhas de pesquisa, minhas
atividades de ensino e extensão e meus esforços para tornar a ciência feita em
nossa disciplina mais aberta, reprodutível e acessível para uso das comunidades
científica, acadêmica e empresarial.
Ao buscar a fonte de vários dos princípios que me guiam hoje em dia, percebi
que o ponto mais adequado para começar seria com uma análise das influências
que tive durante minha criação.

\section{Influências durante a infância e a adolescência}

Meu primeiro contato com a ciência foi através de meus pais,
\href{https://orcid.org/0000-0002-6078-1342}{Virginia Sanches Uieda} e
\href{https://orcid.org/0000-0002-4177-3339}{Wilson Uieda},
ambos professores aposentados do Instituto de Biociências da Universidade
Estadual Paulista Júlio de Mesquita Filho (UNESP) de Botucatu, São Paulo.
Eles rotineiramente incluíam minhas duas irmãs e eu em suas atividades como
docentes da UNESP, o que nos proporcionou oportunidades de aprendizagem únicas
e que foram particularmente influentes na minha formação.
Tenho memórias marcantes de coletar peixes e invertebrados aquáticos com minha
mãe (figura~\ref{fig_riacho}), fotografar morcegos do gênero
\href{https://pt.wikipedia.org/wiki/Artibeus}{\textit{Artibeus}} se alimentando
dos frutos do chapéu-de-sol com meu pai, observar minha mãe corrigindo provas
de zoologia de vertebrados e tentar acertar mais questões que seus alunos,
alimentar os morcegos
\href{https://pt.wikipedia.org/wiki/Desmodus_rotundus}{\textit{Desmodus rotundus}}
que meu pai mantinha em cativeiro com cubos de sangue bovino congelado nos
finais de semana e acompanhar minha mãe na disciplina de campo sobre cetáceos
onde pudemos interagir diretamente com botos-cinza
(\href{https://pt.wikipedia.org/wiki/Sotalia_guianensis}{\textit{Sotalia guianensis}})
em seu habitat natural.

A curiosidade, a dedicação e a ética dos meus pais formaram a base da minha
posição a respeito da ciência e do que significa ser um educador de qualidade.
Essa base e todo o apoio que recebi de meus pais foram fundamentais para
alcançar tudo o que consegui até hoje (i.e., o conteúdo deste memorial).

\section{Reflexão sobre vantagens e privilégios}

Este memorial representa todas as minhas conquistas ao longo da minha carreira.
Dedicação, esforço e talento (i.e., mérito) foram certamente importantes para
meu sucesso profissional.
Porém, seria muito ingênuo de minha parte assumir que esses foram os únicos
fatores que influenciaram minha trajetória.
Por isso, acho importante refletir sobre as vantagens e privilégios que tive
sobre meus contemporâneos para dar contexto ao resto do memorial.

Primeiramente, sou homem, heterossexual, cisgênero e de etnia mista branca
europeia e norte asiática.
A junção desses fatores significa que, por nenhum mérito próprio, tive que
superar um número consideravelmente menor de barreiras ao longo de minha
carreira que outras pessoas.
Fui criado por pais dedicados e com imenso suporte de toda minha família
estendida.
Minha família é de classe média alta e tive acesso a educação privada em boas
escolas.
Ao contrário de alguns dos meus colegas do curso de graduação, não tive que
trabalhar para me sustentar durante meu curso de graduação, podendo me dedicar
exclusivamente aos estudos\footnote{E, é claro, às festas e outras atividades
culturais que enriquecem a experiência universitária.}.

Ter pais acadêmicos, em particular, me conferiu diversas vantagens.
Antes mesmo de ingressar no ensino superior, eu já sabia sobre o estilo de
trabalho, a trajetória para se chegar ao cargo de Professor Doutor, o balanço
entre ensino, pesquisa e extensão, os tipos de cargos administrativos que
existem, entre outros.
Mas talvez a vantagem mais importante que meus pais deram foi a oportunidade
de morar no exterior quando criança.
Entre Agosto de 1996 e Dezembro de 1997, meus pais fizeram um pós-doutorado
na \href{https://www.cornell.edu/}{Cornell University}, E.U.A., levando junto
toda a família.
Por isso, cursei o quinto e sexto ano do ensino fundamental nos Estados Unidos
e aprendi a ler, escrever e falar inglês fluentemente.
Somente percebi o quanto esse único fator (fluência na língua inglesa) me foi
vantajoso após ingressar no curso de Bacharelado em Geofísica da Universidade
de São Paulo (seção~\ref{sec_usp}).
Eu era capaz de ler livros e artigos em inglês em menos tempo que meus colegas,
me comunicava com pesquisadores estrangeiros naturalmente durante meu trabalho
de conclusão de curso e creio que minha fluência na língua foi um fator
importante para conseguir o intercâmbio com a York University, Canadá,
(seção~\ref{sec_york}).

A sorte é outro fator que foi muito importante em diversas etapas da minha
carreira.
Minha decisão de prestar o vestibular da USP para o curso de Geofísica dependeu
de minha irmã mais velha encontrar aleatoriamente um aluno de Geofísica no
``bandeijão'' da USP que lhe contou sobre o curso.
Como eu estava indeciso sobre minhas escolhas de carreira, selecionei Geofísica
como minha primeira opção por conselho de minha irmã sem saber exatamente do
que se tratava o curso.
Ter entrado no curso de Geofísica na USP no ano de 2004, em particular, foi
extremamente oportuno.
A turma da Geofísica de 2004 é simplesmente excepcional.
O apoio da turma foi muito importante, tanto para superar momentos desafiadores
quanto para elevar cada um de nós a alcançar além do que achávamos possível.
Além disso, pude usufruir desse suporte ainda na pós-graduação no Observatório
Nacional (seção~\ref{sec_on}), tanto por conta de vários membros da turma
estarem trabalhando no Rio de Janeiro, quanto por ter meu amigo
\href{https://www.pinga-lab.org/people/oliveira-jr.html}{Vanderlei C. Oliveira Jr.}
comigo na pós-graduação (Vanderlei é atualmente Pesquisador Titular do
Observatório Nacional).
Também tive muita sorte no meu acesso a mentores excelentes:
Manoel S. D'Agrella Filho, Ricardo I. F. Trindade e Naomi Ussami
durante a graduação, Valéria C. F. Barbosa e Carla Braitenberg durante a
pós-graduação e Paul Wessel durante o pós-doutorado.

Todos os fatores descritos acima me proporcionaram acesso diferenciado a
oportunidades e vantagens para conquistá-las.
Porém, um fator que considero de mérito próprio é que tive a perspicácia para
identificar essas oportunidades quando elas se apresentaram, a confiança para
aplicar e a perseverança para usufruir ao máximo de minhas conquistas.

\section{A estrutura deste memorial}

Identificar uma estrutura coerente  para este memorial que minimizasse a
sobreposição de informação entre os capítulos foi uma tarefa desafiadora.
A minha formação, atividades de ensino e pesquisa e, principalmente, minha
atuação na área de software livre estão todas intrinsecamente ligadas.
A estrutura que concebi começa pela minha formação acadêmica no
capítulo~\ref{cap_formacao} e atuação profissional no
capítulo~\ref{cap_atuacao}.
Em seguida, dividi minhas atividades acadêmicas
entre ciência aberta (capítulo~\ref{cap_cienciaaberta}),
linhas de pesquisa (capítulo~\ref{cap_pesquisa})
e ensino e extensão (capítulo~\ref{cap_ensino}).
Algumas informações estão necessariamente repetidas entre alguns capítulos,
por exemplo o software \href{https://www.fatiando.org}{Fatiando a Terra}
é discutido em quase todos os capítulos em diferentes contextos.
Finalmente, apresento considerações finais no capítulo~\ref{cap_conclusao}.


%==============================================================================
\chapter{Formação Acadêmica}
\label{cap_formacao}

\begin{figure}[h]
  \HeroFigPad
  \begin{center}
    \includegraphics[width=\textwidth]{images/vassouras-geomag-observation-2012.jpg}
  \end{center}
  \caption{
    Realizando medidas da direção do campo geomagnético no observatório de
    Vassouras, Rio de Janeiro. A atividade foi parte de uma disciplina de
    instrumentação geofísica que cursei durante a pós-graduação do Observatório
    Nacional.
  }
\end{figure}
\begin{summarybox}[frametitle=\faInfoCircle{}\quad Resumo da formação acadêmica]
  \begin{datelist}
    2004--2009 & Bacharelado em Geofísica -- Universidade de São Paulo \\
    2008--2009 & \faPlane{} Intercâmbio Internacional -- York University, Canadá \\
    2010--2011 & Mestrado em Geofísica -- Observatório Nacional \\
    2011--2016 & Doutorado em Geofísica -- Observatório Nacional
  \end{datelist}
\end{summarybox}

Este capítulo relata a minha formação acadêmica, do Bacharelado ao Doutorado,
refletindo sobre os fatores que influenciaram minhas linhas de pesquisa e o
rumo que tomei durante minha carreira.

\section{Universidade de São Paulo}
\label{sec_usp}

\begin{subsummarybox}[frametitle=\faGraduationCap{}\quad Bacharelado em Geofísica]
  \begin{fa-ul}
    \faUniversity & Universidade de São Paulo \\
    \faCalendar & Fevereiro 2004 -- Novembro 2009 \\
    \faUser & Orientadora: Naomi Ussami\\
    \faInfoCircle & Trabalho de conclusão: Cálculo do tensor gradiente
    gravimétrico utilizando tesseroides (\DOI{10.6084/m9.figshare.963547})
  \end{fa-ul}
\end{subsummarybox}

Ingressei no curso de Bacharelado em Geofísica da Universidade de São Paulo em
2004.
Já no primeiro semestre, o curso desafiou diversos de meus preconceitos sobre
os assuntos abordados.
Uma das experiências mais marcantes foi a disciplina MAC0115 ``Introdução à
Computação para Ciências Exatas e Tecnologia''.
Minha expectativa era aprender sobre funções avançadas de softwares como o
Microsoft Office, talvez aprender sobre algum programa específico para a
geofísica.
Jamais havia imaginado que como parte do meu curso de Geofísica eu aprenderia
como criar meus próprios programas, mas foi exatamente isso que aprendemos
nessa disciplina que foi ministrada de maneira excepcional.
Minha carreira com certeza teria tomado um rumo completamente diferente se
minha primeira experiência com a programação não houvesse sido tão positiva.
Aprendi os conceitos básicos da linguagem de programação C e, junto com meu
amigo \href{https://www.linkedin.com/in/balancin/}{Lucas Balancin}, resolvi
todos os exercícios fornecidos para estudo da disciplina.
Porém, não alcancei um nível suficientemente avançado para enxergar aplicações
imediatas da programação nas demais disciplinas do curso.

Busquei aprender mais sobre a programação através da disciplina optativa
AGG0204 ``Computação para Geofísicos''.
Durante a disciplina, desenvolvi aplicações diretas da programação à geofísica
como o cálculo do International Geomagnetic Reference Field (IGRF) a partir dos
coeficientes de harmônicos esféricos.
Essas aplicações me mostraram o enorme poder da programação no aprendizado de
conceitos complexos da geofísica e da matemática.
Ao criar uma implementação computacional de um método, fui levado a considerar
detalhes e a elaborar perguntas que me passariam despercebidas ao estudar
somente pela teoria.
Além disso, também fui capaz de explorar as possibilidades e os limites de uma
teoria de forma dinâmica e independente.

Nos anos seguintes continuei a estudar programação por conta própria nas horas
vagas e a aplicar à geofísica o que estava aprendendo.
Aprendi como programar nas linguagens Java, C++ e Python (por recomendação do
então aluno de mestrado \href{https://www.linkedin.com/in/fspaolo/}{Fernando Paolo}).
Implementei a Transformada Discreta de Fourier\footnote{Disponível em
\url{https://github.com/leouieda/dft-in-c}} para estudar para a disciplina
AGG0330 ``Processamento de Sinais Digitais''.
Utilizei uma implementação do método \textit{Ant Colony Optimization}
\citep{Socha2008}, que fiz por curiosidade própria, para realizar uma inversão
de velocidades de grupo de ondas
Love\footnote{Disponível em \url{https://github.com/leouieda/love-aco-inv}}
como meu projeto para a disciplina AGG0305 ``Teoria de Ondas Sísmicas e
Estrutura da Terra''.
Cursei a disciplina optativa MAC0122 ``Princípios de Desenvolvimento de
Algoritmos'' onde aprendi os conceitos de estruturas de dados e recursividade
que possibilitaram alguns dos avanços que obtive em \citet{Uieda2016}
(seção~\ref{sec_tesseroids}).

O curso também me forneceu treinamento excepcional em quase todos os métodos de
geofísica.
Tivemos experiências de campo e utilizamos uma ampla variedade de equipamentos
geofísicos.
A junção da base teórica sólida com essa experiência prática foi
extremamente motivante para alunos como eu, que estavam indecisos sobre suas
carreiras e sobre qual rumo seguir após a graduação.

Refletindo agora, quase 19 anos após ingressar na USP, percebo o quão sólida
foi a base que adquiri durante a graduação. Utilizo os conceitos que aprendi
nas disciplinas de computação, álgebra linear, física e métodos potenciais
diariamente. Tendo passado por cinco outras instituições no Brasil e no
exterior, reconheço o quão raro é um curso preparar tão bem seus alunos.
Por isso, sou muito grato a todos os meus professores e ao país por me dar
acesso a essa educação de forma gratuita (outra raridade, principalmente no
exterior).

\subsection{Iniciação científica: Paleomagnetismo}
\label{sec_ic_paleomag}

Durante meu segundo ano de graduação, iniciei um projeto de iniciação
científica com o Professor Manoel Souza D'Agrella Filho.
O objetivo do trabalho era obter um paleo-pólo geomagnético para um conjunto
de diques de idade cambriana da região de Maravilhas, Paraíba.
O projeto intitulado ``Paleomagnetismo e mineralogia magnética dos diques
cambrianos de Maravilhas e Prata (PB)'' foi apoiado por uma bolsa da
FAPESP\footnote{Mais informações em
\url{https://bv.fapesp.br/pt/bolsas/73578/paleomagnetismo-e-mineralogia-magnetica-dos-diques-cambrianos-de-maravilhas-e-prata-pb}}
por um ano.
O trabalho incluiu uma expedição para amostrar novos diques na região de
Monteiro, Paraíba, liderado pelo Professor Ricardo I. F. Trindade.
Os resultados foram apresentados em um poster no XI Simpósio de Iniciação
Científica do IAG/USP \citep{Uieda2006}.
Essa foi a primeira vez que participei de um projeto de pesquisa e apresentei
um poster.
Sou muito grato ao Manuel e o Ricardo pela oportunidade de aprender mais sobre
o paleomagnetismo e pelas experiências de laboratório e de campo.
Percebi com esse projeto que, embora os resultados e sua interpretação tenham
sido muito interessantes, a rotina de laboratório não era algo que eu
conseguiria manter a longo prazo.
Ao mesmo tempo, estava cada vez mais interessado na computação e modelagem
numérica.
Isso me levou a buscar outra área para continuar minha iniciação científica e
trabalho de conclusão de curso.
Mesmo assim, o paleomagnetismo ainda é um assunto que me interessa muito.
Tanto que, 16 anos depois dessa primeira iniciação científica, estou retornando
ao assunto com uma nova linha de pesquisa em microscopia magnética em
colaboração com o Ricardo (seção~\ref{sec_micromag}).

\subsection{Iniciação científica: Gravimetria e computação}
\label{sec_ic_tesseroids}

No final de 2007, durante meu terceiro ano de graduação, me juntei ao grupo da
Professora Naomi Ussami para trabalhar em um projeto que abordava os temas que
mais me interessavam naquele momento: computação, modelagem numérica e
gravimetria.
O projeto intitulado ``Modelagem gravimétrica de corpos tesseroidais'' foi
executado com uma bolsa da
SBGf\footnote{Mais informações em \url{https://sbgf.org.br/programa_ic}}
e em colaboração com a Professora Carla Braitenberg da , Itália.
Nosso objetivo era desenvolver um software que pudesse calcular campos
gravitacionais causado por segmentos de uma esfera (\textit{tesseroides}).
Esse programa seria utilizado para trabalhar com dados da futura missão de
satélite \href{https://www.esa.int/Enabling_Support/Operations/GOCE}{GOCE},
tanto na fase inicial de avaliação de sua sensibilidade a diferentes estruturas
geológicas quanto na fase de processamento e modelagem dos dados obtidos.
Durante as fases iniciais desse projeto, contei com o auxílio da Dra.
Franziska Wild-Pfeiffer, cujo artigo \citep{WildPfeiffer2008} eu estava
tentando reproduzir.
Apresentei meus resultados iniciais no XIII Simpósio de Iniciação Científica do
IAG/USP \citep{Uieda2008}.
No final de 2009, concluí o Bacharelado defendendo o trabalho de conclusão de
curso intitulado ``Cálculo do tensor gradiente gravimétrico utilizando
tesseroides''\footnote{Disponível em \url{https://doi.org/10.6084/m9.figshare.963547}}.

Este trabalho marcou a primeira versão do software Tesseroids
(seção~\ref{sec_tesseroids}), desenvolvido inicialmente na linguagem Python,
e o início de uma linha de pesquisa que abrangeu minha pós-graduação e primeira
coorientação de um aluno de Doutorado (seção~\ref{sec_modelagemdireta}).

\section{York University}
\label{sec_york}

\begin{subsummarybox}[frametitle=\faPlane{}\quad Intercâmbio internacional]
  \begin{fa-ul}
    \faUniversity & York University, Canadá\\
    \faCalendar & Agosto 2008 -- Maio 2009
  \end{fa-ul}
\end{subsummarybox}

Tive a vontade de fazer um intercâmbio no exterior desde o início do curso de
graduação.
Rotineiramente vasculhava as diversas oportunidades divulgadas pela
universidade por uma que oferecesse cursos de Ciências da Terra.
Uma das primeiras que encontrei foi a \href{https://www.yorku.ca/}{York University},
Canadá, cujo curso de Ciências da Terra oferecia diversas disciplinas que
complementariam minha formação na USP, principalmente na área de geodésia.
Me inscrevi no processo seletivo interno da USP para concorrer a uma única vaga
que estava sendo ofertada para alunos de todos os cursos da universidade.
Felizmente fui selecionado e me mudei para Toronto, Canadá, em Agosto de 2008.

Tive uma surpresa ao chegar na York e me apresentar na secretaria de graduação:
o curso de Ciências da Terra havia sido descontinuado no ano anterior por causa
do baixo número de alunos inscritos.
Aparentemente, a página online do curso não havia sido atualizada e por isso
eu baseei meu plano de estudos para o ano em curso inexistente.
Por sorte, a maioria das disciplinas que eu havia escolhido cursar ainda
seriam oferecidas como parte de outros cursos.
Os meus estudos acabaram não sendo tão afetados mas minha experiência não foi
como eu esperava por não ter uma turma de alunos de geociências cursando as
mesmas disciplinas, como era o caso na USP.

Durante minha estadia na York, aprendi sobre sistemas geográficos de
coordenadas, posicionamento, ajustes de redes geodésicas, geodésia física e
levantamentos gravimétricos de alta precisão.
Um destaque dessa experiência foram as aulas do Professor
\href{https://www.yorku.ca/spiros/spiros.html}{Spiros Pagiatakis}.
Suas aulas de geodésia e matemática forneceram a clareza que me faltava nos
conceitos de anomalias da gravidade e a solução prática de problemas inversos
em geofísica.
Observando outros alunos presentes na disciplina, considero que somente pude
aproveitar plenamente essas aulas graças à base sólida que tinha obtido em
outras disciplinas da USP.

Meu tempo em Toronto foi excelente para meu crescimento pessoal, cultural e
acadêmico.
Fiz amizade com pessoas de todos os cantos do planeta (Europeus, Asiáticos,
Canadenses) com os quais mantenho contato até hoje.
O conhecimento que adquiri nas disciplinas me possibilitaram começar a
trabalhar diretamente no meu projeto de Mestrado pois já possuía grande parte
da base teórica e experiência prática computacional necessária.
Por isso, fui capaz de desenvolver um método novo em pouco tempo.

\section{Observatório Nacional}
\label{sec_on}

\begin{subsummarybox}[frametitle=\faGraduationCap{}\quad Mestrado em Geofísica]
  \begin{fa-ul}
    \faUniversity & Observatório Nacional \\
    \faCalendar & Fevereiro de 2010 -- Outubro de 2011 \\
    \faUser & Orientadora:  Valéria C. F. Barbosa\\
    \faInfoCircle & Dissertação: Robust 3D gravity gradient inversion by
    planting anomalous densities (\DOI{10.6084/m9.figshare.16882300})
  \end{fa-ul}
\end{subsummarybox}
\begin{subsummarybox}[frametitle=\faGraduationCap{}\quad Doutorado em Geofísica]
  \begin{fa-ul}
    \faUniversity & Observatório Nacional \\
    \faCalendar & Novembro de 2011 -- Abril de 2016 \\
    \faUser & Orientadora:  Valéria C. F. Barbosa\\
    \faInfoCircle & Tese Modelagem direta e inversão de campos gravitacionais em
    coordenadas esféricas (\DOI{10.6084/m9.figshare.16883689}) \\
    \faTrophy & Ganhador do Prêmio SBGf de Melhor Tese de Doutorado (2015--2017)\footnotemark
  \end{fa-ul}
\end{subsummarybox}
\footnotetext{Mais informações em \url{https://sbgf.org.br/premiacoes}}

Minha ida para o Canadá durante a graduação fez com que eu atrasasse minha
formatura em um ano.
Ao retornar, comecei a explorar as opções do que fazer após terminar a
graduação.
Após conversar com meus amigos que já estavam formados e trabalhando em
empresas voltadas à indústria do petróleo no Rio de Janeiro, cheguei à
conclusão de que ainda gostaria de continuar meus estudos e expandir minhas
atividades de pesquisa.
Minha experiência no Canadá me mostrou o quão benéfico é a exposição a uma
diversidade de formas de pensamento que se obtém em diferentes instituições.
Por isso, após cinco anos na USP, decidi que estava na hora de buscar uma
pós-graduação em outra instituição no Brasil.

O Observatório Nacional (ON) já havia despertado meu interesse após uma visita
que fizemos à instituição em 2007 durante uma viagem de nossa turma de
graduação para participar do Congresso Internacional da Sociedade Brasileira de
Geofísica.
Além disso, meu amigo e colega de turma
\href{https://www.pinga-lab.org/people/oliveira-jr.html}{Vanderlei C. Oliveira Jr.}
já havia se formado e estava cursando o Mestrado em Geofísica do Observatório
Nacional (ON) sob supervisão da Professora
\href{https://www.pinga-lab.org/people/barbosa.html}{Valéria C. F. Barbosa}.
Após uma visita ao Rio de Janeiro em 2009, o Vanderlei me convenceu (sem muito
esforço) a me inscrever no Mestrado do ON ao terminar a graduação na USP.
Ele também convenceu a Valéria a me orientar, o que considero ser um dos
maiores favores que um amigo jamais me fez.
Sou eternamente grato ao Vanderlei pela recomendação e à Valéria por aceitar me
orientar.

O ambiente da pós-graduação do ON era extremamente produtivo e estimulante.
As salas misturavam alunos dos diversos grupos de pesquisa da astronomia e
geofísica, facilitando o intercâmbio de ideias entre os alunos.
Por exemplo, aprendi muito sobre o processamento de dados sísmicos e de GPR
ajudando meu amigo e colega de sala
\href{https://www.linkedin.com/in/saulo-siqueira-martins-78770878/}{Saulo Siqueira Martins}
(atualmente Professor de Geofísica da Universidade Federal do Pará)
a utilizar o software \href{https://www.reproducibility.org/}{Madagascar}.
Esse conhecimento foi extremamente útil nas minhas atividades de ensino na
(capítulo~\ref{cap_ensino}).

A pós-graduação também me forneceu diversas oportunidades de frequentar
congressos internacionais com financiamento da CAPES e de projetos da Valéria.
Essas participações me ajudaram a estabelecer contatos e criar uma rede de
apoio e colaboração internacional.
Por exemplo, os contatos que fiz no congresso
\href{https://conference.scipy.org/scipy2014/}{Scipy} de 2013 e 2014 levaram a
minha participação na diretoria do
\href{https://softwareunderground.org/}{Software Underground}, a organização
de seções em congressos e colaborações com os desenvolvedores do software
\href{https://simpeg.xyz/}{SimPEG}.
O incentivo e a liberdade de escolher meus temas de pesquisa dados pela Valéria
sempre me motivaram a dar o melhor de mim.
Não exagero quando afirmo que conhecer a Valéria foi o acontecimento mais
influente na minha carreira.

Durante a pós-graduação, continuei a perseguir meu interesse na programação,
no software livre e na ciência aberta.
Aprendi como usar o sistema de controle de versão
\href{https://git-scm.com/}{git} e a plataforma
\href{https://github.com}{GitHub} e como criar páginas da internet com HTML e
CSS.
Continuei o desenvolvimento do software \href{https://tesseroids.leouieda.com/}{Tesseroids}
e criei o projeto \href{https://www.fatiando.org}{Fatiando a Terra}
(seção~\ref{sec_fatiando}) junto com alguns colegas da graduação, incluindo o
Vanderlei.
O investimento inicial que fiz na qualidade do código do Fatiando me
permitiu terminar meu projeto de Mestrado em apenas 18 meses,
concluir minha tese de Doutorado enquanto já trabalhava como Professor
Assistente na UERJ (seção~\ref{sec_uerj}) e elaborar aulas interativas sobre
geofísica para meus alunos de geologia.

Em meados de 2013, eu, o Vanderlei e a Valéria iniciamos o grupo de
\href{https://www.pinga-lab.org/}{\textbf{P}roblemas \textbf{In}versos em \textbf{G}eofísic\textbf{a}}
(PINGA).
A conta do grupo no GitHub\footnote{Disponível em \url{https://github.com/pinga-lab}}
agrega os repositórios com o código fonte para reproduzir as publicações do
grupo.
O grupo também conta com uma página na internet\footnote{Página do grupo PINGA: \url{https://www.pinga-lab.org}},
feita em grande parte por mim\footnote{Sou o maior contribuidor em termos de
linhas de código geradas:
\url{https://github.com/pinga-lab/website/graphs/contributors}.},
onde divulgamos as teses, artigos, projetos e integrantes do grupo.

\subsection{Mestrado}

Meu projeto de mestrado era adaptar o método desenvolvido
pela Valéria e seu ex-aluno
\href{https://www.researchgate.net/profile/Fernando-Dias-8}{Fernando Silva Dias}
\citep{SilvaDias2009} para inverter dados de gradiente da gravidade.
Na época, esse tipo de dado estava começando a ser utilizado na área de
recursos minerais mas ainda havia uma falta de métodos de inversão 3D para sua
interpretação.
O projeto estava atrelado ao projeto de Doutorado do aluno
\href{https://www.linkedin.com/in/dionisio-uendro-carlos-093671225/}{Dionisio Uendro Carlos},
que iria utilizar o método desenvolvido por mim para interpretar dados
fornecidos pela empresa \href{https://vale.com/}{Vale}.

A abordagem que eu preferia (e prefiro até hoje) para compreender um assunto
novo é fazer por conta própria a implementação computacional de todos os
conceitos básicos e reproduzir resultados existentes.
Logo, comecei meu Mestrado implementando novamente as rotinas básicas
necessárias para realizar a inversão: o método de modelagem direta de
\citet{Nagy2000}, a geração de dados sintéticos, a solução de problemas
inversos lineares com regularização e a visualização em 3D dos modelos.
Esse código formou a base do projeto
\href{https://www.fatiando.org}{Fatiando a Terra} e ainda sobrevive em partes
de sua encarnação atual (seção~\ref{sec_fatiando}).
Minha vontade era encontrar uma abordagem nova, ao invés de simplesmente seguir
o que já havia sido feito em \citet{SilvaDias2009}.
Sendo uma orientadora consciente, a Valéria corretamente me deu somente até o
final de meu primeiro ano para explorar diferentes opções.
Caso não fosse capaz de desenvolver um método novo, combinamos que eu faria o
projeto inicialmente proposto.
Tendo esse prazo em mente, trabalhei incessantemente durante o ano de 2010 para
desenvolver uma abordagem nova de inversão.

Minha grande descoberta veio quando me deparei com o trabalho de
\citet{Rene1986}.
Este trabalho relativamente desconhecido propôs um método de inversão 2D de
dados de gravidade pouco convencional.
Seu método adiciona elementos iterativamente à solução em torno de ``sementes''
e evita a solução de sistemas lineares, um dos grandes empecilhos
computacionais para a inversão 3D.
Porém, esse trabalho não explorou completamente as vantagens que o conceito de
construir a solução iterativamente possibilitava.
Baseado nas ideias de \citet{Rene1986}, criei um método capaz de inverter de
maneira conjunta dados de gravimetria tradicional e gradiometria gravimétrica
em três dimensões.
Adicionei diversas inovações ao método para torná-lo viável a modelos da ordem
de milhões de elementos e melhor controlar a forma do modelo final.
Essas inovações são resultado direto do meu interesse pela computação, podendo
ser rastreadas às disciplinas que cursei ainda na graduação.
O resultado foi publicado em meu primeiro artigo \citep{Uieda2012}, que formou minha
dissertação e foi apresentado nos congressos internacionais da
Society of Exploration Geophysicists,
European Association of Geoscientists and Engineers,
e Sociedade Brasileira de Geofísica
(seção~\ref{sec_planting}).
Esse artigo também foi meu primeiro experimento em ciência aberta.
Todo o código para produzir os resultados e figuras do artigo foi publicado
em um repositório do GitHub\footnote{Disponível em
\url{https://github.com/pinga-lab/paper-planting-densities}} e o material
suplementar foi publicado no
\href{https://figshare.com/}{figshare}\footnote{Disponíveis em
\url{https://doi.org/10.6084/m9.figshare.91574} e
\url{https://doi.org/10.6084/m9.figshare.91469}}.

Terminei meu mestrado em Outubro de 2011 (quatro meses adiantado) e ingressei
no Doutorado em Geofísica do Observatório Nacional, ainda sob supervisão da
Valéria, logo em seguida.

\subsection{Viagem para Trieste}
\label{sec_triste_carla}

Em 2011, ainda no Mestrado, fui convidado pela Professora Carla Braitenberg
para passar um mês na  Itália, para continuar
o desenvolvimento do software Tesseroids.
Passei Fevereiro de 2011 trabalhando com ela em uma nova versão do software
escrito em linguagem C.
Na época, produzir um software numérico em Python que pudesse alcançar a
performance de programas escritos em C não era uma tarefa fácil.
Por isso, decidimos que a melhor alternativa seria reescrever o software em C.
Essa nova versão mais eficiente do programa seria necessária para o
processamento de dados do satélite GOCE que o grupo de Trieste almejava fazer.
Durante minha estadia em Trieste, reescrevi o software na linguagem C, criei
uma página para a documentação\footnote{Disponível em \url{https://tesseroids.leouieda.com}}
e desenvolvi um algoritmo de discretização adaptativa para combater o problema
de estabilidade numérica do método\footnote{O \textit{commit} 0af974f
introduziu a discretização adaptativa de tesseroides em 11 de Fevereiro de
2011:
\url{https://github.com/leouieda/tesseroids/commit/0af974f26a15f98f1072ccc6c4ebf29588863f51}}.

\subsection{Doutorado}
\label{sec_doutorado}

Meu projeto de Doutorado era desenvolver métodos para inversão de dados de
gravidade 3D em uma aproximação esférica da Terra, combinando assim os temas
do meu trabalho de conclusão de curso de graduação e dissertação de Mestrado.
A aproximação esférica é necessária para a modelagem em escala continental e
global.
Também decidimos que o desenvolvimento dos softwares Tesseroids e Fatiando a
Terra seriam parte dos objetivos principais da tese.
Esses programas seriam os principais ``produtos'' gerados pelo meu Doutorado
para a comunidade científica.

Os dois primeiros anos do meu Doutorado foram dedicados ao desenvolvimento dos
programas e à colaborações com outros membros do recém-formado
\href{https://www.pinga-lab.org}{PINGA}.
Participei da concepção, execução e escrita dos trabalhos
\citet{OliveiraJr2013}, \citet{Melo2013}, \citet{Carlos2014},
\citet{OliveiraJr2015} e \citet{Carlos2016}.
Expandi a gama de funções disponíveis no Fatiando a Terra\footnote{Ver lista
de mudanças nas versões v0.1 e v0.2 em \url{https://legacy.fatiando.org/changelog.html}}
e apresentei meus trabalhos em diversos congressos internacionais.

No final de 2013, me inscrevi e fui aprovado no concurso público para a vaga de
Professor Assistente no Departamento de Geologia Aplicada da .
Entre 2014 e 2016, exerci minhas tarefas de docente da UERJ enquanto terminava
os trabalhos \citet{Uieda2016} e \citet{Uieda2017}.
Esses dois anos foram muito desafiadores, principalmente no período de
adaptação ao meu novo cargo de Professor.
Graças ao investimos que havia feito no Fatiando a Terra nos quatro anos
anteriores, fui capaz de desenvolver, aplicar e publicar o método descrito em
\citet{Uieda2017} durante o pouco tempo vago que tive em 2015 e
2016\footnote{O primeiro \textit{commit} do repositório do GitHub do artigo é
de Março de 2015: \url{https://github.com/pinga-lab/paper-moho-inversion-tesseroids/commit/edd0e33a200bd1946be0020a38d1d362d93f2c36}}.
Em Abril de 2016, defendi minha tese de Doutorado intitulada ``Modelagem direta
e inversão de campos gravitacionais em coordenadas esféricas'', composta pelos
trabalhos \citet{Uieda2013}, \citet{Uieda2016} e \citet{Uieda2017}.
Fui ganhador do Prêmio SBGf de Melhor Tese de Doutorado
(2015--2017)\footnote{Mais informações em \url{https://sbgf.org.br/premiacoes}}
e esses três artigos estão entre meus trabalhos com maior número de
citações\footnote{Segundo o Google Scholar em 27/12/2022:
\url{https://scholar.google.com/citations?user=qfmPrUEAAAAJ&hl=en}}.


\section{Formação complementar em pedagogia}

Minha formação no Bacharelado, Mestrado e Doutorado me prepararam bem para uma
carreira de pesquisa.
Porém, senti que ainda havia lacunas no meu treinamento, principalmente na área
de ensino.
Busquei preencher essas lacunas através dos cursos complementares em técnicas
práticas de ensino e teoria pedagógica descritos abaixo.

\subsection{Software Carpentry}
\label{sec_swcarpentry}

\begin{subsummarybox}[frametitle=\faGraduationCap{}\quad The Carpentries Instructor Training]
  \begin{fa-ul}
    \faUniversity & \href{https://carpentries.org/}{The Carpentries} \\
    \faCalendar & 9--10 de Julho de 2018\\
    \faInfoCircle & Habilitação para organizar e ministrar os cursos
    \textit{Software Carpentry}, \textit{Data Carpentry} e
    \textit{Library Carpentry}, incluindo treinamento em pedagogia e práticas
    de ensino de programação e ciência de dados
  \end{fa-ul}
\end{subsummarybox}

Me deparei com o \href{https://software-carpentry.org/}{Software Carpentry}
em 2008 durante meu intercâmbio na York University.
Na época, a organização consistia de uma página na internet com informações
para treinamento de cientistas em técnicas de engenharia de
software\footnote{Infelizmente, a versão do material de 2008 só está disponível
no repositório \url{https://github.com/swcarpentry/v3}}.
Esse material abriu meus olhos para o mundo da engenharia de software que ia
muito além das disciplinas de programação que cursei na USP durante minha
graduação.
Passei grande parte do meu tempo livre durante os meses de inverno no Canadá
imerso no Software Carpentry, aprendendo sobre o sistema de controle de versão
\href{https://subversion.apache.org/}{subversion} (precursor do
\href{https://git-scm.com/}{git}), testes unitários, programação defensiva,
automatização com o \href{https://www.gnu.org/software/make/}{GNU Make},
expressões regulares, programação em
\href{https://www.gnu.org/software/bash/}{bash}, entre outros.
Busquei aplicar esses conceitos novos imediatamente, tanto para as tarefas das
disciplinas que estava cursando quanto para meu trabalho de conclusão de curso
e para o programa Tesseroids.
Utilizo todas a lições que aprendi com o Software Carpentry diariamente na
minha pesquisa, ensino e até mesmo para escrever esse memorial (que está
armazenado em um repositório privado no GitHub e utiliza o Make para compilação
do código \LaTeX{}).

Atualmente, o Software Carpentry é parte da organização sem fins lucrativos
\href{https://carpentries.org/}{The Carpentries}, que promove
internacionalmente cursos de curta duração em engenharia de software para
cientistas.
Os cursos são ministrados, e frequentemente organizados, voluntariamente por
instrutores credenciados.
Em 2018, realizei o curso de habilitação de instrutores do The Carpentries e me
tornei um instrutor
credenciado\footnote{Mais informações em \url{https://carpentries.org/instructors/\#leouieda}}.
O curso cobre técnicas para ensino de programação baseadas em evidências da
literatura pedagógica \citep[resumidas em][]{Brown2018}.
A habilitação me permite organizar e ministrar cursos oficiais do The
Carpentries.

Utilizo as técnicas aprendidas tanto em minhas aulas de programação em Python
como nas aulas de geofísica que possuem uma componente computacional
(capítulo~\ref{cap_ensino}), que são a grande maioria das aulas que dou
atualmente.
A experiência que tive com o uso eficaz de tecnologias para ensino virtual que
foram utilizadas nas etapas finais do curso (Zoom, Google Docs, etc.) foram
extremamente valiosas durante a transição para o ensino online causada pela
pandemia de COVID em 2020 e 2021.


\subsection{Pedagogia no ensino superior}
\label{sec_pgcap}

\begin{subsummarybox}[frametitle=\faGraduationCap{}\quad Postgraduate Certificate in Academic Practice]
  \begin{fa-ul}
    \faUniversity & Universidade de Liverpool \\
    \faCalendar & Novembro de 2020 -- Maio de 2022 \\
    \faInfoCircle & Curso de pós-graduação em pedagogia no ensino superior que
    me confere o título de \textit{Fellow of the Higher Education
    Academy} (número de referência PR242069)
  \end{fa-ul}
\end{subsummarybox}

Durante meu segundo ano em Liverpool, realizei o curso de pós-graduação
Postgraduate Certificate in Academic Practice (PGCAP) oferecido pela Faculty
of Humanities and Social Sciences da universidade.
A conclusão do PGCAP em 2022 me conferiu o título de \textit{Fellow of the
Higher Education Academy}\footnote{Mais informações em
\url{https://www.advance-he.ac.uk/fellowship/fellowship}},
que é necessário para progressão na carreira acadêmica nas instituições da
Inglaterra.
O curso foi divido em duas partes:
a primeira composta de aulas sobre teoria pedagógica aplicada ao ensino
superior e a segunda composta de um projeto de pesquisa ou revisão em
pedagogia.

Meu projeto para a segunda parte do curso foi uma revisão bibliográfica sobre
a técnica de observação por pares aplicada ao ensino superior
\citep{Cosh1998,Fletcher2018,OKeeffe2021}.
Observação por pares se refere a diversas técnicas que envolvem professores
assistirem e revisarem aulas de outros professores.
Resolvi abordar esse tema após realizar uma sessão de observação por pares
durante a primeira parte do curso.
De todas as atividades que fizemos no PGCAP, essa foi a que mais me beneficiou
e me pareceu ter o maior potencial para difundir boas práticas pedagógicas
entre os professores.
Minha revisão bibliográfica e demais reflexões e notas do curso estão
disponíveis em \url{https://www.leouieda.com/pgcap}.


%==============================================================================
\chapter{Atuação Profissional}
\label{cap_atuacao}

\begin{figure}[h]
  \HeroFigPad
  \begin{center}
    \includegraphics[width=\textwidth]{images/liverpool-gdsl.jpg}
  \end{center}
  \caption{
    Foto de uma apresentação que fiz para o \textit{Geographic Data Science
    Lab} da Universidade de Liverpool em Março de 2020. O propósito da palestra
    foi me apresentar para o grupo pouco após minha chegada em Liverpool e
    tentar estabelecer temas para colaborações futuras.
  }
\end{figure}
\begin{summarybox}[frametitle=\faInfoCircle{}\quad Resumo da atuação profissional]
  \renewcommand{\thempfootnote}{$\dagger$}
  \begin{datelist}
    \textbf{2014--2018} & \textbf{Professor Assistente -- } \\
    \textbf{2017--2019} & \textbf{Pesquisador Visitante -- , E.U.A.} \\
    2019--2022 & Topic Editor\footnote{Posição não remunerada} -- Journal of Open Source Software (voluntário) \\
    2019--2022 & Advisory Council Member\mpfootnotemark[\value{mpfootnote}] -- EarthArXiv \\
    \textbf{2019--atual} & \textbf{Lecturer (\textit{Professor Doutor}) -- University of Liverpool, Reino Unido} \\
    2020--atual & Fellow\mpfootnotemark[\value{mpfootnote}] -- Software Sustainability Institute \\
    2022--atual & Board Member\mpfootnotemark[\value{mpfootnote}] -- Software Underground \\
    2022--atual & Advisory Committee Member\mpfootnotemark[\value{mpfootnote}] -- pyOpenSci
  \end{datelist}
\end{summarybox}

Este capítulo relata minha atuação profissional, tanto como funcionário em
instituições de ensino superior, quanto como voluntário em posições de
liderança em organizações sem fins lucrativos que servem a comunidade
científica.
Os relatos abaixo se referem somente à atividades institucionais e experiências
pessoais.
Minhas linhas de pesquisa (incluindo a orientação de alunos) e atividades de
ensino e extensão serão discutidas nos capítulos~\ref{cap_pesquisa},
\ref{cap_ensino}, respectivamente.


\section{\UFF}
\label{sec_uerj}

\begin{subsummarybox}[frametitle=\faUniversity{}\quad Vínculo institucional]
  \begin{fa-ul}
    \faUser & Professor Assistente \\
    \faMapMarker & Departamento de Geologia Aplicada -- Faculdade de Geologia \\
    \faCalendar & Fevereiro 2014 -- Janeiro 2018\footnotemark{} \\
    \faTrophy & Paraninfo da turma de formandos da Geologia (ano de ingresso 2012)
  \end{fa-ul}
\end{subsummarybox}
\footnotetext{Afastado entre Fevereiro de 2017 e Janeiro de 2018 para trabalhar na }
\begin{subsummarybox}[frametitle=\faList{}\quad Atividades institucionais]
  \begin{datelist}
    2014--2017 & Coordenador: Laboratório de Geofísica de Exploração (LAGEX) \\
    2014--2017 & Coordenador: Projeto Qualitec para contratação de um bolsista de nível superior para atuar no LAGEX \\
    2015--2017 & \textit{Faculty Advisor}: Capítulo Estudantil da Society of Exploration Geophysicists (\textit{UERJ Geophysical Society}) \\
    2015 & Representante docente titular da sub-comissão eleitoral da Faculdade de Geologia
  \end{datelist}
\end{subsummarybox}

No final de 2013, durante meu segundo ano do Doutorado, surgiu a oportunidade
de prestar o concurso público para cargo de Professor Assistente na .
Somente era necessário o título de Mestre e o concurso era para a área de
Geofísica.
Por recomendação da minha orientadora Valéria C. F. Barbosa, decidi prestar o
concurso pois seria uma excelente oportunidade para iniciar uma carreira
acadêmica antes mesmo de terminar meu Doutorado.
Felizmente, fui aprovado em primeiro lugar no concurso e tomei posse do cargo
de Professor Assistente na UERJ em Fevereiro de 2014.
De início, assumi a posição de coordenador do Laboratório de Geofísica de
Exploração (LAGEX) e fui responsável por duas novas disciplinas de geofísica
do Bacharelado em Geologia e outras disciplinas do Bacharelado em Oceanografia
(seção~\ref{sec_ensino_grad}).

Na coordenação do LAGEX, liderei nossa aplicação para uma chamada de projetos
interna da UERJ (QUALITEC) que forneceria financiamento para a
contratação de um bolsista de nível superior por quatro anos.
Nossa aplicação foi bem sucedida e no final de 2014 nomeei o
\href{https://www.linkedin.com/in/victorxalmeida/}{Victor Thadeu Xavier de Almeida}
para assumir a bolsa.
O Victor era responsável por manter os computadores GNU/Linux do LAGEX,
por auxiliar no ensino de disciplinas de graduação que utilizavam o
laboratório e também por contribuir com o desenvolvimento do Fatiando a Terra.
Ter o Victor no LAGEX durante minha estadia foi excelente e elevou minhas
contribuições de ensino, pesquisa e desenvolvimento do Fatiando.

Ainda em 2014, trabalhei com os alunos
\href{https://www.linkedin.com/in/caroline-adolphsson-61723137/}{Caroline Adolphsson Nascimento}
e \href{https://www.linkedin.com/in/gustavo-pereira-780839111/}{Gustavo do Couto Ramos Pereira}
para fundar um capítulo estudantil da
\href{https://seg.org}{Society of Exploration Geophysicists} (SEG) na UERJ.
Os capítulos da SEG proporcionam diversas oportunidades de desenvolvimento
profissional para os alunos através do financiamento de sua participação no
congresso anual nos E.U.A., campeonatos regionais e ciclos de palestras
internacionais.
O capítulo, denominado ``State University of Rio De Janeiro Geophysical
Society'' foi fundado oficialmente em Janeiro de 2015 e ainda encontra-se em
operação\footnote{Segundo \url{https://seg.org/Education/Student/Student-Chapters/Student-Chapter-Details/student-chapter-listing-details/scID/000000440245} (acessado em 03/01/2023)}.

Após terminar meu doutorado em 2016, comecei a cogitar pedir um afastamento de
um ano para fazer um pós-doutorado fora do país.
Isso foi motivado em partes pelo meu cansaço após dois intensos anos
trabalhando em período integral enquanto terminava o Doutorado, mas também em
parte porque minha parceira (e atual esposa)
\href{https://www.acarolcolombo.com/}{Ana Caroline Colombo} iria passar um ano
na \href{https://www.stonybrook.edu/}{Stony Brook University} nos Estados
Unidos como parte de seu doutorado.
A oportunidade de continuarmos no mesmo país surgiu na forma do cargo de
Pesquisador Visitante na
\href{https://www.hawaii.edu/}{aa} para trabalhar com o
\href{https://www.generic-mapping-tools.org/}{Generic Mapping Tools} (GMT),
um dos projetos de software livre de maior impacto na geofísica.
Essa oportunidade era muito boa para ser passada e então pedi meu afastamento
da UERJ por um ano a partir de Fevereiro de 2017.

Uma escolha muito mais desafiadora se apresentou em Janeiro de 2018 quando meu
afastamento chegaria ao fim.
Meu envolvimento no GMT estava sendo proveitoso e havia financiamento para me
manter no cargo por mais um ano e meio, com a possibilidade de conseguirmos
mais recursos no futuro.
Ao mesmo tempo, as condições financeiras e sociais no Brasil continuaram a
piorar, principalmente no Rio de Janeiro.
A UERJ se encontrava em grave situação financeira, causando o atraso no
pagamento dos servidores.
A escolha entre a certeza do meu cargo na UERJ e a incerteza de uma posição
temporária nos Estados Unidos não foi fácil.
Por fim, decidi que a melhor escolha para mim e para minha família naquela fase
da nossa vida seria tentar a sorte no exterior e pedir exoneração do cargo da
UERJ.

Minha experiência na UERJ foi positiva e muito educativa.
Avancei minhas linhas de pesquisa e fiz amizades com outros professores e
servidores da Faculdade de Geologia.
Também foi na UERJ que eu tive confirmação de que é na interação com os alunos,
tanto no papel de professor quanto de mentor, onde encontro a maior satisfação
profissional.
Meus esforços foram reconhecidos pelos alunos pois tive a honra de ser
escolhido como paraninfo da turma de formandos da Geologia em 2016 (ano de
ingresso 2012), que foi a primeira turma a qual dei aulas de geofísica.


\section{\UFRN}
\label{sec_hawaii}

\begin{subsummarybox}[frametitle=\faUniversity{}\quad Vínculo institucional]
  \begin{fa-ul}
    \faUser & Pesquisador Visitante \\
    \faMapMarker & Department of Earth Sciences -- School of Ocean and Earth Science and Technology\\
    \faCalendar & Fevereiro 2017 -- Agosto 2019
  \end{fa-ul}
\end{subsummarybox}

Comecei a contemplar a possibilidade de fazer um pós-doutorado no exterior
após defender minha tese de doutorado em meados de 2016.
Na busca por oportunidades de financiamento, me inscrevi em todas as listas de
email e classificados que pude encontrar\footnote{Até escrevi um artigo no
meu blog com uma lista desses recursos:
\url{https://www.leouieda.com/blog/job-sites.html}}.
Foi assim que me deparei com um email do Professor
\href{https://www.soest.hawaii.edu/pwessel/}{Paul Wessel} divulgando uma
posição para desenvolver uma ponte entre o software
\href{https://www.generic-mapping-tools.org/}{Generic Mapping Tools} (GMT)
e a linguagem de programação Python.
Tanto o Paul quanto o GMT são mundialmente famosos e o meu perfil se encaixava
perfeitamente na descrição das qualificações necessárias para a vaga.
Após uma entrevista por vídeo conferência com o Paul e os outros
desenvolvedores do GMT, fui informado de que havia sido selecionado para a
vaga.
Em Fevereiro de 2017 me mudei do Rio de Janeiro para Honolulu, E.U.A., para
começar essa nova etapa.

Conhecer e trabalhar com o Paul foi o destaque da minha estadia na
(UH).
Aprendi muito com ele, não somente sobre desenvolvimento de software mas sobre
como o sistema acadêmico americano funciona, como escrever projetos para
agências de fomento, como ser um líder que eleva as pessoas ao meu redor e como
ser humilde e reconhecer todos os fatores externos que possibilitaram meu
sucesso.
O jeito descontraído, bem humorado e energético do Paul é contagiante.
Sua paixão e brilhantes contribuições para a ciência são fruto de uma vida
fazendo exatamente o que mais gosta.

Minha experiência na UH foi diversa, incluindo participações em congressos e
até uma experiência de três dias no navio científico
\href{https://www.soest.hawaii.edu/soestwp/tech/watercraft/kilo-moana/}{R/V Kilo Moana}.
Criei uma rede de colaborares nos E.U.A. através do Paul, principalmente com
o grupo do Professor \href{https://topex.ucsd.edu/sandwell/}{David Sandwell}
do Scripps Institution of Oceanography.
Esse grupo desenvolve o software \href{https://github.com/gmtsar}{GMTSAR} para
processamento de dados de Synthetic Aperture Radar (SAR) e a geração de
interferogramas com a técnica InSAR.
Íamos ao Scripps anualmente para trabalhar com o grupo no software e ajudar a
ministrar o curso de GMT e GMTSAR que era promovido pela organização
\href{https://www.unavco.org/}{UNAVCO} (seção~\ref{sec_workshops}).
Minhas contribuições para o GMT serão discutidas mais adiante na
seção~\ref{sec_gmt}.

O tempo que passei em Honolulu foi inesquecível.
Porém, quando comecei a avaliar as opções para permanecer a longo prazo na UH
ou outra instituição do país, percebi que a carreira acadêmica nos E.U.A. era
excessivamente estressante e incerta.
Como eu e minha esposa sentíamos que ainda não estávamos prontos para retornar
ao Brasil, retomei minha busca por oportunidades de emprego no exterior que
possibilitassem um balanço melhor entre a vida pessoal e profissional.
Foi assim que encontrei um anúncio para uma vaga na área de geofísica na
University of Liverpool no Reino Unido.


\section{University of Liverpool}
\label{sec_liverpool}

\begin{subsummarybox}[frametitle=\faUniversity{}\quad Vínculo institucional]
  \begin{fa-ul}
    \faUser & Lecturer (\textit{equivalente a Professor Doutor})\\
    \faMapMarker & Department of Earth, Ocean and Ecological Sciences -- School of Environmental Sciences \\
    \faCalendar & Agosto 2019 -- Presente
  \end{fa-ul}
\end{subsummarybox}
\begin{subsummarybox}[frametitle=\faList{}\quad Atividades institucionais]
  \begin{datelist}
    2020--2022 & Comissão para avaliação do website do departamento\\
    2020--atual & Early Career Academic (ECA) Representative -- Earth Sciences\\
    2022--atual & Coordenador de curso: Bacharelado em Geofísica e Mestrado em Geologia e Geofísica
  \end{datelist}
\end{subsummarybox}

Com meu financiamento para me manter nos E.U.A. chegando ao fim e um desejo de
continuar no exterior por mais tempo, retomei minha busca por novas
oportunidades de pós-doutorado ou uma posição permanente.
No final de 2018 encontrei a chamada para uma vaga na University of Liverpool
de Lecturer (que no Reino Unido é equivalente a Professor Doutor) na área de
geofísica.
Descobri que o curso de geofísica de Liverpool possui uma longa tradição e
que diversos membros do departamento possuem ligações com o Brasil na área de
oceanografia geológica, geomagnetismo e paleomagnetismo.
Por conta desses fatores positivos, apliquei para a vaga e fui chamado para
uma entrevista no início de 2019.
Durante minha primeira viagem a Liverpool, pude confirmar que o departamento
era acolhedor e agradável de se trabalhar.
Mesmo com o \textit{jet lag} severo por conta da diferença de 10 horas entre
Honolulu e Liverpool, fui bem sucedido no processo seletivo e dei início ao meu
cargo de Lecturer em Agosto de 2019.

Ao chegar em Liverpool, fundei o grupo de pesquisa
\href{https://www.compgeolab.org/}{Computer-Oriented Geoscience Lab} (CompGeoLab)
com meu então aluno de doutorado
\SantiagoLink{} e comecei a buscar outros
alunos para se juntarem ao grupo (mais informações sobre orientações na
seção~\ref{sec_orientacao}).
Ministrei um total de sete disciplinas de graduação, incluindo trabalho de
campo de geofísica, programação em Python, sensoriamento remoto e geodinâmica
(seção~\ref{sec_ensino_grad}).
Assumi o cargo de representante de acadêmicos em início de carreira
(\textit{Early Career Academic Representative}), ou seja, servidores no nível de
Lecturer.
Minha responsabilidades incluem a organização eventos para desenvolvimento
profissional, eventos sociais entre departamentos, mentoria de novos Lecturers
e representação da categoria em comissões administrativas da universidade.
Entre 2020 e 2022, participei de uma comissão interna do departamento para
avaliar, organizar e atualizar nossa página na internet\footnote{Disponível em
\url{https://www.liverpool.ac.uk/earth-ocean-and-ecological-sciences/}}.
Em 2022, assumi o cargo de coordenador dos cursos de Bacharelado em Geofísica
e Mestrado em Geologia e Geofísica.

Como coordenador, sou responsável por recrutar alunos, alocar professores para
as disciplinas, revisar a estrutura do curso, revisar outros cursos da School
of Environmental Sciences, organizar atividades para os calouros, dar apoio aos
alunos e lidar com casos administrativos como transferências de curso,
trancamento, etc.
Atualmente, eu e o coordenador dos cursos de geologia estamos reformulando a
estrutura dos cursos para modernizá-los e possibilitar mais integração entre as
áreas.
Além disso, estamos criando uma nova especialização em geofísica para o
Bacharelado em Física.
O que mais senti falta durante os dois anos e meio que passei em Honolulu era
o contato direto com os alunos.
Estar novamente em uma posição que me permite ensinar e atuar como mentor é
muito gratificante.
Por isso, eu almejava assumir a posição de coordenador do curso em algum ponto
para ter uma visão mais geral de como os cursos são manejados pela
universidade.
Não esperava que a oportunidade viesse tão cedo (somente por conta de problemas
de saúde do coordenador anterior) mas fiquei contente em assumir a
responsabilidade e poder ter um impacto positivo no curso.


\section{Atuação na Comunidade Científica}
\label{sec_comunidade}

\begin{subsummarybox}[frametitle=\faList{}\quad Resumo das atividades]
  \begin{datelist}
    2019--2022 & Topic Editor -- \href{https://joss.theoj.org/}{Journal of Open Source Software} (ISSN 2475-9066) \\
    2019--2022 & Advisory Council Member -- \href{https://eartharxiv.org/}{EarthArXiv} \\
    2020--atual & Fellow -- \href{https://software.ac.uk}{Software Sustainability Institute} \\
    2022--atual & Board Member -- \href{https://softwareunderground.org}{Software Underground} \\
    2022--atual & Advisory Committee Member -- \href{https://www.pyopensci.org/}{pyOpenSci}
  \end{datelist}
\end{subsummarybox}

Além dos vínculos institucionais acima descritos, tenho uma participação
extensa na comunidade científica, principalmente na interseção entre
geociências, infraestrutura digital da ciência (e.g., software livre
científico) e ciência aberta.
Fui membro da banca de trabalhos de conclusão de Doutorado e Mestrado da
Christian-Albrechts-Universität zu Kiel (aluno Peter Haas),
University of Liverpool (aluna Yael Annemiek Engbers)
.
Atuei como revisor dos periódicos\footnote{Mais informações em \url{https://www.webofscience.com/wos/author/rid/\ResearcherID}}:
Geophysical Journal International,
Geophysics,
Journal of Geodesy,
Pure and Applied Geophysics,
Journal of Applied Geophysics,
Geophysical Prospecting,
Central European Journal of Geosciences,
Computers and Geosciences
e
Journal of Open Source Software.
Organizei sessões para os congressos internacionais AGU Fall Meeting de 2018 e
2019 e EGU General Assembly de 2021.

\subsection{Journal of Open Source Software}

Em 2019 fui convidado a me juntar ao corpo editorial do
\href{https://joss.theoj.org/}{Journal of Open Source Software}
(JOSS)\footnote{Mais informações em \url{https://joss.theoj.org/about\#editors\_emeritus}}
como Topic Editor na área de geociências.
O JOSS é um periódico que é operado no modelo
\href{https://en.wikipedia.org/wiki/Diamond_open_access}{\textit{diamond open acess}}
onde a publicação é gratuita, os autores retem seus direitos autorais e os
artigos são disponibilizados gratuitamente com uma licença
\href{https://creativecommons.org/licenses/by/4.0/}{Creative Commons Atribution} (CC-BY).
Seu objetivo é fornecer crédito, através de publicações revisadas por pares,
aos cientistas que se dedicam à criação de ferramentas de software livre para
o benefício da comunidade científica.
Necessitei me afastar dessa posição na metade de 2022 quando assumi o cargo
de coordenador do curso de graduação para dar conta da carga horária
administrativa mais elevada.
Espero poder retornar ao JOSS no futuro próximo.

\subsection{Software Underground}
\label{sec_swung}

O \href{https://softwareunderground.org/}{Software Underground} teve seu início
em 2014 como uma lista de
emails\footnote{Ainda disponível em \url{https://groups.google.com/g/softwareunderground}}
para pessoas interessadas em geociências e programação.
Em seguida, a comunidade migrou para a plataforma
\href{https://softwareunderground.org/slack}{Slack} onde cresceu rapidamente,
atualmente contando com mais de 4000 membros.
Em 2020, o Software Underground se tornou uma sociedade profissional sem fins
lucrativos incorporada no Canadá.
Além da plataforma Slack, a nova sociedade organiza eventos online e
presenciais e dá apoio aos projetos de software livre desenvolvidos pela
comunidade (como o Fatiando a Terra).

Estou envolvido no Software Underground desde o início.
Ter essa comunidade online ativa foi ainda mais importante durante o isolamento
forçado por conta da pandemia de COVID em 2020 e 2021.
Em 2022 me juntei à diretoria da sociedade como
\textit{Board Member}\footnote{Mais informações em \url{https://softwareunderground.org/board}}.
Meus maiores objetivos como parte da diretoria são estabelecer um mecanismo de
apoio financeiro pra projetos de software livre em geociências e progredir com
a iniciativa de publicações científicas regidas pela sociedade, que está sendo
feito em parceria com meu amigo \href{https://row1.ca/}{Rowan Cockett}, outro
membro da diretoria e fundador da plataforma \href{https://curvenote.com/}{Curvenote}.

\subsection{Software Sustainability Institute}
\label{sec_ssi}

Em 2020, fui premiado com um \textit{Fellowship} do
\href{https://software.ac.uk/}{Software Sustainability Institute} (SSI) que
inclui a afiliação não remunerada ao instituto e acesso a financiamento para
organizar eventos e atividades relacionadas à missão de aprimorar a criação e
manutenção de software para pesquisa.
Como \textit{Fellow}, eu tenho acesso à rede de contatos do instituto e
participação nos eventos e cursos organizados por eles.
Utilizei meu financiamento para organizar um encontro de geocientistas com
interesse na ciência aberta chamado
\href{https://hackmd.io/@leouieda/uk-geo-code-meetup}{Geo+Code}
(figura~\ref{fig_geocode}),
onde demos início ao desenvolvimento de recursos educacionais abertos para
geofísica aplicada (seção~\ref{sec_openedu}).
O evento contou com a participação de 15 pesquisadores, professores e
profissionais da indústria de nove instituições diferentes do Reino Unido e
Irlanda.
Já estamos planejando uma segunda e terceira iteração do evento para
continuarmos o trabalho e produzir um livro digital aberto de geofísica
aplicada com ênfase no uso de computação para o aprendizado.

\subsection{EarthArXiv}

O \href{https://eartharxiv.org/}{EarthArXiv} é um repositório de
\href{https://en.wikipedia.org/wiki/Preprint}{preprints} criado em 2017 e
mantido pela comunidade geocientífica.
Entre 2019 e 2022, servi como membro do
Advisory Council\footnote{Mais informações em \url{https://eartharxiv.github.io/AdvisoryCouncil.html}},
auxiliando na migração do repositório para uma nova plataforma hospedada na
\href{https://cdlib.org/}{California Digital Library} e na avaliação de
submissões antes de serem publicadas.

\subsection{pyOpenSci}

A organização \href{https://www.pyopensci.org/}{pyOpenSci} foi fundada em
2019 pela Dra. \href{https://www.leahwasser.com}{Leah Wasser}.
Baseada no modelo do \href{https://ropensci.org/}{rOpenSci}, a organização tem
como objetivo ajudar cientistas a desenvolverem software livre de qualidade na
linguagem Python.
Conheci a Leah durante um painel sobre dados abertos na AGU Fall Meeting de
2018 e me envolvi nas etapas iniciais do estabelecimento do pyOpenSci através
das sessões de mesa redonda que organizei na AGU Fall Meeting de 2018 e 2019.
Em 2022, me juntei oficialmente ao projeto como
\textit{Advisory Committee Member}\footnote{Mais informações em
\url{https://www.pyopensci.org/our-community/\#pyopensci-working-advisory-committee}},
auxiliando na criação das normas para revisão de submissões e no estabelecimento
de parcerias com outras organizações como o Journal of Open Source Software e
o Software Underground.


%==============================================================================
\chapter{Ciência Aberta}
\label{cap_cienciaaberta}

\begin{figure}[h]
  \HeroFigPad
  \begin{center}
    \includegraphics[width=\textwidth]{images/geopluscode.jpg}
  \end{center}
  \caption{
    Foto do evento \textit{Geo+Code UK} que organizei com meu financiamento
    do \href{https://software.ac.uk/}{Software Sustainability Institute} em
    Novembro de 2022. Durante o evento, demos início à criação de um
    livro texto digital sobre geofísica aplicada que será desenvolvido
    conjuntamente por educadores de diversas instituições do Reino Unido e
    Irlanda.
  }
  \label{fig_geocode}
\end{figure}
\begin{summarybox}[frametitle=\faInfoCircle{}\quad Portfólio de produção em ciência aberta]
  \begin{fa-ul}
    \faUser & Página pessoal: \url{https://www.leouieda.com}
      \\
    \faUsers & Grupo de pesquisa: \url{https://www.compgeolab.org}
      \\
    \faGithub & GitHub: \url{https://github.com/leouieda}
      (código, material didático) \\
    \aiFigshare & figshare: \url{https://figshare.com/authors/Leonardo\_Uieda/97471}
      (dados, apresentações, material suplementar) \\
    \aiImpactstory & Impactstory: \url{https://impactstory.org/u/0000-0001-6123-9515}
      (análise contextual da produção aberta) \\
    \faYoutube & YouTube: \url{https://youtube.com/LeonardoUieda}
      (palestras, tutoriais, aulas)
  \end{fa-ul}
\end{summarybox}

Este capítulo relata minhas atividades relacionadas a ciência aberta:
desenvolvimento de software livre, dados abertos, reprodutibilidade e recursos
educacionais abertos.
Essas atividades estão intrinsecamente ligadas às minhas linhas de pesquisa
(capítulo~\ref{cap_pesquisa}) e atividades de ensino
(capítulo~\ref{cap_ensino}).
Porém, decidi dedicar um capítulo a elas pois as considero atividades
complementares e tão importantes quanto publicações e aulas dadas.

\section{Introdução}

Meu primeiro contato com o movimento de \href{https://www.fsf.org/}{software livre}
foi durante meu curso de graduação na Universidade de São Paulo (seção~\ref{sec_usp}),
onde utilizávamos computadores com o sistema GNU/Linux e o software
\href{https://en.wikipedia.org/wiki/Seismic_Unix}{Seismic Unix} nas nossas aulas.
Fui cativado pelo princípio de garantir a todos a liberdade para modificar e
experimentar com programas e a cultura de se desenvolver produtos para o bem
comum de maneira colaborativa e transparente.
Para mim, esses são os ideais que a ciência representa mas que na prática acabam
não sendo realizados por diversas razões, incluindo a elevada competitividade
e falta de incentivos que dominam a ciência no século XXI.

Desde a elaboração de meu primeiro artigo \citep{Uieda2012}, decidi que iria
sempre buscar atingir esses ideais de transparência e colaboração sem barreiras
em tudo o que faço, mesmo que o resultado disso fosse que meu currículo não
seria bom o suficiente para uma carreira acadêmica.
Felizmente, esse receio inicial não se realizou e percebo hoje as grandes
vantagens em termos de impacto, reputação e oportunidades que essa dedicação me
proporcionou.
Atualmente todos os meus artigos como primeiro autor, e diversos como coautor,
incluem todo o código necessário para reproduzir todos os resultados
apresentados.
Mais que isso, busco utilizar somente dados que estão disponíveis com
licenças abertas (e.g., \href{https://creativecommons.org/licenses/by/4.0/}{CC-BY})
e publicar em acesso aberto para garantir que qualquer pessoa interessada
possa reproduzir meus resultados.
Esses princípios estão descritos de forma mais extensa no manual de operações\footnote{Disponível
em \url{https://github.com/compgeolab/manual}}
do \href{https://www.compgeolab.org/}{Computer-Oriented Geoscience Lab},
um documento que criamos para informar novos colaboradores e membros
do grupo sobre nossas expectativas em relação à ciência aberta.
Essa abordagem se estende ao material didático que desenvolvo para minhas aulas
e quase todos os outros aspectos da minha atuação profissional, incluindo
figuras ilustrativas e apresentações em formato oral e pôster.
Todo esse material pode ser encontrado nas diversas plataformas listadas no
``Portfólio de produção em ciência aberta'' acima.

\vspace{0.5cm}
\begin{subsummarybox}[frametitle=\faInfoCircle{}\quad Apresentações sobre ciência aberta]
  \begin{paperlist}
    2022 &
      \Me.
      Getting started with Open Science,
      \emph{SPIN SPIN-ITN: Seismological Parameters and Instrumentation}.
      \GitHub{leouieda/2022-05-06-spin-open-science}.
      \\
    2021 &
      \Me, \Santiago.
      Python-based workflows for small-to-medium sized data: what works, what
      doesn't, and what can be improved,
      \emph{AGU Fall Meeting}. \GitHub{compgeolab/agu2021}.
      \\
    ~ &
      \Me.
      Academia e software livre: Desafios e oportunidades no Brasil e no exterior,
      \emph{National Observatory's SEG and EAGE Student Chapter},
      Rio de Janeiro, Brazil.
      \GitHub{leouieda/2021-07-22-on}.
      \YouTube{r2x-DN6laj8}.
      \\
    2020 &
      \Me.
      Geophysical research powered by open-source,
      \emph{Departamento de Geofísica, IAG, Universidade de São Paulo}.
      \GitHub{leouieda/2020-06-18-usp}.
      \YouTube{VqI8BX1Yg54}.
      \\
    ~ &
      \Me.
      Geophysical research powered by open-source,
      \emph{Christian Albrechts Universität zu Kiel},
      Kiel, Germany.
      \GitHub{leouieda/2020-07-01-kiel}.
      \\
    ~ &
      \Me.
      Geophysical research powered by open-source,
      \emph{Technische Universität Bergakademie Freiberg}.
      \GitHub{leouieda/2020-06-04-freiberg}.
      \\
    ~ &
      \Me.
      Geophysical research powered by open-source,
      \emph{Geographic Data Science Lab, University of Liverpool}.
      \GitHub{leouieda/liverpool-gdsl-2020}.
      \\
    2019 &
      \Me.
      Building the foundations for open-source geophysics,
      \emph{University of Liverpool}.
      \DOI{10.6084/m9.figshare.10255832}.
      \\
    2017 &
      \Me, \Paul.
      Nurturing reliable and robust open-source scientific software,
      \emph{AGU Fall Meeting}.
      \YouTube{0GO4ZZ5Ry6M}.
  \end{paperlist}
\end{subsummarybox}


\section{Software livre}
\label{sec_software}

O termo \textit{software livre} se refere a programas de computador que
respeitam as liberdades de seus usuários de acessar, reutilizar e modificar o
seu código fonte.
Essas liberdades são geralmente garantidas pelo uso de licenças aprovadas pela
\href{https://opensource.org/}{Open Source Initiative} (OSI).
Desde a graduação, estou envolvido na produção de software livre para uso na
ciência.
Sou o criador dos programas
\href{https://tesseroids.leouieda.com}{Tesseroids},
\href{https://www.fatiando.org}{Fatiando a Terra},
\href{https://www.pygmt.org}{PyGMT} e
\href{https://www.compgeolab.org/xlandsat}{xlandsat}.
Além disso, contribuo com o desenvolvimento de outros projetos de software
livre\footnote{Como pode ser observado pela minha atividade no GitHub:
\url{https://github.com/leouieda}}, principalmente na linguagem Python.

Todos esses projetos são utilizados na minha pesquisa
(capítulo~\ref{cap_pesquisa}) e ensino (capítulo~\ref{cap_ensino}).
Para promover a sinergia entre essas atividades e o desenvolvimento dos
softwares, adotamos a seguinte abordagem no
\href{https://www.compgeolab.org/}{CompGeoLab}:

\begin{itemize}
  \item Todas as inovações metodológicas resultantes da nossa pesquisa devem
    ser incluídas em algum software livre, seja um dos que desenvolvemos
    internamente ou projetos desenvolvidos pela comunidade científica.
  \item Código que é de caráter inovador poderá ser desenvolvido privadamente
    até o momento da publicação do artigo/tese/dissertação. Após a publicação,
    o código deve ser integrado a um projeto de software livre.
  \item Código desenvolvido ao longo da pesquisa que não é de caráter
    inovador (e.g., funções baseadas em trabalhos já publicados ou rotinas
    básicas) deve ser incluído em algum software livre imediatamente.
  \item O software desenvolvido pelo CompGeoLab deve ser distribuído com uma
    licença que facilite seu uso pelo setor privado (e.g.,
    \href{https://opensource.org/licenses/BSD-3-Clause}{BSD} ou
    \href{https://opensource.org/licenses/MIT}{MIT}).
\end{itemize}

Essas regras visam maximizar o impacto de nossa pesquisa, possibilitando o
usufruto de nossas inovações sem restrições para toda a comunidade científica e
o setor privado.
Até o momento, essa abordagem também se mostrou muito vantajosa para a minha
carreira e para as de meus colaboradores.
As publicações que são acompanhadas pelas ferramentas computacionais
\citep[e.g.,][]{Uieda2016,Uieda2017} costumam ser mais citadas que minhas
outras publicações\footnote{Segundo dados da plataforma Google Scholar
\url{https://scholar.google.com/citations?user=qfmPrUEAAAAJ} (acessado em
10/01/2023)}.

Apresento a seguir um resumo da minha produção relacionada a software livre.

\subsection{Tesseroids}
\label{sec_tesseroids}

\begin{figure}[h]
  \SoftwareFigPad
  \begin{center}
    \includegraphics[width=\textwidth]{images/tesseroids.jpg}
  \end{center}
  \caption{Logo do software Tesseroids. A maçã caindo da letra ``T''
  é uma alusão à lei da gravitação de Newton.}
\end{figure}
\begin{summarybox}[frametitle=\faInfoCircle{}\quad Informações sobre o projeto]
  \begin{fa-ul}
    \faLink & Página principal: \url{https://tesseroids.leouieda.com}
    \\
    \faGithub & Código: \url{https://github.com/leouieda/tesseroids}
    \\
    \faGavel & Licença: \href{https://github.com/leouieda/tesseroids/blob/master/LICENSE.txt}{BSD 3-clause}
    \\
    \aiGoogleScholarSquare & 146 citações no \href{https://scholar.google.com/citations?view\_op=view\_citation\&hl=en\&user=qfmPrUEAAAAJ\&citation\_for\_view=qfmPrUEAAAAJ:AXPGKjj\_ei8C}{Google Scholar}\footnotemark{} (acessado em 27/12/2022)
  \end{fa-ul}
\end{summarybox}
\footnotetext{Citações ao trabalho \citet{Uieda2016}.}
\begin{subsummarybox}[frametitle=\faFilePdf{}\quad Artigos publicados]
  \begin{paperlist}
    2016 &
      \Me, \Val, \Carla.
      Tesseroids: Forward modeling gravitational fields in spherical coordinates,
      \emph{Geophysics}, \DOI{10.1190/geo2015-0204.1}.
      \GitHub{pinga-lab/paper-tesseroids}.
  \end{paperlist}
\end{subsummarybox}

O Tesseroids foi meu primeiro projeto de software, o tema do meu
trabalho de conclusão de curso de graduação (seção~\ref{sec_ic_tesseroids})
e um dos capítulos da minha tese de doutorado (seção~\ref{sec_doutorado}).
A primeira versão do software foi feita na linguagem C na forma de programas
de linha de comando.
Cada programa era capaz de ler a geometria dos tesseroides e calcular o
potencial gravitacional ou uma de suas primeiras ou segundas derivadas
espaciais nos pontos especificados pelo usuário.
Como estava aprendendo a linguagem Python, resolvi reescrever o código ainda
durante minha iniciação científica para poder utilizar as diversas bibliotecas
disponíveis na linguagem e evitar o trabalho de compilar o código para
diferentes sistemas operacionais.
Porém, a versão do código em Python era consideravelmente mais lenta que a
versão original em C.
Isso era devido à minha limitação como programador, não às limitações da
linguagem e ferramentas disponíveis na época.
Em 2011, durante meu mestrado, fui convidado pela Professora Carla Braitenberg
para passar um mês na  reescrevendo o software na linguagem C como no
modelo que havia feito inicialmente.
Essa última versão do Tesseroids iniciada em Trieste possui diversas vantagens
sobre as anteriores:

\begin{enumerate}
  \item Execução mais rápida por conta do código optimizado em C.
  \item Documentação na forma de uma página na internet, que aprendi como gerar
    através do meu trabalho no Fatiando a Terra (seção~\ref{sec_fatiando}).
  \item Programas para calcular o feito de prismas retangulares retos em
    coordenadas esféricas, usados para avaliar os resultados obtidos com
    tesseroides.
  \item Programas para auxiliar na geração de modelos topográficos.
  \item Testes unitários para verificar o funcionamento correto do código de
    forma automática\footnote{Testes unitários disponíveis em \url{https://github.com/leouieda/tesseroids/tree/master/test}}.
  \item Distribuição de versões compiladas do código para as plataformas Linux
    e Windows em 32 e 64 bits.
  \item Utilização do serviço \href{https://travis-ci.org/github/leouieda/tesseroids}{TravisCI} de
    \href{https://pt.wikipedia.org/wiki/Integra%C3%A7%C3%A3o_cont%C3%ADnua}{integração contínua}
    para executar os testes unitários automaticamente cada vez que uma mudança
    é feita no código.
  \item Algoritmo de discretização adaptativa dos tesseroides para garantir
    acurácia melhor que 0.1\% dos resultados \citep{Uieda2016}.
\end{enumerate}

Essa última versão do Tesseroids foi descrita no artigo \citet{Uieda2016}, que
é um dos meus trabalhos mais citados\footnote{Segundo a plataforma Google
Scholar
\url{https://scholar.google.com/citations?view_op=view_citation&hl=en&user=qfmPrUEAAAAJ&citation_for_view=qfmPrUEAAAAJ:AXPGKjj_ei8C}
(acessado em 27/12/2022)}.
O desenvolvimento do Tesseroids foi interrompido em 2017 para que eu pudesse me
dedicar mais ao Fatiando a Terra e ao meu novo trabalho na
(seção~\ref{sec_hawaii}).
Porém, o método desenvolvido em \citet{Uieda2016} e aprimorado em
\citet{Soler2019} foi implementado no Fatiando a Terra.
A versão atual do código Python para a modelagem com tesseroides, desenvolvido
pelo meu ex-aluno de doutorado \SantiagoLink{} (seção~\ref{sec_orientacao})
para a biblioteca \href{https://www.fatiando.org/harmonica/}{Harmonica} (parte
do Fatiando a Terra), é mais rápida que a versão em C.


\subsection{Fatiando a Terra}
\label{sec_fatiando}

\begin{figure}[h]
  \SoftwareFigPad
  \begin{center}
    \includegraphics[width=\textwidth]{images/fatiando.jpg}
  \end{center}
  \caption{
    Logo do projeto Fatiando a Terra (meio da figura) e os logos dos softwares
    que atualmente fazem parte do projeto: Pooch, Verde, Harmonica e Boule (da
    esquerda para a direita).
}
\end{figure}
\begin{summarybox}[frametitle=\faInfoCircle{}\quad Informações sobre o projeto]
  \begin{fa-ul}
    \faLink & Página principal: \url{https://www.fatiando.org}
    \\
    \faGithub & Código: \url{https://github.com/fatiando}
    \\
    \faGavel & Licença: \href{https://opensource.org/licenses/BSD-3-Clause}{BSD 3-clause}
    \\
    \aiGoogleScholarSquare & 119 citações no \href{https://scholar.google.com/citations?user=qfmPrUEAAAAJ}{Google Scholar}\footnotemark{} (acessado em 27/12/2022)
  \end{fa-ul}
\end{summarybox}
\footnotetext{Total de citações aos trabalhos \citet{Uieda2013}, \citet{Uieda2018} e \citet{Uieda2020}.}
\begin{subsummarybox}[frametitle=\faFilePdf{}\quad Artigos publicados]
  \begin{paperlist}
    2020 &
      \Me, \Santiago, \Remi, \Hugo, \MattTurk, \Shapero, \Anderson, \Leeman.
      Pooch: A friend to fetch your data files.
      \emph{Journal of Open Source Software}.
      \DOI{10.21105/joss.01943}.
      \GitHub{fatiando/pooch}.
      \\
    2018 &
      \Me. Verde: Processing and gridding spatial data using Green's functions.
      \emph{Journal of Open Source Software}.
      \DOI{10.21105/joss.00957}.
      \GitHub{fatiando/verde}.
  \end{paperlist}
\end{subsummarybox}
\begin{subsummarybox}[frametitle=\faFile{}\quad Trabalhos completos em anais de eventos]
  \begin{paperlist}
    2013 &
      \Me, \Bi, \Val.
      Modeling the Earth with Fatiando a Terra,
      \emph{Proceedings of the 12th Python in Science Conference}.
      \DOI{10.25080/Majora-8b375195-010}.
      \GitHub{leouieda/scipy2013}.
  \end{paperlist}
\end{subsummarybox}
\begin{subsummarybox}[frametitle=\faComment{}\quad Outras apresentações]
  \begin{paperlist}
    2021 &
      \Me, \LLi, \Santiago, \Agustina.
      Design useful tools that do one thing well and work together: rediscovering
      the UNIX philosophy while building the Fatiando a Terra project,
      \emph{AGU Fall Meeting}.
      \GitHub{fatiando/agu2021}.
      \\
    ~ &
      \Me, \Santiago, \Agustina.
      Open-science for gravimetry: tools, challenges, and opportunities,
      \emph{GFZ Helmholtz Centre Potsdam}.
      \GitHub{leouieda/2021-06-22-gfz}.
      \YouTube{z-5dvWfB\_SM}.
      \\
    ~ &
      \Me, \Santiago, \Agustina.
      Fatiando a Terra: Open-source tools for geophysics,
      \emph{Geophysical Society of Houston}.
      \GitHub{fatiando/2021-gsh}.
      \\
    ~ &
      \Me, \Santiago, \Agustina, \LPerozzi, \MWieczorek.
      Harmonica and Boule: Modern Python tools for geophysical gravimetry,
      \emph{EGU General Assembly}.
      \DOI{10.5194/egusphere-egu21-8291}.
      \GitHub{fatiando/egu2021}.
      \\
    2015 &
      \Me.
      Fatiando a Terra: construindo uma base para ensino e pesquisa de geofísica,
      \emph{Universidade de São Paulo}.
      \DOI{10.6084/m9.figshare.1381870}
      \\
    2014 &
      \Me, \Bi, \Val.
      Using Fatiando a Terra to solve inverse problems in geophysics,
      \emph{Scipy}.
      \DOI{10.6084/m9.figshare.1089987}.
  \end{paperlist}
\end{subsummarybox}

Durante meu curso de graduação, eu e meus colegas
\href{https://www.pinga-lab.org/people/oliveira-jr.html}{Vanderlei C. Oliveira Jr.},
\href{https://www.linkedin.com/in/hbuenos/}{Henrique Bueno dos Santos},
\href{https://www.linkedin.com/in/andr%C3%A9-ferreira-lopes/}{André Lopes Ferreira} e
\href{https://www.linkedin.com/in/josecaparica/}{José Fernando Caparica Jr.}
começamos a planejar o desenvolvimento de um software livre capaz de modelar
todos os tipos de dados geofísicos.
Chamávamos esse projeto ambicioso de ``Fatiando a Terra'' pois nosso objetivo
era modelar a Terra inteira (fatiá-la em polígonos) utilizando todos os dados
disponíveis.
O software seria escrito na linguagem C++ e chegamos até a criar um diagrama
das componentes principais que iríamos implementar\footnote{Esse diagrama ainda
existe no histórico do repositório do GitHub: \url{https://github.com/fatiando/fatiando/blob/10c8ff7c17df53e3e0abd83f1ce8d2a3f6bc57aa/fluxo-simples.pdf}}.
Por razões óbvias, não alcançamos nosso objetivo até o final do nosso curso de
graduação.
Porém, o nome do projeto sobreviveu.
Em 30 de Abril de 2010, no início do
meu mestrado, transformei o Fatiando a Terra em uma biblioteca, chamada
\texttt{fatiando}, escrita na linguagem Python\footnote{O momento exato em que
essa mudança aconteceu está registrado no repositório do GitHub:
\url{https://github.com/fatiando/fatiando/commit/928515b0fcfdccecbc4f661ed2469390ef43ec1d}}.
Meu novo objetivo passou a ser agregar todo o código que estava desenvolvendo
para minha dissertação e para as disciplinas da pós-graduação.
Por conta disso, a biblioteca inclui funções para a modelagem direta e inversão
de diversos métodos geofísicos (e.g., métodos potenciais em 2D e 3D, perfilagem
sísmica vertical, condução de calor geotermal, entre outros).

Uma grande parte do desenvolvimento inicial, incluindo a criação da primeira
versão da página \url{https://www.fatiando.org}, ocorreu em preparo para o
curso ``Tópicos de inversão em geofísica'' que ministrei com o Vanderlei na XVI
Escola de Verão de Geofísica do IAG-USP em 2012 (seção~\ref{sec_workshops}).
O software continuou a crescer durante minha pós-graduação, contando com a
participação de outros 12 desenvolvedores\footnote{Mais informações em
\url{https://github.com/fatiando/fatiando/graphs/contributors}}.
Utilizei o Fatiando como parte integral das minhas aulas de geofísica na
UERJ e contratei o bolsista Victor Thadeu Xavier de Almeida para trabalhar no
desenvolvimento das funções para processamento sísmico (seção~\ref{sec_uerj}).

Em 2016, o aluno \SantiagoLink{} se juntou à equipe de desenvolvimento do
Fatiando como parte de seu projeto de doutorado (seção~\ref{sec_orientacao}).
O Santiago é um programador talentoso e rapidamente aprendeu como participar
do desenvolvimento do Fatiando, criar exemplos para a documentação e atuar como
mentor para novos programadores.
Simultaneamente, o grupo
\href{https://softwareunderground.org/}{Software Underground} estava se
formando (seção~\ref{sec_swung}) e nos conectando com os criadores dos projetos
de software livre
\href{https://simpeg.xyz/}{SimPEG} \citep{Cockett2015},
\href{https://www.pygimli.org}{pyGIMLi} \citep{Rucker2017} e
\href{https://www.gempy.org/}{GemPy} \citep{delaVarga2019},
todos escritos na linguagem Python para modelagem direta e inversão.
Através dessas interações e das conversas semanais que tinha com o Santiago,
percebemos que estava na hora de redefinir os objetivos do Fatiando para
nos alinharmos com esses outros projetos.
Nossa decisão\footnote{Resumida em um artigo publicado no meu blog:
\url{https://www.leouieda.com/blog/future-of-fatiando.html}} foi de
interromper o desenvolvimento da biblioteca \texttt{fatiando} e separar suas
funções em bibliotecas menores com escopos mais bem definidos.
As funções que não estavam sendo utilizadas ou que já existiam em outras
bibliotecas seriam abandonadas.
Essa também seria uma oportunidade para modernizar o nosso código e torná-lo
mais eficiente e fácil de usar.

As novas bibliotecas que são parte do projeto Fatiando a Terra são:

\begin{itemize}
  \item \href{https://www.fatiando.org/verde}{\textbf{Verde}}:
    A primeira biblioteca que foi desenvolvida para a nova fase do Fatiando. O
    Verde contém funções e classes para processar e interpolar dados
    distribuídos irregularmente.
  \item \href{https://www.fatiando.org/harmonica}{\textbf{Harmonica}}:
    Nossa biblioteca para processamento e modelagem de dados de métodos
    potenciais. O Harmonica é liderado pelo Santiago e inclui funções para
    modelagem direta, correção topográfica, processamento com fontes
    equivalentes (seção~\ref{sec_eql}) e filtros no domínio da frequência.
  \item \href{https://www.fatiando.org/boule}{\textbf{Boule}}:
    Biblioteca para o cálculo do campo de gravidade gerado por elipsóides de
    referência (i.e., a gravidade normal). As funções e classes do Boule eram
    originalmente parte do Harmonica. O Boule foi criado em colaboração com os
    desenvolvedores do \href{https://github.com/SHTOOLS/SHTOOLS}{SHTools}
    \citep{Wieczorek2018} para que pudéssemos utilizar suas funções
    independentemente do Harmonica. O cálculo da gravidade normal em qualquer
    ponto fora do elipsóide é feito através da solução analítica de
    \citet{Lakshmanan1991} e \citet{Li2001}. Logo, a correção de ar-livre não é
    necessária para o cálculo de distúrbios da gravidade.
  \item \href{https://www.fatiando.org/pooch}{\textbf{Pooch}}:
    Uma biblioteca para baixar dados da internet e armazená-lo localmente. O
    Pooch não é diretamente relacionado à geofísica e foi criado em colaboração
    com os desenvolvedores do \href{https://github.com/Unidata/MetPy}{MetPy}
    \citep{May2016}. Durante o congresso Scipy de 2018, notamos que diversas
    bibliotecas em Python, incluindo o Verde e o MetPy, possuíam códigos
    semelhantes para baixar dados.
    Por isso, criamos o Pooch para que todos pudéssemos utilizá-lo e eliminar
    o código repetido.
  \item \href{https://www.fatiando.org/ensaio}{\textbf{Ensaio}}:
    Biblioteca que utiliza o Pooch para baixar os dados abertos que utilizamos
    nos tutoriais e nas documentações dos outros softwares.
  \item \href{https://www.fatiando.org/choclo}{\textbf{Choclo}}:
    A mais recente adição ao Fatiando. O Choclo é desenvolvido e liderado pelo
    Santiago. Essa biblioteca implementa rotinas altamente otimizadas para
    modelagem direta em métodos potenciais. Assim como o Boule, o código
    presente no Choclo era inicialmente parte do Harmonica mas está sendo
    separado para que possa ser usado tanto no Harmonica como no SimPEG. Esse
    trabalho é parte do pós-doutorado que o Santiago está fazendo com a
    Professora \href{https://lindseyjh.ca/}{Lindsey Heagy} (uma das criadoras
    do SimPEG) na University of British Columbia, Canadá.
\end{itemize}

Nossa reestruturação foi acompanhada de um esforço para aumentar o engajamento
e a diversidade de voluntários no projeto.
Eu e o Santiago começamos a orientar e ensinar pessoas interessadas, buscar
ativamente contribuidores em nossas redes sociais e organizar reuniões semanais
para criar uma comunidade em torno do Fatiando.
Nossos esforços foram bem sucedidos e o Fatiando conta hoje em dia com a
participação regular de outras cinco pessoas.

O escopo bem definido de cada uma das bibliotecas (ao invés de ``modelar
toda a Terra'') também contribui para sua adoção pela comunidade científica.
Um exemplo claro de sucesso é o Pooch, que atualmente é utilizado por 88 outros
softwares\footnote{Segundo a página
\url{https://libraries.io/pypi/pooch/dependents} (acessada em 11/01/2023)}.
Como consequência, o Pooch agregou mais de 14 milhões de downloads\footnote{Segundo a página
\url{https://pepy.tech/project/Pooch} (acessada em 11/01/2023)} e 31 pessoas
participaram do seu desenvolvimento\footnote{Segundo a página
\url{https://github.com/fatiando/pooch/graphs/contributors} (acessada em
11/01/2023)}.
Sabemos que as outras bibliotecas também estão sendo utilizadas pela comunidade
pelas mais de 50 mil visualizações anuais das nossas páginas de documentação,
com visitantes originados de todos os continentes (exceto a Antártica)\footnote{Segundo a página
\url{https://plausible.io/fatiando.org} (acessada em 11/01/2023)}.



\subsection{Generic Mapping Tools}
\label{sec_gmt}

\begin{figure}[h]
  \SoftwareFigPad
  \begin{center}
    \includegraphics[width=\textwidth]{images/gmt.jpg}
  \end{center}
  \caption{Logo do Generic Mapping Tools (GMT), gerado pelo próprio GMT
    utilizando o comando
  \href{https://docs.generic-mapping-tools.org/latest/gmtlogo.html}{\texttt{gmt logo}}.}
\end{figure}
\begin{summarybox}[frametitle=\faInfoCircle{}\quad Informações sobre o projeto GMT]
  \begin{fa-ul}
    \faLink & Página principal: \url{https://www.generic-mapping-tools.org}
    \\
    \faGithub & Código: \url{https://github.com/GenericMappingTools}
    \\
    \faGavel & Licença: \href{https://opensource.org/licenses/LGPL-3.0}{GNU LGPL}
    \\
    \aiGoogleScholarSquare & 934 citações no \href{https://scholar.google.com/citations?view\_op=view\_citation\&hl=en\&user=qfmPrUEAAAAJ\&citation\_for\_view=qfmPrUEAAAAJ:hkOj\_22Ku90C}{Google Scholar}\footnotemark{} (acessado em 27/12/2022)
  \end{fa-ul}
\end{summarybox}
\footnotetext{Citações ao trabalho \citet{Wessel2019}.}
\begin{summarybox}[frametitle=\faInfoCircle{}\quad Informações sobre o projeto PyGMT]
  \begin{fa-ul}
    \faLink & Página principal: \url{https://www.pygmt.org}
    \\
    \faGithub & Código: \url{https://github.com/GenericMappingTools/pygmt}
    \\
    \faGavel & Licença: \href{https://github.com/GenericMappingTools/pygmt/blob/main/LICENSE.txt}{BSD 3-clause}
    \\
    \aiGoogleScholarSquare & 43 citações no \href{https://scholar.google.com/citations?view\_op=view\_citation\&hl=en\&user=qfmPrUEAAAAJ\&citation\_for\_view=qfmPrUEAAAAJ:-\_dYPAW6P2MC}{Google Scholar}\footnotemark{} (acessado em 27/12/2022)
  \end{fa-ul}
\end{summarybox}
\footnotetext{Citações ao trabalho \citet{Uieda2022}.}
\begin{subsummarybox}[frametitle=\faFilePdf{}\quad Artigos publicados]
  \begin{paperlist}
    2019 &
      \Paul, \Joaquim, \Me, \Remko, \Florian, \Walter, \Dongdong.
      The Generic Mapping Tools, Version 6.
      \emph{Geochemistry, Geophysics, Geosystems}.
      \DOI{10.1029/2019GC008515}.
  \end{paperlist}
\end{subsummarybox}
\begin{subsummarybox}[frametitle=\faInfoCircle{}\quad Apresentações]
  \begin{paperlist}
    2019 &
      \Me, \Paul.
      PyGMT: Accessing the Generic Mapping Tools from Python,
      \emph{AGU Fall Meeting}.
      \DOI{10.6084/m9.figshare.11320280}
      \\
    2018 &
      \Me, \Paul.
      Building an object-oriented Python interface for the Generic Mapping Tools,
      \emph{Scipy}.
      \DOI{10.6084/m9.figshare.6814052}
      \YouTube{6wMtfZXfTRM}
      \\
    ~ &
      \Me, \Paul.
      Integrating the Generic Mapping Tools with the Scientific Python Ecosystem,
      \emph{AOGS $15^{th}$ Annual Meeting}.
      \DOI{10.6084/m9.figshare.6399944}
      \\
    2017 &
      \Me, \Paul.
      A modern Python interface for the Generic Mapping Tools,
      \emph{AGU Fall Meeting}.
      \DOI{10.6084/m9.figshare.5662411}
      \\
    ~  &
      \Me, \Paul.
      Bringing the Generic Mapping Tools to Python,
      \emph{Scipy}.
      \DOI{10.6084/m9.figshare.7635833}
      \YouTube{93M4How7R24}
      \end{paperlist}
\end{subsummarybox}

O \href{https://www.generic-mapping-tools.org}{Generic Mapping Tools} (GMT)
é um dos softwares livres mais utilizados na geofísica.
Ele foi criado na década de 1980 por dois alunos de doutorado do Lamont-Doherty
Earth Observatory, E.U.A.,
\href{https://www.star.nesdis.noaa.gov/star/Smith_WHF.php}{Walter H. F. Smith}
e \href{https://www.soest.hawaii.edu/pwessel/}{Paul Wessel}.
O GMT é um programa de linha de comando (i.e., sem interface gráfica) escrito
na linguagem C.
O programa oferece dezenas de comandos para processar e visualizar dados
geofísicos.
Meu envolvimento com o projeto começou em 2017 quando fui contratado pelo Paul
para criar uma ponte entre o GMT e a linguagem Python (seção~\ref{sec_hawaii}).
O resultado desse meu trabalho foi a criação do software
\href{https://www.pygmt.org}{PyGMT}.

Meu primeiro desafio para tornar o GMT acessível da linguagem Python foi
realizar a compilação do software de maneira compatível com a bibliotecas
científicas do Python (numpy, scipy, etc.).
Isso foi possível graças à plataforma
\href{https://conda-forge.org/}{Conda-Forge}, que automatiza a compilação e
distribuição de software, e a ajuda imensa do
\href{https://github.com/ocefpaf}{Filipe Fernandes}, um dos líderes do
Conda-Forge e um brilhante oceanógrafo e programador brasileiro.
Gastei meus primeiros seis meses de trabalho para superar esse desafio.
Em seguida, dei início ao desenvolvimento do código em Python que seria capaz
de executar rotinas da biblioteca em C do GMT.
Para isso, utilizei uma tecnologia chamada \textit{C foreign function
interface} (C FFI) que permite a interação de bibliotecas em C diretamente com
outras linguagens.
Construir essa interface foi um trabalho árduo mas que formou o núcleo que o
PyGMT usa para se comunicar com o GMT.
Meu investimento valeu a pena pois o PyGMT depende desse núcleo até hoje com
poucas modificações nos últimos anos.
Inicialmente, optei por concentrar meus esforços nessa parte do código que é
complexa e requer conhecimento profundo do GMT, bibliotecas em C e funções
avançadas em Python.
Também investi muito do meu tempo criando documentação, incluindo um guia para
desenvolvedores, e tornando o processo de desenvolvimento do PyGMT
automatizado e simples.
Tomei essas decisões para facilitar ao máximo o envolvimento futuro de novos
desenvolvedores voluntários no projeto, quebrando algumas das barreiras que
normalmente impossibilitam a participação de programadores novatos.
Como resultado disso, o projeto agora conta com a participação de oito outros
desenvolvedores e mais de 40 contribuidores esporádicos\footnote{Uma lista dos
principais desenvolvedores está disponível em \url{https://www.pygmt.org/latest/team.html}
e uma lista dos contribuidores está disponível em \url{https://github.com/GenericMappingTools/pygmt/graphs/contributors}}.
Sou muito grato à dedicação do
\href{https://github.com/seisman}{Dongdong Tian},
\href{https://weiji14.github.io/}{Wei Ji Leong} e
\href{https://github.com/maxrjones}{Max Jones}
que assumiram posições de liderança no projeto quando meu envolvimento diminuiu
em 2019 ao me mudar para Liverpool.
Tenho muito orgulho de dizer que o PyGMT continua crescendo e evoluindo sem
minha participação direta no seu desenvolvimento.
Considero o estabelecimento da comunidade que se formou em torno do PyGMT a
minha maior conquista relacionada a software livre.

Além do meu trabalho no PyGMT, também fui responsável pela transição do
desenvolvimento do GMT para a plataforma GitHub (motivada pela falha do
servidor utilizado anteriormente durante um encontro de desenvolvedores), a
criação da atual página do projeto \url{https://www.generic-mapping-tools.org}
e a modernização e automatização da compilação da página de documentação do
GMT\footnote{Disponível em \url{https://docs.generic-mapping-tools.org}}.
Também fui responsável pela escrita e coordenação, junto com o Paul,
de dois projetos financiados pela \href{https://www.nsf.gov/}{National Science
Foundation} (NSF)\footnote{Disponíveis em
\url{https://www.nsf.gov/awardsearch/showAward?AWD_ID=1829371} e
\url{https://www.nsf.gov/awardsearch/showAward?AWD_ID=1948602}}.
O objetivo principal desses projetos era estabelecer um plano para o futuro
do GMT sem tanto envolvimento direto do Paul, que estava perto da aposentadoria.
Com esse financiamento, pudemos atualizar a documentação do GMT e torná-la mais
acessível, promover eventos para recrutar desenvolvedores e treinar usuários
(seção~\ref{sec_workshops}), organizar encontros dos desenvolvedores e
contratar o Doutor \href{https://github.com/maxrjones}{Max Jones} para
trabalhar no GMT e PyGMT.
Como resultado, o time de desenvolvedores do GMT conta agora com mais três
pessoas, envolvidas não só na programação mas também na organização de eventos
e divulgação do projeto.


\subsection{xlandsat}

\begin{summarybox}[frametitle=\faInfoCircle{}\quad Informações sobre o projeto]
  \begin{fa-ul}
    \faLink & Página principal: \url{https://compgeolab.org/xlandsat}
    \\
    \faGithub & Código: \url{https://github.com/compgeolab/xlandsat}
    \\
    \faGavel & Licença: \href{https://github.com/compgeolab/xlandsat/blob/main/LICENSE.txt}{MIT}
  \end{fa-ul}
\end{summarybox}

Este é o mais recente software que foi criado no âmbito do
\href{https://www.compgeolab.org}{CompGeoLab}, tendo sido iniciado em Dezembro
de 2022.
O xlandsat é uma biblioteca feita para facilitar o processamento e visualização
de dados de sensoriamento remoto dos satélites
\href{https://en.wikipedia.org/wiki/Landsat_program}{Landsat 8 e 9} da
NASA e da USGS.
A biblioteca é capaz de ler os dados no formato do repositório
\href{https://earthexplorer.usgs.gov/}{EarthExplorer} e organizá-los em
estruturas de dados da biblioteca \href{https://xarray.dev/}{xarray}
\citep{Hoyer2017}, uma das ferramentas mais utilizadas para processamento de
dados geocientíficos.

A criação do xlandsat foi motivada pelas minhas aulas de sensoriamento
remoto na disciplina ``ENVS258 Environmental Geophysics''
(seção~\ref{sec_ensino_grad}).
Antes de ministrar essa disciplina, meu conhecimento de processamento de
imagens de satélite era mínimo.
Ao longo do preparo de meu material didático, aprendi muito sobre o assunto.
Criei uma coleção de funções escritas na linguagem Python para auxiliar meus
alunos a processarem os dados baixados do
\href{https://earthexplorer.usgs.gov/}{EarthExplorer} em seus relatórios.
Em 2022, decidi que estava na hora de organizar esse código em uma biblioteca
que poderia ser utilizada por meus alunos na disciplina em 2023.
Além disso, o sensoriamento remoto despertou meu interesse acadêmico e
eu necessitava de uma maneira fácil de explorar as possíveis aplicações e
limitações desses dados.
Por exemplo, utilizei o xlandsat para criar uma visualização da erupção de
Dezembro de 2022 do vulcão Mauna Loa, Havaí (figura~\ref{fig_maunaloa}).

\begin{figure}[tb]
  \begin{center}
    \includegraphics[width=\textwidth]{images/mauna-loa-landsat-2022-12-02.jpg}
  \end{center}
  \caption{
    Imagem da erupção de Dezembro de 2022 do vulcão Mauna Loa, Havaí, composta
    pelas bandas infravermelhas do satélite Landsat 9.
    O vulcão está cercado por nuvens (branco e azul claro). O fluxo de lava
    atual está na direção Sul/Norte em vermelho e verde. A cratera principal
    pode ser vista no centro da imagem em vermelho escuro.
    As manchas em marrom e preto são fluxos de lava de erupções anteriores.
    Também está visível a cratera Hale Ma'uma'u do vulcão Kīlauea no canto
    inferior direito da imagem.
    O código Python para reproduzir essa imagem está disponível em um artigo na
    minha página pessoal \url{https://www.leouieda.com/blog/mauna-loa.html} e
    também no repositório do GitHub
    \url{https://github.com/compgeolab/mauna-loa-landsat-2022}.
    Fonte da imagem: \citet[][CC0]{Uieda2022maunaloa}.
  }
  \label{fig_maunaloa}
\end{figure}


\section{Recursos educacionais abertos}
\label{sec_openedu}

\begin{subsummarybox}[frametitle=\faFilePdf{}\quad Artigos publicados em revistas]
  \begin{paperlist}
    2017 &
      \Me.
      Step-by-step NMO correction,
      \emph{The Leading Edge},
      \DOI{10.1190/tle36020179.1}.
      \GitHub{pinga-lab/nmo-tutorial}.
      \\
    2014 &
      \Me, \Bi, \Val.
      Geophysical tutorial: Euler deconvolution of potential-field data,
      \emph{The Leading Edge},
      \DOI{10.1190/tle33040448.1}.
      \GitHub{pinga-lab/paper-tle-euler-tutorial}.
  \end{paperlist}
\end{subsummarybox}
\begin{subsummarybox}[frametitle=\faBook{}\quad Recursos computacionais]
  \begin{paperlist}
    2021 &
      \Me. A quick introduction to machine learning.
      \GitHub{leouieda/ml-intro}.
      \\
    2020 &
      \Me. Introduction to lithosphere dynamics.
      \GitHub{leouieda/lithosphere}.
      \\
    2020 &
      \Me. Introduction to remote sensing.
      \GitHub{leouieda/remote-sensing}.
      \\
    2015 &
      \Me. Matemática Especial 1: Introdução à computação e métodos numéricos.
      \GitHub{mat-esp/about}.
      \\
    2015 &
      \Me. Geofísica 2: Sismologia e métodos eletromagnéticos.
      \GitHub{leouieda/geofisica2}.
      \\
    2015 &
      \Me. Geofísica 1: Gravimetria e magnetometria.
      \GitHub{leouieda/geofisica1}.
  \end{paperlist}
\end{subsummarybox}
\begin{subsummarybox}[frametitle=\faYoutube{}\quad Apostilas]
  \begin{paperlist}
    2012 &
      \Bi, \Me. Tópicos de inversão em geofísica.
      \DOI{10.6084/m9.figshare.1192984}.
      \GitHub{pinga-lab/inverse-problems}.
  \end{paperlist}
\end{subsummarybox}
\begin{subsummarybox}[frametitle=\faYoutube{}\quad Vídeos]
  \begin{paperlist}
    2022 & A geophysical tour of mid-ocean ridges. \YouTube{NzJmRlJCNbQ}
      \\
    2022 & Anatomy of a PyGMT figure. \YouTube{96\_reU\_yh5I}
      \\
    2021 & Downloading Landsat 8 images from USGS EarthExplorer. \YouTube{Wn\_G4fvitV8}
      \\
    2021 & Searching on Google for openly licensed images. \YouTube{ISu51NB5Z28}
      \\
    2020 & From scattered data to gridded products using Verde. \YouTube{-xZdNdvzm3E}
  \end{paperlist}
\end{subsummarybox}

O termo ``recursos educacionais abertos'' (REA ou \textit{open educational
resources} em inglês) foi estabelecido pela UNESCO\footnote{Mais informações em
\url{https://www.unesco.org/en/open-educational-resourcess}} para se referir a
qualquer material destinado ao ensino e aprendizagem que
esteja no domínio publico ou sob direitos autorais regidos por uma licença
aberta que permita acesso gratuito, reutilização, adaptação e redistribuição do
material
(e.g., \href{https://creativecommons.org/licenses/by/4.0/}{Creative Commons Attribution}).
No Brasil, o uso obrigatório de REAs foi adotado pelo Sistema Universidade
Aberta do Brasil em 2016\footnote{Segundo a página
\url{https://www.gov.br/capes/pt-br/acesso-a-informacao/acoes-e-programas/educacao-a-distancia/universidade-aberta-do-brasil/recursos-educacionais-abertos/}
(acessada em 11/02/2023)}
e a CAPES criou o portal \url{https://educapes.capes.gov.br} para indexar REAs
produzidos por instituições brasileiras que oferecem cursos a distância.
Segundo a definição acima, produzo recursos educacionais abertos desde minha
primeira experiência de ensino em 2012 e a criação da apostila
``Tópicos de inversão em geofísica'' \citep{OliveiraJr2012}.
Todo o material que crio para uso nas minhas disciplinas e cursos de curta
duração estão disponíveis livremente com licenças
\href{https://creativecommons.org/licenses/by/4.0/}{Creative Commons Attribution}
ou
\href{https://opensource.org/licenses/BSD-3-Clause}{BSD}/\href{https://opensource.org/licenses/MIT}{MIT}
(para o código fonte).
Também sou o autor de dois tutoriais publicados na revista
\href{https://library.seg.org/journal/leedff}{The Leading Edge} que visam
explicar de maneira interativa conceitos básicos de geofísica.

Acredito que todo material educacional produzido por instituições públicas deve
ser disponibilizado livremente para benefício da população.
Além disso, compartilhar recursos educacionais entre professores e instituições
tem o potencial de elevar o ensino de todos os envolvidos.
A colaboração na produção de recursos possibilita a criação de material de
qualidade superior do que poderia ser atingida por uma única pessoa.
Essa cultura de colaboração em recursos abertos, como é feito no âmbito
de software livre, não é comum no ensino superior.
Por isso, organizei o evento
\href{https://hackmd.io/@leouieda/uk-geo-code-meetup}{Geo+Code} em Novembro de
2022 com meu financiamento do Software Sustainability Institute
(seção~\ref{sec_ssi}).
O principal objetivo do evento era juntar geocientistas do Reino Unido com um
interesse em ciência aberta e dar início a colaborações.
Durante o evento, demos início a criação de um livro aberto sobre geofísica
aplicada utilizando recursos computacionais e dados abertos.
O livro, ainda em estágio de planejamento, será desenvolvido no repositório
do GitHub \url{https://github.com/GeophysicsLibrary/applied-geophysics},
hospedado na organização \href{https://github.com/GeophysicsLibrary}{Geophysics Library}
que fundei em 2018 para agregar REAs voltados à geofísica.

No futuro, pretendo investir na criação de livros abertos na Geophysics
Library.
Pretendo utilizar para isso o material didático que desenvolvi para minhas
disciplinas da University of Liverpool e da UERJ.
Tenho planos de organizar outras edições do Geo+Code internacionalmente,
para as quais já possuo parceiros interessados no
\href{https://www.icrag-centre.org/}{iCRAG} na Irlanda e na
\href{https://www.gla.ac.uk/}{University of Glasgow} na Escócia.
Meu objetivo é criar uma comunidade internacional de geocientistas dedicados a
criação de recursos educacionais abertos com ênfase no uso da computação e de
dados abertos para facilitar a aprendizagem.

%==============================================================================
\chapter{Atividades de Ensino, Mentoria e Extensão}
\label{cap_ensino}

\begin{figure}[h]
  \HeroFigPad
  \begin{center}
    \includegraphics[width=\textwidth]{images/agu-2019-git-lesson.jpg}
  \end{center}
  \caption{
    Foto tirada durante o curso ``Best Practices for Developing and Sustaining
    Your Open-Source Research Software'' que ministrei durante o
    \href{https://github.com/agu-ossi/2019-agu-oss}{AGU Fall Meeting de 2019}
    em São Francisco, E.U.A.
  }
\end{figure}
\begin{summarybox}[frametitle=\faChalkboardTeacher{}\quad Resumo das atividades]
  \begin{fa-ul}
    \faUserGraduate & Orientações concluídas: 11 de graduação, 1 de mestrado, 1
      coorientação de doutorado \\
    \faUser & Orientações em andamento: 1 de graduação, 1 de doutorado, 1
      coorientação de doutorado \\
    \faChalkboardTeacher & 10 disciplinas de graduação ministradas \\
    \faClock & 17 cursos de curta duração ministrados internacionalmente \\
    \faCheckSquare & Habilitação em pedagogia e técnicas de ensino aplicadas ao
      ensino superior \\
    \faLightbulb & Tópicos ensinados incluem: gravimetria, magnetometria,
    sismologia, sensoriamento remoto, métodos numéricos, programação em Python,
    métodos de campo em geofísica, introdução à geologia, problemas inversos,
    geofísica global e geodinâmica da litosfera.
  \end{fa-ul}
\end{summarybox}

As atividades de ensino e mentoria de alunos são onde encontro a maior
satisfação profissional.
Quase nenhuma outra atividade oferece a oportunidade de ter um impacto positivo
direto na vida de outras pessoas.
Minha abordagem para o ensino é muito influenciada pela minha experiência na
graduação (seção~\ref{sec_usp}) e minhas atividades de ciência aberta
(capítulo~\ref{cap_cienciaaberta}).
Utilizo a computação nas minhas aulas para empoderar os alunos com as
habilidades que necessitam para explorar conceitos e dados reais de maneira
independente.
Essa abordagem se mostrou particularmente eficaz em conjunto com uma sólida
base teórica, visualizações interativas e uma seleção de dados abertos
disponíveis para os alunos.
Este capítulo relata minhas atividades de ensino, mentoria e extensão,
incluindo minha abordagem pedagógica e meu papel na criação e ministração de
disciplinas de graduação.
Minha atuação como coordenador do curso de graduação na University of Liverpool
é discutida na seção~\ref{sec_liverpool}.

\section{Orientações}
\label{sec_orientacao}

Minha primeira experiência como mentor de um jovem cientista foi a coorientação
do aluno \SantiagoLink{} junto com o
Professor Mario E. Giménez da Universidad Nacional de San Juan, Argentina,
entre 2016 e 2022.
O primeiro contato que tive com o Santiago foi através do software
Fatiando a Terra (seção~\ref{sec_fatiando}).
Em 2015, ele começou a se voluntariar com o projeto, implementando funções
para manejar dados em malhas regulares e o cálculo de espectros de potência.
Quando ele e o Mario me convidaram para coorientar sua tese de doutorado em
2016, fiquei feliz em aceitar continuar trabalhando com ele de maneira mais
regular.
Inicialmente, a proposta era que eu fosse seu orientador principal.
Mas como essa seria minha primeira orientação e seria feita inteiramente de
forma remota, achei mais prudente começar como coorientador.
O Santiago fez excelente progresso nas linha de pesquisa em modelagem com
tesseroides (seção~\ref{sec_modelagemdireta}) e camada equivalente
(seção~\ref{sec_eql}) durante seu doutorado.
Seu envolvimento no Fatiando a Terra aumentou com o tempo. Hoje em dia ele
ocupa uma posição de liderança no projeto, estabelecendo direções para o
desenvolvimento e atuando como mentor para novos membros da comunidade.
Ele é o criador e principal desenvolvedor de duas novas componentes do
Fatiando: \href{https://www.fatiando.org/harmonica}{Harmonica} e
\href{https://www.fatiando.org/choclo/}{Choclo}.
O Santiago também foi fundamental no estabelecimento do nosso grupo de pesquisa,
o \href{https://www.compgeolab.org/}{Computer-Oriented Geoscience Lab}
(CompGeoLab).
Nossas longas conversas sobre o papel da computação na ciência,
reprodutibilidade dos resultados numéricos e formas de fazer ciência aberta
foram as maiores influências que tive para estabelecer os princípios do
CompGeoLab, codificados em nosso manual
\url{https://github.com/compgeolab/manual}.
Em 2022, o Santiago defendeu sua tese de doutorado\footnote{Disponível em
\url{https://github.com/santisoler/phd-thesis}} com sucesso e continua
colaborando com o CompGeoLab regularmente.
Vale a pena apontar que, durante esses 6 anos de orientação, eu e o Santiago
nunca nos encontramos em pessoa por conta de diversos desencontros e a pandemia
de COVID.
Aprendemos juntos como criar uma relação próxima de trabalho inteiramente
online usando uma mistura de chamadas por vídeo semanais e mensagens regulares
pela plataforma Slack.
Atualmente, o Santiago está fazendo um pós-doutorado na University of British
Columbia (UBC), Canadá, sob supervisão da Professora
\href{https://lindseyjh.ca/}{Lindsey J. Heagy}.
Seu projeto inclui trabalhar no software livre
\href{https://simpeg.xyz/}{SimPEG}, desenvolvido pelo grupo da UBC, e facilitar
interações entre o SimPEG e o Fatiando a Terra.
A experiência de trabalhar com o Santigo foi um dos maiores privilégios da
minha carreira.
Santiago é um pesquisador brilhante, um engenheiro de software excepcional
e um verdadeiro amigo.

Iniciei minha segunda experiência de orientação a nível de doutorado em 2021
com a aluna \IndiaLink{} da University of Liverpool, coorientada pelo Professor
\href{https://www.pinga-lab.org/people/oliveira-jr.html}{Vanderlei C. Oliveira Jr.}
do Observatório Nacional e pelo Professor
\href{https://www.liverpool.ac.uk/~holme/}{Richard Holme} da University of
Liverpool.
A India foi nossa aluna do curso de Bacharelado em Geofísica de Liverpool e era
uma das melhores alunas de sua turma.
Ela foi selecionada pela School of Environmental Sciences para uma posição
dupla: meio período como aluna de doutorado e meio período empregada como
\textit{Graduate Teaching Assistant}, auxiliando no ensino de disciplinas de
graduação dos cursos de geociências e geografia física.
Acho importante realçar que a seleção para a vaga foi feita inteiramente
baseada no mérito da India e não levou em consideração o projeto de doutorado
ou os orientadores.
Seu projeto de doutorado iniciou uma nova linha de pesquisa para o CompGeoLab
onde buscamos aprimorar o uso de dados de aeromagnéticos para a estimativa do
fluxo geotermal na Antártica (seção~\ref{sec_antartica}).
Através desse projeto, nos envolvemos com a organização
\href{https://www.scar.org/}{Scientific Committee on Antarctic Research}
(SCAR), especificamente no grupo
\href{https://www.scar-instant.org/index.php/research-themes/theme-2-solid-earth-ice-interactions/sc1-antarctic-geothermal-heat-flux}{SC1
Antarctic Geothermal Heat Flux}.
A India também estabeleceu contato com pesquisadores da British Antarctic
Survey, que criaram a compilação de dados magnéticos antárticos ADMAP2
\citep{Golynsky2018}, para obter os dados brutos e consultá-los sobre os
detalhes do processamento que fizeram.
A India é uma pesquisadora com atitude excepcionalmente profissional e
com iniciativa própria, além de ser dedicada e inteligente.
Seu projeto ainda está em fase inicial mas tenho certeza de que ela produzirá
trabalhos excelentes.

Também em 2021, fui convidado pelo Professor Ricardo I. F. Trindade a
coorientar o aluno de doutorado \GelsonLink{} da Universidade de São Paulo.
Eu e o Ricardo havíamos conversado durante encontros no AGU Fall Meeting sobre
algumas ideias de adaptar técnicas de métodos potenciais, como a deconvolução
de Euler (seção~\ref{sec_euler}), a dados de microscopia magnética.
O Gelson se juntou ao CompGeoLab para dar início a essa nova linha de pesquisa
(seção~\ref{sec_micromag}), me trazendo de volta ao assunto da minha primeira
iniciação científica na USP (seção~\ref{sec_usp}).
Em apenas um ano, o Gelson já foi bem sucedido na adaptação da deconvolução de
Euler e o método de \citet{OliveiraJr2015}, junto com técnicas de processamento
de imagens que aprendi com minha disciplina de sensoriamento remoto em
Liverpool (seção~\ref{sec_ensino_grad}), para estimar o momento de dipolo
de grãos individuais de minerais ferromagnéticos.
Em 2022, fomos contemplados com um projeto da
\href{https://royalsociety.org/}{Royal Society} para realizar intercâmbios
entre a São Paulo e Liverpool para dar início à colaboração entre as duas
instituições.
Até o momento, utilizamos o financiamento da Royal Society para trazer o Gelson
para Liverpool e trabalharmos em seu primeiro artigo, que está nos processos
finais de escrita antes da submissão.
Orientar o Gelson tem sido um verdadeiro privilégio.
Ele é respeitoso, dedicado, inteligente, possui uma base sólida em geociências,
tem iniciativa própria e aprende rapidamente conceitos novos que estão fora de
sua zona de conforto.
Não tenho dúvida de que nossa colaboração produzirá pesquisa de alto impacto
nessa linha emergente.

Fui o orientador de 11 trabalhos de conclusão de curso e uma dissertação de
mestrado do curso de geofísica da University of Liverpool, com alunos
atuando em diversas linhas de pesquisa\footnote{Uma lista completa dos alunos
e seus projetos está disponível no meu Currículo Lattes: \url{https://lattes.cnpq.br/\Lattes}}.
Aprendi com essas orientações como guiar alunos sem experiência prévia em
pesquisa pelas etapas iniciais de um projeto e como ajustar o nível dos
projetos propostos com o nível dos alunos no final de um curso de graduação no
Reino Unido.
Foi particularmente gratificante observar os alunos progredirem durante seus
projetos e produzirem trabalhos de conclusão com uma qualidade muito acima do
que eles achavam que seriam capazes.


\section{Cursos de curta duração}
\label{sec_workshops}

\begin{subsummarybox}[frametitle=\faClock{}\quad Cursos e workshops ministrados online]
  \begin{paperlist}
    2022 &
      Crafting beautiful maps with PyGMT.
      \textit{EGU General Assembly}.
      \GitHub{GenericMappingTools/egu22pygmt}
      \\
    ~ &
      A geophysical tour of mid-ocean ridges.
      \textit{Transform 2022} (online).
      \GitHub{leouieda/transform2022}.
      \YouTube{NzJmRlJCNbQ}
      \\
    2021 &
      The Generic Mapping Tools for Geodesy.
      \textit{UNAVCO} (online).
      \GitHub{GenericMappingTools/2021-unavco-course}
      \\
    2020 &
      Let's build a geophysical inversion with Python.
      \textit{IRTG-2379 Graduate School: Modern Inverse Problems},
      \textit{RWTH Aachen University} (online).
      \GitHub{compgeolab/2020-aachen-inverse-problems}
      \\
    ~ &
      The Generic Mapping Tools for Geodesy.
      \textit{UNAVCO} (online).
      \GitHub{GenericMappingTools/2020-unavco-course}.
      \YouTube{EQgxDmCXvj4}
      \\
    ~  &
      From scattered data to gridded products using Verde.
      \textit{Transform 2020} (online).
      \GitHub{fatiando/transform2020}.
      \YouTube{-xZdNdvzm3E}
  \end{paperlist}
\end{subsummarybox}
\begin{subsummarybox}[frametitle=\faClock{}\quad Cursos e workshops ministrados presencialmente]
  \begin{paperlist}
    2019 &
      Best Practices for Developing and Sustaining Your Open-Source Research Software.
      \textit{AGU Fall Meeting}.
      \GitHub{agu-ossi/2019-agu-oss}
      \\
    ~  &
      Become a Generic Mapping Tools Contributor Even If You Can't Code.
      \textit{AGU Fall Meeting}
      \\
    ~  &
      The Generic Mapping Tools for Geodesy.
      \textit{Scripps Institution of Oceanography} and \textit{UNAVCO}.
      \GitHub{GenericMappingTools/2019-unavco-course}.
      \YouTube{uPUt4\_kd6m8}
      \\
    ~  &
      Introduction to Python Workshop (Earth Sciences REU program).
      \textit{Department of Geology and Geophysics, }.
      \GitHub{leouieda/2019-06-reu-python}
      \\
    2018 &
      Best Practices for Modern Open-Source Research Codes.
      \textit{AGU Fall Meeting}.
      \GitHub{agu-ossi/2018-agu-oss}
      \\
    ~  &
      Git and GitHub: What are their uses? Are they worth the effort? Let's find out!
      \textit{ASPRS UHM Student Chapter, }
      \\
    2017 &
      Introduction to Python.
      \textit{Department of Geology and Geophysics, }.
      \GitHub{leouieda/python-hawaii-2017}
      \\
    2016 &
      Python for Geologists (SAGEO).
      \textit{Faculdade de Geologia, }.
      \GitHub{leouieda/python-geologia-2016}
      \\
    ~  &
      Python como uma ferramenta numérica em Ciências da Terra: uma nova
      abordagem de programação.
      \textit{XVIII Escola de Verão de Geofísica do IAG-USP}.
      \GitHub{leouieda/verao2016}
      \\
    2014 &
      Tópicos de inversão em geofísica.
      \textit{III Semana de Geofísica da UnB}.
      \GitHub{pinga-lab/inversao-unb-2014}
      \\
    2012 &
      Tópicos de inversão em geofísica.
      \textit{XVI Escola de Verão de Geofísica do IAG-USP}.
      \GitHub{pinga-lab/inversao-iag-2012}
  \end{paperlist}
\end{subsummarybox}

Minha primeira experiência com o ensino foi através do curso ``Tópicos de
inversão em geofísica'' que ministrei em 2012 junto com meu amigo e então
colega de doutorado
\href{https://www.pinga-lab.org/people/oliveira-jr.html}{Vanderlei C. Oliveira Jr.}
na XVI Escola de Verão de Geofísica do IAG-USP.
Foi durante esse curso que percebi minha paixão pelo ensino e decidi seguir a
carreira acadêmica para poder combinar ensino, pesquisa e extensão.
Desde então, ministrei diversos cursos de curta duração e workshops em formato
online e presencial.
Esses cursos complementam o ensino tradicional em disciplinas de graduação e
pós-graduação, fornecendo a oportunidade de experimentar com tecnologias,
formatos de ensino e tópicos pouco tradicionais.

A maioria dos cursos que ministrei estão relacionados à programação. O formato
curto é adequado para uma introdução à conceitos básicos de programação ou
para abordar um assunto específico (e.g., como criar mapas com o
\href{https://www.pygmt.org}{PyGMT},
como interpolar dados com o \href{https://www.fatiando.org/verde}{Verde}
ou como criar testes unitários para seu software).
Por isso, acho as ``escolas de verão'' e ``semanas da geofísica'' organizadas
pelas universidades tão proveitosas.
Esses cursos também podem fornecer aos alunos um contato com especialistas de
todo o mundo.
Esse contato pode inclusive ser feito com um orçamento limitado
devido aos avanços recentes nas plataformas de vídeo conferência e a difusão de
atividades online causados pela pandemia de COVID.


\section{Disciplinas de graduação}
\label{sec_ensino_grad}

\begin{subsummarybox}[frametitle=\faGraduationCap{}\quad Disciplinas ministradas na ]
  \begin{courselist}
    2015--2016 &
      IME03-1366 Matemática Especial I.
      \newline
      \GitHub{mat-esp/about}
      \\
    2014--2016 &
      FGEL04-12422 Geofísica II.
      \newline
      \GitHub{leouieda/geofisica2}
      \\
    ~ &
      FGEL04-12421 Geofísica I.
      \newline
      \GitHub{leouieda/geofisica1}
      \\
    2015 &
      FGEL01-00805 Geologia Geral I.
  \end{courselist}
\end{subsummarybox}
\begin{subsummarybox}[frametitle=\faGraduationCap{}\quad Disciplinas ministradas na University of Liverpool]
  \begin{courselist}
    2023--atual  &
      ENVS219: Earth and Environmental Data Science (\textit{em
      desenvolvimento}).
      \\
    2020--atual  &
      ENVS398: Global Geophysics and Geodynamics.
      \newline
      \GitHub{leouieda/lithosphere}
      \\
    ~ &
    ENVS258: Environmental Geophysics.
      \newline
      \GitHub{leouieda/remote-sensing}.
      \newline
      \GitHub{leouieda/gravity-processing}.
      \\
    ~ &
    ENVS386: Geophysical Data Modelling.
      \newline
      \GitHub{leouieda/ml-intro}.
      \\
    ~ &
      ENVS101/106: Study Skills and GIS (tutorial).
      \\
    2019--2021 &
      ENVS123: Introduction to Geoscience and Earth History.
      \\
    2019--2020  &
      ENVS362: Geophysics Field School.
  \end{courselist}
\end{subsummarybox}

Na , tive a oportunidade de criar o conteúdo de três disciplinas de
graduação: Matemática Especial I e Geofísica I e II.
A disciplina Matemática Especial I pertence ao curso de Bacharelado em
Oceanografia e cobria tópicos avançados de matemática.
Meu papel ao assumir essa disciplina era convertê-la em uma introdução à
programação em Python e ao cálculo numérico.
Decidi incluir no início da disciplina uma introdução ao software de controle
de versão \href{https://git-scm.com/}{git} e à plataforma
\href{https://github.com/}{GitHub}.
Assim, pude manejar a disciplina inteiramente pelo GitHub, com cada lição
sendo armazenada em um repositório da organização
\url{https://github.com/mat-esp}.
Durante as aulas práticas, os alunos se dividiam em grupos e cada grupo era
automaticamente fornecido com uma cópia do repositório da lição pela plataforma
\href{https://classroom.github.com/}{GitHub Classroom}.
Ao final da aula, os alunos submetiam suas soluções para a tarefa da lição
também pelo GitHub, onde recebiam as notas e correções.

As disciplinas de geofísica, cobrindo uma introdução aos métodos geofísicos,
são parte do curso de Bacharelado em Geologia e haviam acabado de serem
reformuladas quando assumi meu cargo na UERJ em 2014.
Logo, pude criar o conteúdo das disciplinas por conta própria e estabelecer
como gostaria que fossem estruturadas.
Optei por dividi-las entre aulas teóricas e aulas práticas computacionais.
Nas práticas, utilizei o software \href{https://jupyter.org/}{Jupyter} para
criar \textit{notebooks} que explicavam os conceitos abordados em aula
utilizando uma combinação de texto, equações, código pronto para demonstrar os
conceitos, tarefas para serem executadas através de visualizações interativas
(figura~\ref{fig_notebooksismica}) e perguntas para serem respondidas como
parte da avaliação somativa da disciplina.
Essa abordagem foi bem recebida pelos alunos.
Inclusive, fui escolhido como paraninfo da turma de formandos da Geologia em
2016 (ano de ingresso 2012).

\begin{figure}[t]
  \begin{center}
    \includegraphics[width=\textwidth]{images/seismic-waves-demo.jpg}
  \end{center}
  \caption{
    Exemplo de um notebook usado na minha disciplina ``Geofísica 2'' da UERJ
    para ensinar o conceito de ondas sísmicas, reflexão, refração e conversão
    de ondas P em ondas S ao interagir com uma interface geológica. O notebook
    contém instruções, teoria, perguntas e código pronto que os alunos
    podem executar e modificar para criar animações da propagação de ondas
    elásticas (utilizando o código de diferenças finitas do Fatiando a Terra).
    A figura no notebook é parte de uma animação da propagação de uma onda P
    que incide sobre uma interface, gerando ondas P e S refletidas e
    refratadas.
    As cores representam a soma do divergente e o rotacional do campo de
    deformações, mostrando as frentes de onda.
    Vetores indicam o deslocamento de cada ponto do modelo, com ondas P e S
    tendo deslocamento perpendicular e paralelo às frentes de onda,
    respectivamente.
  }
  \label{fig_notebooksismica}
\end{figure}

Na University of Liverpool, participei de diversas disciplinas dos cursos de
Geofísica e Geologia, ministrando
programação em Python para Ciências da Terra em ``ENVS101
Study Skills and GIS'',
introdução à estrutura da Terra e isostasia em ``ENVS123 Introduction to
Geoscience and Earth History'',
inversão não-linear e aprendizagem de máquinas em ``ENVS386 Geophysical Data
Modelling''
e a matéria de campo do terceiro ano de geofísica
``ENVS362 Geophysics Field School''.
Continuo com a abordagem computacional que desenvolvi na UERJ, dessa vez
incluindo tarefas onde os alunos devem escrever parte do código.
Atualmente, adotei a metodologia de
\href{https://pt.wikipedia.org/wiki/Aula_invertida}{aula invertida}, produzindo
vídeos explicando a base teórica para os alunos assistirem independentemente e
utilizando todo o tempo em sala de aula para atividades práticas com os
notebooks e discussões.

Criei a nova disciplina optativa ``ENVS398: Global Geophysics and Geodynamics''
junto com o Professor
\href{https://www.liverpool.ac.uk/environmental-sciences/staff/andrew-biggin/}{Andy Biggin}.
Utilizamos aulas gravadas para ensinar o conteúdo teórico.
As partes práticas da disciplina são dividas em duas partes.
Durante metade da disciplina, ministrada pelo Andy, os alunos aprendem sobre o
núcleo e o manto terrestre, discutindo artigos recentes da literatura durante
as aulas presenciais.
Durante a outra metade da disciplina, ministrada por mim, os alunos aprendem
sobre a geodinâmica da litosfera.
Nas aulas práticas, os alunos desenvolvem a implementação computacional dos
modelos abordados nas aulas teóricas e os comparam a dados reais.
Utilizo os notebooks com parte do seu código fornecido por mim para os alunos
construírem suas soluções em etapas gradativamente mais desafiadoras.
Também utilizo um conjunto global de dados de distúrbios da gravidade, fluxo de
calor geotermal, topografia e idade do assoalho oceânico para os alunos
interpretarem e processarem livremente.
Essa matéria foi consistentemente elogiada pelos alunos nos formulários de
avaliação semestrais das disciplinas.

Sou o responsável pela disciplina ``ENVS258 Environmental Geophysics'' onde
ensino uma introdução ao sensoriamento remoto, processamento e aquisição de
dados de gravidade e com uma componente de campo, onde introduzimos aos alunos
os equipamentos que possuímos em Liverpool (magnetometria, GRP,
eletrorresistividade, GPS, EM-31 e refração sísmica).
Desenvolvi todo o material prático para a componente de sensoriamento remoto,
utilizando novamente os notebooks e dados abertos dos satélites Landsat
fornecidos na plataforma \href{https://earthexplorer.usgs.gov/}{EarthExplorer}
da USGS.
A avaliação somativa dessa componente é um relatório onde os alunos escolhem um
tema dentro do escopo da disciplina, fazem a pesquisa bibliográfica, baixam os
dados relevantes do EarthExplorer, processam os dados usando notebooks em
Python e geram suas visualizações e conclusões.
Para a grande maioria dos alunos, essa é sua primeira tarefa independente e o
seu primeiro contato com a pesquisa.
A qualidade e criatividade dos relatórios que os alunos produzem frequentemente
me surpreende.
Em múltiplas ocasiões, incorporei o trabalho de alunos nas minhas aulas porque
eram simplesmente superiores aos exemplos que eu pude desenvolver\footnote{Por
exemplo, as práticas
\url{https://github.com/leouieda/remote-sensing/blob/main/practicals/practical2.ipynb}
e
\url{https://github.com/leouieda/remote-sensing/blob/main/practicals/practical4.ipynb}}.
A componente de sensoriamento remoto é sempre elogiada pelos alunos nas
avaliações do curso, o que me dá muito orgulho porque foi um tema que aprendi
quase inteiramente através de ministrar essa disciplina.

Estou criando a disciplina ``ENVS229 Earth and Environmental Data Science'' com
os Professores \href{https://www.liverpool.ac.uk/environmental-sciences/staff/ben-edwards/}{Ben Edwards}
e \href{https://www.liverpool.ac.uk/environmental-sciences/staff/greig-paterson/}{Greig Paterson}.
Essa disciplina fornecerá conhecimentos intermediários de programação em Python
e técnicas de estatística e análise de dados geocientíficos para todos os
cursos de Ciências da Terra do departamento.
Isso é atualmente possível porque em 2021 introduzi um curso de programação em
Python na disciplina do primeiro ano ``ENVS101 Study Skills and GIS'',
fornecendo aos alunos a base necessária para cursar a nova disciplina.
Nosso desejo de criar essa disciplina foi motivado pela falta de treinamento
que os nossos alunos de geologia e geografia física recebem em análise de
dados, atividade que atualmente é altamente valorizada por empregadores em
geociências e ciência de dados.

\section{Atividades de Extensão}

Minhas atividades na área de extensão universitária são mais limitadas que
minha atuação em outras áreas.
Parte da razão é meu foco adicional em outras atividades, como minha atuação
em ciência aberta (capítulo~\ref{cap_cienciaaberta}).
Por conta da pandemia de COVID de 2020, muitas das atividades de extensão que
eram promovidas pela universidade estavam canceladas durante a maior parte da
minha estadia em Liverpool.
Em 2022, com a retomada das atividades presenciais, participei de três eventos
na University of Liverpool chamados de \textit{Open Days}, durante os quais
os alunos de ensino médio visitam a universidade para conhecer mais sobre os
cursos oferecidos.
Na parte das Ciências da Terra, fizemos demonstrações sobre como a viscosidade
influencia o fluxo de lava, como o Ground Penetrating Radar (GPR) detecta
objetos em subsuperfície e como o conceito de isostasia explica a espessura
crustal em regiões montanhosas.
Além dos Open Days, fui voluntário no evento \textit{Hour of Code} para ensinar
programação para alunos de ensino fundamental da Salt Lake Elementary School em
Honolulu, E.U.A.
Também fui entrevistado nos podcasts de divulgação das geociências
\href{https://undersampledrad.io}{Undersampled Radio}
(episódio ``\href{https://undersampledrad.io/home/2016/7/open-sourcery}{Open
Sourcery}'' de 19/05/2016)
e \href{https://www.dontpanicgeocast.com/166}{Don't Panic Geocast}
(episódio ``\href{https://www.dontpanicgeocast.com/166}{You are headed to a warm and sunny place}''
de 27/04/2018).
Essas atividades são extremamente gratificantes e gostaria de dedicar mais
tempo a elas no futuro.
Em particular, gostaria de expandir uma atividade que desenvolvi para explicar
aquisições aeromagnéticas utilizando ímãs enterrados e o aplicativo de celular
\href{https://phyphox.org/}{Phyphox}, que dá acesso aos dados registrados pelos
magnetômetros dos celulares modernos.


\chapter{Projeto de Atuação Profissional}

Além de ministrar aulas, pretendo desenvolver projetos com IA para auxiliar startups e empresas júnior da cidade na
implementação de modelos generativos em seus serviços e produtos. O objetivo é ensinar estratégias para o uso de IAs
quantizadas e de baixo custo operacional, permitindo a implementação diretamente no cliente e, assim, reduzindo os
custos para as startups em especial as empresas do Parque Tecnológico do IMD.

Também planejo iniciar projetos pilotos para o uso de IA em diversas áreas, tais como:
Inclusão Digital para Idosos: Ajudar idosos a navegar no ambiente digital com segurança atravez de cursos voltados ao uso de assistentes virtuais. Reduzindo sua vulnerabilidade a informações falsas e promovendo sua inclusão digital.
Tutoria Virtual para Estudantes: Implementar pipilenes de modelos para serem tutores virtuais para dar suporte personalizado aos alunos em disciplinas específicas, ajudando a melhorar o desempenho acadêmico e a reduzir a carga de trabalho dos professores.
Secretários Virtuais Acadêmicos: Desenvolver secretários virtuais para simplificar processos acadêmicos, facilitando a vida dos alunos e aumentando a eficiência no trabalho dos servidores.
Gestão Inteligente de Energia e Resíduos: Implementar modelos inteligentes para otimizar o uso de energia no campus, alertando para luzes ou ar-condicionado deixados ligados em salas, e para uma coleta de lixo eficiente, com notificações automáticas para agilizar o recolhimento.
Apoio a acessibiliade: Auxiliar  o desenvolvimento soluções que auxiliem a acessibilidade como cães guias virtuais para pessoas com deficiencia visual, leitores de libras mais avançados para deficientes auditivos e assistentes para pessoas com dificuldade de socialização.


%==============================================================================
\chapter{Conclusão}
\label{cap_conclusao}

Minha trajetória acadêmica e profissional é marcada pela busca constante por aprendizado, inovação e contribuição à
sociedade por meio da tecnologia. Cada etapa, desde meus primeiros projetos na graduação até os desafios atuais no
doutorado, moldou minhas competências e consolidou minha paixão pela docência e pela pesquisa aplicada.

Minha motivação para atuar no IMD está diretamente ligada ao potencial de impactar positivamente a formação de
profissionais e à oportunidade de desenvolver soluções práticas para desafios reais. Acredito que minha experiência
em áreas como sistemas interativos, ensino a distância e inteligência artificial, combinada com meu comprometimento
com a inclusão digital e acessibilidade, me permitirá agregar valor ao Instituto e contribuir para o sucesso de
iniciativas como o IA360.

Estou determinado a continuar expandindo o conhecimento e promovendo inovações que transcendam a academia, criando um
impacto duradouro na indústria, na sociedade e no ambiente educacional. Este memorial reflete não apenas o percurso
que me trouxe até aqui, mas também minha visão de futuro e o desejo de construir um legado de transformação e excelência.



%==============================================================================
\backmatter
\bibliographystyle{apalike-doi}
\bibliography{references}

\end{document}

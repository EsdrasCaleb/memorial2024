%%%%%%%%%%%%%%%%%%%%%%%%%%%%%%%%%%%%%%%%%%%%%%%%%%%%%%%%%%%%%%%%%%%%%%%%%%%%%%%
% Memorial para concurso público de Professor Doutor na USP.
%
% Formatação inspirada em:
% * https://tug.org/pracjourn/2008-1/mori/mori.pdf
% * https://github.com/santisoler/phd-thesis
% * https://github.com/compgeolab/dissertation-template
%%%%%%%%%%%%%%%%%%%%%%%%%%%%%%%%%%%%%%%%%%%%%%%%%%%%%%%%%%%%%%%%%%%%%%%%%%%%%%%

%%%%%%%%%%%%%%%%%%%%%%%%%%%%%%%%%%%%%%%%%%%%%%%%%%%%%%%%%%%%%%%%%%%%%%%%%%%%%%%
% Set a class and import packages
\documentclass[10pt,a4paper,oneside]{book}

% Variables
\newcommand{\Year}{2024}
\newcommand{\Author}{Esdras Caleb Oliveira Silva}
\newcommand{\Title}{Memorial e Projeto de Atuação Profissional de \Author{} para concurso público - Professor do Magistério Superior Adjunto-A - IMD/UFRN}
\newcommand{\Email}{esdrascaleb@gmail.com}
\newcommand{\ORCID}{0000-0001-5232-5067}
\newcommand{\GoogleScholar}{zWxct1EAAAAJ}
\newcommand{\Lattes}{5923627359710427}

% Variables for easier typing of some names
\newcommand{\UFF}{Universidade Federal Fluminense}
\newcommand{\UFRN}{Universidade Federal do Rio Grande do Norte}

% Names for citing coauthors
\newcommand{\Me}{\textbf{Uieda, L}}
\newcommand{\Val}{Barbosa, VCF}
\newcommand{\Bi}{Oliveira Jr, VC}
\newcommand{\Paul}{Wessel, P}
\newcommand{\Joaquim}{Luis, J}
\newcommand{\Remko}{Scharroo, R}
\newcommand{\Florian}{Wobbe, F}
\newcommand{\Walter}{Smith, WHF}
\newcommand{\Dongdong}{Tian, D}
\newcommand{\Bridget}{Smith-Konter, B}
\newcommand{\Eric}{Xu, X}
\newcommand{\David}{Sandwell, DT}
\newcommand{\Carla}{Braitenberg, C}
\newcommand{\Naomi}{Ussami, N}
\newcommand{\Manoel}{D'Agrella-Filho, MS}
\newcommand{\JB}{Silva, JBC}
\newcommand{\Dai}{Sales, DP}
\newcommand{\Figura}{Melo, FF}
\newcommand{\Dio}{Carlos, DU}
\newcommand{\BragaVale}{Braga, MA}
\newcommand{\YLi}{Li, Y}
\newcommand{\Angeli}{Angeli, G}
\newcommand{\Peres}{Peres, G}
\newcommand{\Everton}{Bomfim, EP}
\newcommand{\Eder}{Molina, E}
\newcommand{\Gomes}{Gomes, AAS}
\newcommand{\Santiago}{Soler, SR}
\newcommand{\Agustina}{Pesce, A}
\newcommand{\Gimenez}{Gimenez, ME}
\newcommand{\Kristoffer}{Hallam, KAT}
\newcommand{\Guangdong}{Zhao, G}
\newcommand{\Bo}{Chen, B}
\newcommand{\JLiu}{Liu, J}
\newcommand{\LChen}{Chen, L}
\newcommand{\RGuo}{Guo, R}
\newcommand{\MKaban}{Kaban, MK}
\newcommand{\Lindsey}{Heagy, LJ}
\newcommand{\Lion}{Krischer, L}
\newcommand{\Rene}{Gassmoeller, R}
\newcommand{\Bane}{Sullivan, CB}
\newcommand{\Jens}{Klump, JF}
\newcommand{\LBarba}{Barba, LA}
\newcommand{\JBazan}{Bazan, J}
\newcommand{\JBrown}{Brown, J}
\newcommand{\RGuimera}{Guimera, RV}
\newcommand{\MGymrek}{Gymrek, M}
\newcommand{\AHanna}{Alex Hanna}
\newcommand{\KHuff}{Huff, KD}
\newcommand{\DKatz}{Katz, DS}
\newcommand{\CMadan}{Madan, CR}
\newcommand{\KMoerman}{Moerman, KM}
\newcommand{\KNiemeyer}{Niemeyer, KE}
\newcommand{\JPoulson}{Poulson, JL}
\newcommand{\PPrins}{Prins, P}
\newcommand{\KRam}{Ram, K}
\newcommand{\ARokem}{Rokem, A}
\newcommand{\Arfon}{Smith, AM}
\newcommand{\GThiruvathukal}{Thiruvathukal, GK}
\newcommand{\KThyng}{Thyng, KM}
\newcommand{\BWilson}{Wilson, BE}
\newcommand{\Yehudi}{Yehudi, Y}
\newcommand{\Remi}{Rampin, R}
\newcommand{\Hugo}{van Kemenade, H}
\newcommand{\MattTurk}{Turk, M}
\newcommand{\Shapero}{Shapero, D}
\newcommand{\Anderson}{Banihirwe, A}
\newcommand{\Leeman}{Leeman, J}
\newcommand{\JEbbing}{Ebbing, J}
\newcommand{\AGuy}{Guy, A}
\newcommand{\JFarquharson}{Farquharson, J}
\newcommand{\AKushnir}{Kushnir, A}
\newcommand{\FWadsworth}{Wadsworth, F}
\newcommand{\LPerozzi}{Perozzi, L}
\newcommand{\MWieczorek}{Wieczorek, MA}
\newcommand{\LLi}{Li, L}
\newcommand{\Ricardo}{Trindade, RIF}

% Links to webpages I use often
\newcommand{\SantiagoLink}{\href{https://www.santisoler.com/}{Santiago R. Soler}}
\newcommand{\VanderleiLink}{\href{https://www.pinga-lab.org/people/oliveira-jr.html}{Vanderlei C. Oliveira Jr.}}
\newcommand{\SandwellLink}{\href{https://topex.ucsd.edu/sandwell/}{David Sandwell}}
\newcommand{\ValeriaLink}{\href{https://www.pinga-lab.org/people/barbosa.html}{Valéria C. F. Barbosa}}
\newcommand{\PaulLink}{\href{https://www.soest.hawaii.edu/pwessel/}{Paul Wessel}}
\newcommand{\IndiaLink}{\href{https://www.compgeolab.org/team/\#indiauppal}{India Uppal}}
\newcommand{\GelsonLink}{\href{https://www.compgeolab.org/team/\#Souza-junior}{Gelson Ferreira de Souza Junior}}

\newcommand{\GMTLink}{\href{https://www.generic-mapping-tools.org}{Generic Mapping Tools}}
\newcommand{\CompGeoLabLink}{\href{https://www.compgeolab.org/}{Computer-Oriented Geoscience Lab}}
\newcommand{\SwunngLink}{\href{https://softwareunderground.org/}{Software Underground}}
\newcommand{\FatiandoLink}{\href{https://www.fatiando.org}{Fatiando a Terra}}
\newcommand{\PyGMTLink}{\href{https://www.pygmt.org}{PyGMT}}
\newcommand{\SSILink}{\href{https://software.ac.uk/}{Software Sustainability Institute}}

% Import packages
\usepackage[utf8]{inputenc}
\usepackage[T1]{fontenc}
\usepackage[brazil]{babel}
\usepackage{geometry}
\usepackage{graphicx}
\usepackage{amssymb}
\usepackage{amsmath}
\usepackage{mathpazo}
\usepackage{hyperref}
% create fancy headers
\usepackage{fancyhdr}
% commands for managing dates and its formats
\usepackage{datetime2}
% improved urls with proper hyphenation
\usepackage{xurl}
% Control over enumerate and itemize
\usepackage{enumitem}
% Tweak the look of captions
\usepackage{caption}
% To control the style of section titles
\usepackage{titlesec}
% Add the bibliography to the table of contents
\usepackage[nottoc,chapter]{tocbibind}
\usepackage[round,authoryear,sort]{natbib}
% show dois as links on references
\usepackage{doi}
% Icon and fonts (requires using xelatex or luatex)
\usepackage{fontawesome5}
\usepackage{academicons}
\usepackage{fontspec}
\usepackage[mono]{notomath}
% To make everything neater
\usepackage{microtype}
% To make fancy text boxes
\usepackage{xcolor}
\usepackage[framemethod=default]{mdframed}
% For fancy and multipage tables
\usepackage{tabularx}
\usepackage{ltablex}
% To define custom environments
\usepackage{environ}
\usepackage{setspace}
% Reference sections by name
\usepackage{nameref}
% Better handling of footnotes inside summary boxes
\usepackage{footmisc}
%%%%%%%%%%%%%%%%%%%%%%%%%%%%%%%%%%%%%%%%%%%%%%%%%%%%%%%%%%%%%%%%%%%%%%%%%%%%%%%

%%%%%%%%%%%%%%%%%%%%%%%%%%%%%%%%%%%%%%%%%%%%%%%%%%%%%%%%%%%%%%%%%%%%%%%%%%%%%%%
% Configuration of the document

\geometry{%
  left=30mm,
  right=30mm,
  top=20mm,
  bottom=15mm,
  headsep=5mm,
  headheight=5mm,
  footskip=10mm,
  includehead=true,
  includefoot=true
}

% Increase the line spacing
\SetSinglespace{1.2}
\onehalfspacing

% Remove spacing between enumerate/itemize items
\setlist{nosep}

% Padding between the first figure and the chapter title
\newcommand{\HeroFigPad}{\vspace{-1cm}}

% Padding before the software logo figures
\newcommand{\SoftwareFigPad}{\vspace{-0.3cm}}

% Add a link to a DOI
\newcommand{\DOI}[1]{\url{https://doi.org/#1}}

% Add a link to a GitHub repository
\newcommand{\GitHub}[1]{\faGithub{} Código: \url{https://github.com/#1}}

% Add a link to a YouTube video
\newcommand{\YouTube}[1]{\faYoutube{} Vídeo: \url{https://youtu.be/#1}}

% Add a link to a supplementary data
\newcommand{\Data}[1]{\faChartBar{} Dados: \url{https://doi.org/#1}}

% Add a link to a preprint
\newcommand{\Preprint}[1]{\faLockOpen{} Preprint: \url{https://doi.org/#1}}

% Make a Unicode bullet symbol
\newcommand{\Bullet}{•\enspace}

% Define custom colors
\definecolor{lu_gray}{gray}{0.98}
\definecolor{lu_darkgray}{gray}{0.3}
\definecolor{lu_blue}{RGB}{32, 96, 194}
\definecolor{lu_lightblue}{RGB}{238, 245, 250}
\definecolor{lu_yellow}{RGB}{255, 193, 7}
\definecolor{lu_lightyellow}{RGB}{255, 249, 230}

% Customize how Chapter headings are displayed
\titleformat{\chapter}[display]{\normalfont}{\large Capítulo \thechapter}{0pt}{\huge}[\titlerule]
\titlespacing*{\chapter}{0pt}{-40pt}{40pt}

% Set the spacing between bibliography entries (requires natbib)
\setlength{\bibsep}{0pt}

% Configure captions
\captionsetup{labelfont=bf,font={small,color=lu_darkgray},skip=0pt}

% Define a fancy text box
\mdfdefinestyle{summarybox}{%
  leftline=true,
  rightline=false,
  topline=false,
  bottomline=false,
  linewidth=4pt,
  linecolor=lu_blue,
  frametitlefont=\bfseries\color{black}\small,
  frametitlebackgroundcolor=lu_lightblue,
  frametitleaboveskip=7pt,
  frametitlebelowskip=7pt,
  frametitlerule=true,
  frametitlerulewidth=1pt,
  backgroundcolor=lu_gray,
  innertopmargin=7pt,
  innerbottommargin=10pt,
  innerleftmargin=15pt,
  innerrightmargin=15pt,
  skipbelow=5pt,
  skipabove=0pt,
}
\newmdenv[style=summarybox]{summarybox}
\mdfdefinestyle{subsummarybox}{%
  leftline=true,
  rightline=false,
  topline=false,
  bottomline=false,
  linewidth=4pt,
  linecolor=lu_yellow,
  frametitlefont=\bfseries\color{black}\small,
  frametitlebackgroundcolor=lu_lightyellow,
  frametitleaboveskip=7pt,
  frametitlebelowskip=7pt,
  frametitlerule=true,
  frametitlerulewidth=1pt,
  backgroundcolor=lu_gray,
  innertopmargin=7pt,
  innerbottommargin=10pt,
  innerleftmargin=15pt,
  innerrightmargin=15pt,
  skipbelow=5pt,
  skipabove=0pt,
}
\newmdenv[style=subsummarybox]{subsummarybox}

% Define something like an fa-ul and a date list
\NewEnviron{fa-ul}{%
  \vspace{-0.4cm}
  \small
  \renewcommand{\arraystretch}{1.25}
  \begin{tabularx}{\linewidth}{@{}p{0.05\linewidth}@{}@{}p{0.95\linewidth}@{}}
    \BODY
  \end{tabularx}%
}
\NewEnviron{datelist}{%
  \vspace{-0.4cm}
  \small
  \renewcommand{\arraystretch}{1.25}
  \begin{tabularx}{\linewidth}{@{}p{0.15\linewidth}@{}@{}p{0.85\linewidth}@{}}
    \BODY
  \end{tabularx}%
}
\NewEnviron{paperlist}{%
  \vspace{-0.4cm}
  \small
  \renewcommand{\arraystretch}{1.25}
  \begin{tabularx}{\linewidth}{@{}p{0.08\linewidth}@{}@{}p{0.92\linewidth}@{}}
    \BODY
  \end{tabularx}%
}
\NewEnviron{courselist}{%
  \vspace{-0.4cm}
  \small
  \renewcommand{\arraystretch}{1.25}
  \begin{tabularx}{\linewidth}{@{}p{0.15\linewidth}@{}@{}p{0.85\linewidth}@{}}
    \BODY
  \end{tabularx}
}

% Define a fancy enumerate that has a title
\NewEnviron{fancyenum}[2]{%
  \vspace{0.25cm}
  \noindent#1\quad\textbf{#2}:
  \vspace{0.25cm}
  \begin{enumerate}
    \BODY
  \end{enumerate}
}

% Configure hyperref and add PDF metadata
\hypersetup{
    colorlinks,
    allcolors=lu_blue,
    pdftitle={\Title},
    pdfauthor={\Author},
    pdftex,
    breaklinks=true,
}

% make urls use the same font as every other text
\urlstyle{same}

% Prevent footnotes from being broken into multiple pages
\interfootnotelinepenalty=10000

% Configure headers and footers
\fancyhf{}
\lhead{\fontsize{9pt}{0}\selectfont\itshape \nouppercase\leftmark}
\chead{}
\rhead{\fontsize{9pt}{0}\selectfont \thepage}
\cfoot{}
\renewcommand{\headrulewidth}{0.3pt}
%%%%%%%%%%%%%%%%%%%%%%%%%%%%%%%%%%%%%%%%%%%%%%%%%%%%%%%%%%%%%%%%%%%%%%%%%%%%%%%

%%%%%%%%%%%%%%%%%%%%%%%%%%%%%%%%%%%%%%%%%%%%%%%%%%%%%%%%%%%%%%%%%%%%%%%%%%%%%%%
\begin{document}

\pagestyle{plain}
\frontmatter

\begin{titlepage}
  \begin{center}
    \includegraphics[height=2cm]{images/logo.pdf}
    \vspace{1cm}

    CONCURSO PÚBLICO

    PROFESSOR ADJUNTO A EM MLOPS

    INSTITUTO METROPOLE DIGITAL
    \vspace{5cm}

    \textbf{\LARGE MEMORIAL E PROJETO DE ATUAÇÂO PROFISSIONAL}
    \vspace{1cm}

    \textbf{\LARGE \MakeUppercase{\Author{}}}
    \vspace{5cm}

    {\small
      Apresentado para concurso público de títulos e provas para cargo de

      Professor Adjunto junto ao Instituto Metropole Digital

      Universidade Federal do Rio Grande do Norte.
      \vspace{1cm}

      Edital 069/2024-PROGESP
    }
    \vfill

    \Year{}
  \end{center}
\end{titlepage}

%==============================================================================
\chapter*{Resumo}

Possuo Bacharelado em Engenharia de Telecomunicações e Mestrado em Ciência da Computação pela \UFF{}.
Atualmente, estou cursando o Doutorado em Ciência da Computação na \UFRN{}.

Atuei como professor no programa PRONATEC e monitor da disciplina de Microprocessadores na \UFF{}. Desde a graduação,
venho me dedicando ao desenvolvimento e implementação de sistemas, com destaque para minha atuação no Laboratório de
Difração de Raios X da UFF (\textbf{LDRX}) durante uma bolsa do CNPq. Posteriormente, ampliei essa experiência em meu
estágio na GO2WEB e no desenvolvimento de sistemas de ensino a distância na UFF, além de meu trabalho atual na SEDIS-UFRN.
Essa trajetória também me proporcionou sólida expertise em extração e tratamento de dados.

Meu interesse em Inteligência Artificial começou na graduação, mas foi com o advento das LLMs que tive a oportunidade
de trabalhar mais intensamente com essa tecnologia, utilizando modelos abertos como o LLaMA~\cite{touvron2023llama}.
Minha atuação acadêmica inclui prêmios por trabalhos em Televisão Digital, com menções honrosas e publicações em eventos
relevantes.

Este memorial apresenta minha formação e trajetória profissional, incluindo reflexões sobre os fatores que me trouxeram
até aqui, as lições aprendidas ao longo do caminho e minha motivação para retornar à carreira acadêmica. Além disso, o
documento relata meus planos futuros para minha atuação no IMD e os projetos que pretendo desenvolver.



% usar linha do tempo?

%==============================================================================
\tableofcontents

\mainmatter
\pagestyle{fancy}

%==============================================================================
\chapter{Introdução}

\begin{summarybox}[frametitle=\faInfoCircle{}\quad Informações para contato]
  \begin{fa-ul}
    \faEnvelope & email: \href{mailto:\Email}{\Email} \\
    \aiLattes & Currículo Lattes: \url{https://lattes.cnpq.br/\Lattes} \\
    \faUser & Página pessoal: \url{https://esdrascaleb.github.io/} \\
    \aiOrcid & ORCID: \href{https://orcid.org/\ORCID}{\ORCID} \\
  \end{fa-ul}
\end{summarybox}

Ao longo da jornada acadêmica, acumulamos mais do que apenas certificados, diplomas e boas amizades.
Carregamos conosco o conhecimento e a experiência adquiridos, que permanecem como ferramentas valiosas, independentemente dos caminhos que seguimos.

Este documento tem como objetivo apresentar minhas atividades acadêmicas e profissionais,
com vistas à minha candidatura ao corpo docente do Instituto Metrópole Digital. As atividades estão organizadas em
tópicos que refletem as etapas da minha trajetória acadêmica e profissional, conectadas aos valores e motivações pessoais
que guiaram cada fase.

\section{Influências durante a infância e a adolescência}
Nasci e cresci em uma cidade de médio porte chamada Governador Valadares, em Minas Gerais. 
Meus pais se formaram em Engenharia Elétrica e desde criança tive contato com um computador 
inicialmente no trabalho do meu pai e em seguida em casa. Esse contexto sempre despertou em mim
uma paixão pela computação. Apesar de não usar computadores para programar sempre
tive curiosidade e configurava o computador para o tornar capaz de rodar os jogos que eu queria, mesmo quando o
hardware teoricamente não permitia.

Meu passatempo na escola no recreio costumava ser ler sobre as ultimas novidades
tecnologicas na biblioteca, a escola assinava as  revistas Galileu e Superinteressante que sempre traziam informações
interessantes e relevantes.

Quando tive acesso a internet me dediquei a aprender a utilizar os buscadores para encontrar 
a informação que eu gostava e me aprofundei em meus conhecimentos em implementação de sistemas
instalando e configurando emuladores para diversas plataformas em meu computador. Como não tinhamos
muito dinheiro eu podia jogar jogos antigos gratuitamente com o uso de emuladores.

Ao me mudar para o Rio de Janeiro tive aulas de programação na escola aprendendo Pascal e SQL. 
Até o vestibular tive muito interesse pela area de programação em especial para jogos chegando a 
fazer alguns projetos pequenos por conta própria e lendo sobre a área. Tentei vestibular para Engenharia da Computação
na UFRJ, mas passei mal em um dos dias de prova o que me custou uma nota ruim em química. Porem consegui ser aprovado
no vestibular da UFF para Engenharia de Telecomunicações.


\section{A estrutura deste memorial}

Este memorial está estruturado da seguinte forma. O capitulo~\ref{cap_uff} apresenta minha 
experiencia na graduação e mestrado na UFF. O capitulo~\ref{cap_atuacao} apresenta minhas atividades 
profissionais e académicas apos o mestrado. O capitulo~\ref{cap_pesquisa} apresenta minha pesquisa
atual e as razões de voltar a academia com o doutorado. O capitulo~\ref{cap_proje} apresenta
meus planos para o IMD e o IA360. E por fim o capitulo~\ref{cap_conclusao} apresenta minhas 
considerações finais.


%==============================================================================
\chapter{Formação Acadêmica na UFF}
\label{cap_uff}

\begin{summarybox}[frametitle=\faInfoCircle{}\quad Resumo da Formação acadêmica na UFF]
  \begin{datelist}
    \textbf{2005--2010} & \textbf{Bacharelado em Engenharia de Telecomunicações -- Universidade Federal Fluminense} \\
    2005 & Voluntariado Vestibular Solidário \\
    2005-2006 & Voluntariado Laboratório de Computação da Engenharia - LACE \\
    2006 & Voluntário Pet-Tele \\
    2006--2007 & Bolsita CNPQ no Laboratório de Difração de Raios X - LDRX \\
    2008-2009 & Estágio na GO2WEB \\
    2010 & Participação no Software Registrado IPÚBLICA - Sistema de Gestão Pública \\
    2010 & Monitor da Disciplina de Microprocessadores \\
    2010 & Menção Honrosa com o Jogo DamsTV no WTVDI/WebMedia 2010 \\
    2010 & ShortPaper DamasTV no SBGames2010 \\
    2010 & Voluntário Brasil Game Show 2010 \\
    \textbf{2011--2013} & \textbf{Mestrado em Ciencia da Computação -- Universidade Federal Fluminense} \\
    2011 & Menção Honrosa com o Jogo DamsTV no 1º IPTV Application Challenge \\
    2011 & Voluntário Brasil Game Show 2011 \\
    2012 & Participação no Projeto Eduroam com a RNP \\
    2012 & Participação como Monitor do curso do Eduroam no 18º Seminário RNP de Capacitação e Inovação \\
    2012 & Participação como Monitor do curso do Eduroam no IdP Forum RNP \\
    2013 & Publicação do paper "JNS: An alternative authoring language for specifying NCL multimedia documents" no ICMEW 2013 \\
    2013 & Publicação do paper "NCL4WEB - Translating NCL Applications to HTML5 Web Pages no DocEng" 2013 \\
    2013 & Publicação no Software Registrado JNS Translator \\
    2013 & Participação no Software Registrado NCL4WEB \\
    2013 & Participação no Software Registrado JNS \\
  \end{datelist}
\end{summarybox}

Neste capitulo irei detalhar minha experiencia acadêmica durante a graduacao e mestrado. Minha 
atuação em pesquisa e extenção. Assim como os principais prémios e publicações.

\section{Graduação em Engenharia de Telecomunicação}
\label{sec_grad}


\section{Mestrado em Ciência da Computação}
\label{sec_mes}


%==============================================================================
\chapter{Atuação Profissional apos Mestrado}
\label{cap_atuacao}

\begin{summarybox}[frametitle=\faInfoCircle{}\quad Resumo da Atuação Profissional]
  \begin{datelist}
    2013 & Professor do Curso de Cabeamento Estruturado no IFRN Parnamirim pelo PRONATEC \\
    2014-2015 & Atuação como Desenvolvedor Moodle na SEDIS pela FUNPEC \\
    2015 & Participação da organização do capitulo local da Global Game Jam \\
    2015-2020 & Participação no Projeto AVASUS \\
    2023 & Participação do projeto Piloto a Capes para Monitoramento do Ensino a Distáncia \\
  \end{datelist}
\end{summarybox}





%==============================================================================
\chapter{Pesquisa Atual no Doutorado}
\label{cap_pesquisa}


\section{Atividades de Extensão}




\chapter{Projeto de Atuação Profissional}
\label{cap_proje}

Além de ministrar aulas, pretendo desenvolver projetos com IA para auxiliar startups e empresas júnior da cidade na
implementação de modelos generativos em seus serviços e produtos. O objetivo é ensinar estratégias para o uso de IAs
quantizadas e de baixo custo operacional, permitindo a implementação diretamente no cliente e, assim, reduzindo os
custos para as startups em especial as empresas do Parque Tecnológico do IMD.

Também planejo iniciar projetos pilotos para o uso de IA em diversas áreas, tais como:
Inclusão Digital para Idosos: Ajudar idosos a navegar no ambiente digital com segurança atravez de cursos voltados ao uso de assistentes virtuais. Reduzindo sua vulnerabilidade a informações falsas e promovendo sua inclusão digital.
Tutoria Virtual para Estudantes: Implementar pipilenes de modelos para serem tutores virtuais para dar suporte personalizado aos alunos em disciplinas específicas, ajudando a melhorar o desempenho acadêmico e a reduzir a carga de trabalho dos professores.
Secretários Virtuais Acadêmicos: Desenvolver secretários virtuais para simplificar processos acadêmicos, facilitando a vida dos alunos e aumentando a eficiência no trabalho dos servidores.
Gestão Inteligente de Energia e Resíduos: Implementar modelos inteligentes para otimizar o uso de energia no campus, alertando para luzes ou ar-condicionado deixados ligados em salas, e para uma coleta de lixo eficiente, com notificações automáticas para agilizar o recolhimento.
Apoio a acessibiliade: Auxiliar  o desenvolvimento soluções que auxiliem a acessibilidade como cães guias virtuais para pessoas com deficiencia visual, leitores de libras mais avançados para deficientes auditivos e assistentes para pessoas com dificuldade de socialização.


%==============================================================================
\chapter{Conclusão}
\label{cap_conclusao}

Minha trajetória acadêmica e profissional é marcada pela busca constante por aprendizado, inovação e contribuição à
sociedade por meio da tecnologia. Cada etapa, desde meus primeiros projetos na graduação até os desafios atuais no
doutorado, moldou minhas competências e consolidou minha paixão pela docência e pela pesquisa aplicada.

Minha motivação para atuar no IMD está diretamente ligada ao potencial de impactar positivamente a formação de
profissionais e à oportunidade de desenvolver soluções práticas para desafios reais. Acredito que minha experiência
em áreas como sistemas interativos, ensino a distância e inteligência artificial, combinada com meu comprometimento
com a inclusão digital e acessibilidade, me permitirá agregar valor ao Instituto e contribuir para o sucesso de
iniciativas como o IA360.

Estou determinado a continuar expandindo o conhecimento e promovendo inovações que transcendam a academia, criando um
impacto duradouro na indústria, na sociedade e no ambiente educacional. Este memorial reflete não apenas o percurso
que me trouxe até aqui, mas também minha visão de futuro e o desejo de construir um legado de transformação e excelência.



%==============================================================================
\backmatter
\bibliographystyle{apalike-doi}
\bibliography{references}

\end{document}

%%%%%%%%%%%%%%%%%%%%%%%%%%%%%%%%%%%%%%%%%%%%%%%%%%%%%%%%%%%%%%%%%%%%%%%%%%%%%%%
% Memorial para concurso público de Professor Doutor na USP.
%
% Formatação inspirada em:
% * https://tug.org/pracjourn/2008-1/mori/mori.pdf
% * https://github.com/santisoler/phd-thesis
% * https://github.com/compgeolab/dissertation-template
%%%%%%%%%%%%%%%%%%%%%%%%%%%%%%%%%%%%%%%%%%%%%%%%%%%%%%%%%%%%%%%%%%%%%%%%%%%%%%%

%%%%%%%%%%%%%%%%%%%%%%%%%%%%%%%%%%%%%%%%%%%%%%%%%%%%%%%%%%%%%%%%%%%%%%%%%%%%%%%
% Set a class and import packages
\documentclass[10pt,a4paper,oneside]{book}

% Variables
\newcommand{\Year}{2024}
\newcommand{\Author}{Esdras Caleb Oliveira Silva}
\newcommand{\Title}{Memorial e Projeto de Atuação Profissional de \Author{} para concurso público - Professor do Magistério Superior Adjunto-A - IMD/UFRN}
\newcommand{\Email}{esdrascaleb@gmail.com}
\newcommand{\ORCID}{0000-0001-5232-5067}
\newcommand{\GoogleScholar}{zWxct1EAAAAJ}
\newcommand{\Lattes}{5923627359710427}

% Variables for easier typing of some names
\newcommand{\UFF}{Universidade Federal Fluminense}
\newcommand{\UFRN}{Universidade Federal do Rio Grande do Norte}

% Names for citing coauthors
\newcommand{\Me}{\textbf{Uieda, L}}
\newcommand{\Val}{Barbosa, VCF}
\newcommand{\Bi}{Oliveira Jr, VC}
\newcommand{\Paul}{Wessel, P}
\newcommand{\Joaquim}{Luis, J}
\newcommand{\Remko}{Scharroo, R}
\newcommand{\Florian}{Wobbe, F}
\newcommand{\Walter}{Smith, WHF}
\newcommand{\Dongdong}{Tian, D}
\newcommand{\Bridget}{Smith-Konter, B}
\newcommand{\Eric}{Xu, X}
\newcommand{\David}{Sandwell, DT}
\newcommand{\Carla}{Braitenberg, C}
\newcommand{\Naomi}{Ussami, N}
\newcommand{\Manoel}{D'Agrella-Filho, MS}
\newcommand{\JB}{Silva, JBC}
\newcommand{\Dai}{Sales, DP}
\newcommand{\Figura}{Melo, FF}
\newcommand{\Dio}{Carlos, DU}
\newcommand{\BragaVale}{Braga, MA}
\newcommand{\YLi}{Li, Y}
\newcommand{\Angeli}{Angeli, G}
\newcommand{\Peres}{Peres, G}
\newcommand{\Everton}{Bomfim, EP}
\newcommand{\Eder}{Molina, E}
\newcommand{\Gomes}{Gomes, AAS}
\newcommand{\Santiago}{Soler, SR}
\newcommand{\Agustina}{Pesce, A}
\newcommand{\Gimenez}{Gimenez, ME}
\newcommand{\Kristoffer}{Hallam, KAT}
\newcommand{\Guangdong}{Zhao, G}
\newcommand{\Bo}{Chen, B}
\newcommand{\JLiu}{Liu, J}
\newcommand{\LChen}{Chen, L}
\newcommand{\RGuo}{Guo, R}
\newcommand{\MKaban}{Kaban, MK}
\newcommand{\Lindsey}{Heagy, LJ}
\newcommand{\Lion}{Krischer, L}
\newcommand{\Rene}{Gassmoeller, R}
\newcommand{\Bane}{Sullivan, CB}
\newcommand{\Jens}{Klump, JF}
\newcommand{\LBarba}{Barba, LA}
\newcommand{\JBazan}{Bazan, J}
\newcommand{\JBrown}{Brown, J}
\newcommand{\RGuimera}{Guimera, RV}
\newcommand{\MGymrek}{Gymrek, M}
\newcommand{\AHanna}{Alex Hanna}
\newcommand{\KHuff}{Huff, KD}
\newcommand{\DKatz}{Katz, DS}
\newcommand{\CMadan}{Madan, CR}
\newcommand{\KMoerman}{Moerman, KM}
\newcommand{\KNiemeyer}{Niemeyer, KE}
\newcommand{\JPoulson}{Poulson, JL}
\newcommand{\PPrins}{Prins, P}
\newcommand{\KRam}{Ram, K}
\newcommand{\ARokem}{Rokem, A}
\newcommand{\Arfon}{Smith, AM}
\newcommand{\GThiruvathukal}{Thiruvathukal, GK}
\newcommand{\KThyng}{Thyng, KM}
\newcommand{\BWilson}{Wilson, BE}
\newcommand{\Yehudi}{Yehudi, Y}
\newcommand{\Remi}{Rampin, R}
\newcommand{\Hugo}{van Kemenade, H}
\newcommand{\MattTurk}{Turk, M}
\newcommand{\Shapero}{Shapero, D}
\newcommand{\Anderson}{Banihirwe, A}
\newcommand{\Leeman}{Leeman, J}
\newcommand{\JEbbing}{Ebbing, J}
\newcommand{\AGuy}{Guy, A}
\newcommand{\JFarquharson}{Farquharson, J}
\newcommand{\AKushnir}{Kushnir, A}
\newcommand{\FWadsworth}{Wadsworth, F}
\newcommand{\LPerozzi}{Perozzi, L}
\newcommand{\MWieczorek}{Wieczorek, MA}
\newcommand{\LLi}{Li, L}
\newcommand{\Ricardo}{Trindade, RIF}

% Links to webpages I use often
\newcommand{\SantiagoLink}{\href{https://www.santisoler.com/}{Santiago R. Soler}}
\newcommand{\VanderleiLink}{\href{https://www.pinga-lab.org/people/oliveira-jr.html}{Vanderlei C. Oliveira Jr.}}
\newcommand{\SandwellLink}{\href{https://topex.ucsd.edu/sandwell/}{David Sandwell}}
\newcommand{\ValeriaLink}{\href{https://www.pinga-lab.org/people/barbosa.html}{Valéria C. F. Barbosa}}
\newcommand{\PaulLink}{\href{https://www.soest.hawaii.edu/pwessel/}{Paul Wessel}}
\newcommand{\IndiaLink}{\href{https://www.compgeolab.org/team/\#indiauppal}{India Uppal}}
\newcommand{\GelsonLink}{\href{https://www.compgeolab.org/team/\#Souza-junior}{Gelson Ferreira de Souza Junior}}

\newcommand{\GMTLink}{\href{https://www.generic-mapping-tools.org}{Generic Mapping Tools}}
\newcommand{\CompGeoLabLink}{\href{https://www.compgeolab.org/}{Computer-Oriented Geoscience Lab}}
\newcommand{\SwunngLink}{\href{https://softwareunderground.org/}{Software Underground}}
\newcommand{\FatiandoLink}{\href{https://www.fatiando.org}{Fatiando a Terra}}
\newcommand{\PyGMTLink}{\href{https://www.pygmt.org}{PyGMT}}
\newcommand{\SSILink}{\href{https://software.ac.uk/}{Software Sustainability Institute}}

% Import packages
\usepackage[utf8]{inputenc}
\usepackage[T1]{fontenc}
\usepackage[brazil]{babel}
\usepackage{geometry}
\usepackage{graphicx}
\usepackage{amssymb}
\usepackage{amsmath}
\usepackage{mathpazo}
\usepackage{hyperref}
% create fancy headers
\usepackage{fancyhdr}
% commands for managing dates and its formats
\usepackage{datetime2}
% improved urls with proper hyphenation
\usepackage{xurl}
% Control over enumerate and itemize
\usepackage{enumitem}
% Tweak the look of captions
\usepackage{caption}
% To control the style of section titles
\usepackage{titlesec}
% Add the bibliography to the table of contents
\usepackage[nottoc,chapter]{tocbibind}
\usepackage[round,authoryear,sort]{natbib}
% show dois as links on references
\usepackage{doi}
% Icon and fonts (requires using xelatex or luatex)
\usepackage{fontawesome5}
\usepackage{academicons}
\usepackage{fontspec}
\usepackage[mono]{notomath}
% To make everything neater
\usepackage{microtype}
% To make fancy text boxes
\usepackage{xcolor}
\usepackage[framemethod=default]{mdframed}
% For fancy and multipage tables
\usepackage{tabularx}
\usepackage{ltablex}
% To define custom environments
\usepackage{environ}
\usepackage{setspace}
% Reference sections by name
\usepackage{nameref}
% Better handling of footnotes inside summary boxes
\usepackage{footmisc}
%%%%%%%%%%%%%%%%%%%%%%%%%%%%%%%%%%%%%%%%%%%%%%%%%%%%%%%%%%%%%%%%%%%%%%%%%%%%%%%

%%%%%%%%%%%%%%%%%%%%%%%%%%%%%%%%%%%%%%%%%%%%%%%%%%%%%%%%%%%%%%%%%%%%%%%%%%%%%%%
% Configuration of the document

\geometry{%
  left=30mm,
  right=30mm,
  top=20mm,
  bottom=15mm,
  headsep=5mm,
  headheight=5mm,
  footskip=10mm,
  includehead=true,
  includefoot=true
}

% Increase the line spacing
\SetSinglespace{1.2}
\onehalfspacing

% Remove spacing between enumerate/itemize items
\setlist{nosep}

% Padding between the first figure and the chapter title
\newcommand{\HeroFigPad}{\vspace{-1cm}}

% Padding before the software logo figures
\newcommand{\SoftwareFigPad}{\vspace{-0.3cm}}

% Add a link to a DOI
\newcommand{\DOI}[1]{\url{https://doi.org/#1}}

% Add a link to a GitHub repository
\newcommand{\GitHub}[1]{\faGithub{} Código: \url{https://github.com/#1}}

% Add a link to a YouTube video
\newcommand{\YouTube}[1]{\faYoutube{} Vídeo: \url{https://youtu.be/#1}}

% Add a link to a supplementary data
\newcommand{\Data}[1]{\faChartBar{} Dados: \url{https://doi.org/#1}}

% Add a link to a preprint
\newcommand{\Preprint}[1]{\faLockOpen{} Preprint: \url{https://doi.org/#1}}

% Make a Unicode bullet symbol
\newcommand{\Bullet}{•\enspace}

% Define custom colors
\definecolor{lu_gray}{gray}{0.98}
\definecolor{lu_darkgray}{gray}{0.3}
\definecolor{lu_blue}{RGB}{32, 96, 194}
\definecolor{lu_lightblue}{RGB}{238, 245, 250}
\definecolor{lu_yellow}{RGB}{255, 193, 7}
\definecolor{lu_lightyellow}{RGB}{255, 249, 230}

% Customize how Chapter headings are displayed
\titleformat{\chapter}[display]{\normalfont}{\large Capítulo \thechapter}{0pt}{\huge}[\titlerule]
\titlespacing*{\chapter}{0pt}{-40pt}{40pt}

% Set the spacing between bibliography entries (requires natbib)
\setlength{\bibsep}{0pt}

% Configure captions
\captionsetup{labelfont=bf,font={small,color=lu_darkgray},skip=0pt}

% Define a fancy text box
\mdfdefinestyle{summarybox}{%
  leftline=true,
  rightline=false,
  topline=false,
  bottomline=false,
  linewidth=4pt,
  linecolor=lu_blue,
  frametitlefont=\bfseries\color{black}\small,
  frametitlebackgroundcolor=lu_lightblue,
  frametitleaboveskip=7pt,
  frametitlebelowskip=7pt,
  frametitlerule=true,
  frametitlerulewidth=1pt,
  backgroundcolor=lu_gray,
  innertopmargin=7pt,
  innerbottommargin=10pt,
  innerleftmargin=15pt,
  innerrightmargin=15pt,
  skipbelow=5pt,
  skipabove=0pt,
}
\newmdenv[style=summarybox]{summarybox}
\mdfdefinestyle{subsummarybox}{%
  leftline=true,
  rightline=false,
  topline=false,
  bottomline=false,
  linewidth=4pt,
  linecolor=lu_yellow,
  frametitlefont=\bfseries\color{black}\small,
  frametitlebackgroundcolor=lu_lightyellow,
  frametitleaboveskip=7pt,
  frametitlebelowskip=7pt,
  frametitlerule=true,
  frametitlerulewidth=1pt,
  backgroundcolor=lu_gray,
  innertopmargin=7pt,
  innerbottommargin=10pt,
  innerleftmargin=15pt,
  innerrightmargin=15pt,
  skipbelow=5pt,
  skipabove=0pt,
}
\newmdenv[style=subsummarybox]{subsummarybox}

% Define something like an fa-ul and a date list
\NewEnviron{fa-ul}{%
  \vspace{-0.4cm}
  \small
  \renewcommand{\arraystretch}{1.25}
  \begin{tabularx}{\linewidth}{@{}p{0.05\linewidth}@{}@{}p{0.95\linewidth}@{}}
    \BODY
  \end{tabularx}%
}
\NewEnviron{datelist}{%
  \vspace{-0.4cm}
  \small
  \renewcommand{\arraystretch}{1.25}
  \begin{tabularx}{\linewidth}{@{}p{0.15\linewidth}@{}@{}p{0.85\linewidth}@{}}
    \BODY
  \end{tabularx}%
}
\NewEnviron{paperlist}{%
  \vspace{-0.4cm}
  \small
  \renewcommand{\arraystretch}{1.25}
  \begin{tabularx}{\linewidth}{@{}p{0.08\linewidth}@{}@{}p{0.92\linewidth}@{}}
    \BODY
  \end{tabularx}%
}
\NewEnviron{courselist}{%
  \vspace{-0.4cm}
  \small
  \renewcommand{\arraystretch}{1.25}
  \begin{tabularx}{\linewidth}{@{}p{0.15\linewidth}@{}@{}p{0.85\linewidth}@{}}
    \BODY
  \end{tabularx}
}

% Define a fancy enumerate that has a title
\NewEnviron{fancyenum}[2]{%
  \vspace{0.25cm}
  \noindent#1\quad\textbf{#2}:
  \vspace{0.25cm}
  \begin{enumerate}
    \BODY
  \end{enumerate}
}

% Configure hyperref and add PDF metadata
\hypersetup{
    colorlinks,
    allcolors=lu_blue,
    pdftitle={\Title},
    pdfauthor={\Author},
    pdftex,
    breaklinks=true,
}

% make urls use the same font as every other text
\urlstyle{same}

% Prevent footnotes from being broken into multiple pages
\interfootnotelinepenalty=10000

% Configure headers and footers
\fancyhf{}
\lhead{\fontsize{9pt}{0}\selectfont\itshape \nouppercase\leftmark}
\chead{}
\rhead{\fontsize{9pt}{0}\selectfont \thepage}
\cfoot{}
\renewcommand{\headrulewidth}{0.3pt}
%%%%%%%%%%%%%%%%%%%%%%%%%%%%%%%%%%%%%%%%%%%%%%%%%%%%%%%%%%%%%%%%%%%%%%%%%%%%%%%

%%%%%%%%%%%%%%%%%%%%%%%%%%%%%%%%%%%%%%%%%%%%%%%%%%%%%%%%%%%%%%%%%%%%%%%%%%%%%%%
\begin{document}

\pagestyle{plain}
\frontmatter

\begin{titlepage}
  \begin{center}
    \includegraphics[height=2cm]{images/logo.pdf}
    \vspace{1cm}

    CONCURSO PÚBLICO

    PROFESSOR ADJUNTO A EM MLOPS

    INSTITUTO METROPOLE DIGITAL
    \vspace{5cm}

    \textbf{\LARGE MEMORIAL E PROJETO DE ATUAÇÂO PROFISSIONAL}
    \vspace{1cm}

    \textbf{\LARGE \MakeUppercase{\Author{}}}
    \vspace{5cm}

    {\small
      Apresentado para concurso público de títulos e provas para cargo de

      Professor Adjunto junto ao Instituto Metropole Digital

      Universidade Federal do Rio Grande do Norte.
      \vspace{1cm}

      Edital 069/2024-PROGESP
    }
    \vfill

    \Year{}
  \end{center}
\end{titlepage}

%==============================================================================
\chapter*{Resumo}

Possuo Bacharelado em Engenharia de Telecomunicações e Mestrado em Ciência da Computação pela \UFF{}.
Atualmente, estou cursando o Doutorado em Ciência da Computação na \UFRN{}.

Desde a graduação, venho me dedicando ao desenvolvimento e à implementação de sistemas, com destaque para minha atuação
no Laboratório de Difração de Raios X da UFF (\textbf{LDRX}), criando sistemas para o auxílio no tratamento de dados.
Posteriormente, ampliei essa experiência em meu estágio na GO2WEB, onde tratei dados mal formatados de um censo escolar
municipal. Na \UFF{}, tive o privilégio de também fazer parte da implementação do Eduroam em nível nacional, bem como de
participar do desenvolvimento de soluções para popularizar o uso do sistema brasileiro de televisão digital.

Meu interesse em Inteligência Artificial começou na graduação, mas foi com o advento das LLMs que tive a oportunidade de
trabalhar mais intensamente com essa tecnologia, desenvolvendo soluções que fazem uso dela~\cite{sistematicreviem}.
Minha atuação acadêmica inclui prêmios por trabalhos em Televisão Digital, com menções honrosas e publicações em eventos
relevantes. Minha experiência em docência inclui a monitoria da disciplina de Microprocessadores na \UFF{} e a
ministração de disciplinas do curso PRONATEC no IFRN.

Este memorial apresenta minha formação e trajetória profissional, incluindo reflexões sobre os fatores que me trouxeram
até aqui, as lições aprendidas ao longo do caminho e minha motivação para retornar à carreira acadêmica.
Além disso, o documento relata meus planos para minha atuação no IMD e os projetos que pretendo desenvolver.


%==============================================================================
\tableofcontents

\mainmatter
\pagestyle{fancy}

%==============================================================================
\chapter{Introdução}

\begin{summarybox}[frametitle=\faInfoCircle{}\quad Informações para contato]
  \begin{fa-ul}
    \faEnvelope & email: \href{mailto:\Email}{\Email} \\
    \aiLattes & Currículo Lattes: \url{https://lattes.cnpq.br/\Lattes} \\
    \faUser & Página pessoal: \url{https://esdrascaleb.github.io/} \\
    \aiOrcid & ORCID: \href{https://orcid.org/\ORCID}{\ORCID} \\
  \end{fa-ul}
\end{summarybox}

\section{Influências durante a infância e a adolescência}
Nasci e cresci em Governador Valadares, uma cidade de médio porte localizada em Minas Gerais. Meus pais são formados em Engenharia Elétrica e, desde a infância, tive contato com computadores, inicialmente no trabalho de meu pai e, posteriormente, em casa. Esse ambiente despertou em mim uma paixão pela computação. Embora não tenha começado a programar tão cedo, sempre fui curioso e gostava de configurar o computador para rodar os jogos que queria, mesmo quando o hardware não era compatível.

Durante a escola, meu passatempo no recreio era ler sobre as últimas novidades tecnológicas na biblioteca. A escola assinava as revistas \textit{Galileu} e \textit{Superinteressante}, que traziam informações fascinantes e relevantes sobre ciência e tecnologia.

Quando tive acesso à internet, me dediquei a aprender a utilizar os buscadores para encontrar informações que me interessavam. Aprofundei meus conhecimentos em implementação de sistemas ao instalar e configurar emuladores para diferentes plataformas em meu computador. Como não tínhamos muitos recursos financeiros, os emuladores me permitiram jogar jogos antigos gratuitamente.

Após me mudar para o Rio de Janeiro, tive aulas de programação na escola, aprendendo Pascal e SQL. O interesse por programação, especialmente na área de jogos, seguiu me acompanhando até o vestibular, onde fiz alguns projetos pequenos por conta própria e continuei estudando sobre o assunto. Tentei o vestibular para Engenharia da Computação na UFRJ, mas, infelizmente, passei mal em um dos dias de prova, o que prejudicou minha nota em química. No entanto, fui aprovado no vestibular da UFF para o curso de Engenharia de Telecomunicações. Essa escolha foi inspirada em meu pai, que é Engenheiro de Telecomunicações na Oi, desde os tempos em que a empresa ainda era a Telebras.

\section{A estrutura deste memorial}

Este memorial está estruturado da seguinte forma: o Capítulo~\ref{cap_uff} aborda minha experiência durante a graduação
e o mestrado na UFF. O Capítulo~\ref{cap_atuacao} descreve minhas atividades profissionais e acadêmicas realizadas após
o mestrado. No Capítulo~\ref{cap_pesquisa}, apresento minha pesquisa atual e os motivos que me levaram a retornar à
academia para cursar o doutorado. O Capítulo~\ref{cap_proje} detalha meus planos para atuação no IMD e no projeto IA360.
Por fim, o Capítulo~\ref{cap_conclusao} traz minhas considerações finais.


%==============================================================================
\chapter{Formação Acadêmica na UFF}
\label{cap_uff}

\begin{summarybox}[frametitle=\faInfoCircle{}\quad Resumo da Formação acadêmica na UFF]
  \begin{datelist}
    \textbf{2005--2010} & \textbf{Bacharelado em Engenharia de Telecomunicações -- Universidade Federal Fluminense} \\
    2005-2006 & Voluntariado Laboratório de Computação da Engenharia - LACE \\
    2006 & Voluntário Pet-Tele \\
    2006 & Voluntariado Vestibular Solidário \\
    2006-2007 & Bolsita CNPQ no Laboratório de Difração de Raios X - LDRX \\
    2007-2008 & Bolsista Laboratório de AutoCad \\
    2008-2009 & Estágio na GO2WEB \\
    2010 & Autoria no Software Registrado IPÚBLICA - SISTEMA INTELIGENTE DE GESTÃO PÚBLICA  \\
    2010 & Monitor da Disciplina de Microprocessadores \\
    2010 & Menção Honrosa com o Jogo DamasTV no WTVDI/WebMedia 2010 \\
    2010 & ShortPaper DamasTV no SBGames2010 \\
    2010 & Voluntário Brasil Game Show 2010 \\
    \textbf{2011--2013} & \textbf{Mestrado em Ciencia da Computação -- Universidade Federal Fluminense} \\
    2011 & Menção Honrosa com o Jogo DamasTV no 1º IPTV Application Challenge \\
    2011 & Voluntário Brasil Game Show 2011 \\
    2012 & Participação no Projeto Eduroam com a RNP \\
    2012 & Participação como Monitor do curso do Eduroam no 18º Seminário RNP de Capacitação e Inovação \\
    2012 & Participação como Monitor do curso do Eduroam no IdP Forum RNP \\
    2013 & Publicação do paper "JNS: An alternative authoring language for specifying NCL multimedia documents" no ICMEW 2013 \\
    2013 & Publicação do paper "NCL4WEB - Translating NCL Applications to HTML5 Web Pages no DocEng" 2013 \\
    2013 & Autoria no Software Registrado JNS Translator \\
    2013 & Autoria no Software Registrado NCL4WEB \\
    2013 & Autoria no Software Registrado JNS \\
  \end{datelist}
\end{summarybox}

Este capítulo detalha minha experiência acadêmica durante a graduação e o mestrado, incluindo minha atuação em pesquisa
e extensão, além dos principais prêmios e publicações.

\section{Graduação em Engenharia de Telecomunicação}
\label{sec_grad}
\begin{subsummarybox}[frametitle=\faGraduationCap{}\quad Bacharelado em Engenharia de Telecomunicações]
  \begin{fa-ul}
    \faUniversity & Universidade Federal Fluminense \\
    \faCalendar & Fevereiro 2005 -- Dezembro 2010 \\
    \faUser & Orientadora: Debora Muchaluat-Saade\\
    \faInfoCircle & DamasTV: um Jogo de Damas para o SBTVD~\cite{saad2010damastv}
  \end{fa-ul}
\end{subsummarybox}
Desde o primeiro ano da graduação, procurei formas de acesso a uma bolsa de pesquisa. No segundo período, tornei-me
voluntário no laboratório de informática, com o objetivo de me tornar bolsista efetivo no período seguinte, quando a
seleção fosse reaberta.

No período seguinte, participei da seleção para o PET-Tele, mas fui rejeitado. No entanto, ingressei no grupo como
voluntário, onde participei de cursos e ajudei na organização do Encontro Regional de PET na UFF, em 2006. No PET-Tele,
também tive a oportunidade de participar da Semana de Engenharia da Computação e Eletrônica da UFRJ. Nesse evento,
aprendi um pouco sobre Inteligência Artificial e fiquei instigado a tentar a docência, o que me motivou a lecionar por
alguns meses no pré-vestibular comunitário da UFF.

No mesmo período, ingressei no Laboratório de Difração de Raios-X, pertencente ao Instituto de Física da UFF, como
bolsista do CNPq. Minha experiência com a iniciação científica no Instituto de Física foi muito instrutiva. Sob a
orientação do professor Carlos Basílio (hoje na UFMG), aprendi sobre o tratamento de dados e desenvolvi uma aplicação
para auxiliar na seleção visual de outliers. Também tive contato com software livre e sistemas de implementação,
auxiliando na configuração da rede interna do laboratório e na instalação de softwares para acesso remoto.

Durante quase todos os anos da graduação, participei das semanas acadêmicas promovidas na Engenharia, frequentando
especialmente para conhecer novidades e pesquisas. Era como na escola, mas agora, em vez de ler revistas, eu podia
conferir as inovações pessoalmente.

Após concluir minha pesquisa no LDRX, no quinto período, tornei-me monitor do laboratório de AutoCAD, onde aprendi mais
sobre o sistema Linux. Nesse contexto acadêmico, também fui integrante da liderança do movimento estudantil Alfa e Ômega,
no qual já participava desde o terceiro período. Durante essa experiência, desenvolvi habilidades de articulação para
falar em público, gestão de pessoas e liderança. O grupo oferecia cursos e treinamentos práticos, que enriqueceram
minhas competências. Além disso, participei como voluntário no Pan-Americano, organizado pela More Than Gold, onde
tive a oportunidade de treinar meu inglês.

No sétimo período, comecei a procurar estágios e, em 2008, fui aprovado em um estágio de um ano e meio na Go2Web, uma
empresa de desenvolvimento de sistemas web. Nessa empresa, aprendi programação orientada a objetos com C\# e o framework
.NET, além de programação web com ASP, HTML, JavaScript e CSS. Participei do desenvolvimento do software de gestão
governamental do município do Rio de Janeiro, iPública, que utilizava a biblioteca Google Maps para criar um painel
de gestão. Também ajudei em um censo escolar da rede pública, projetando o sistema usado para enviar os dados ao banco
de dados e auxiliando na recuperação de mais de 100 mil registros de dados mal formatados.

Em 2009, cursei uma disciplina optativa com minha futura orientadora, a professora Débora Christina Muchaluat Saad,
sobre multimídia e televisão digital. Meu trabalho de conclusão da disciplina foi a primeira versão do DamasTV, um jogo
para o middleware interativo Ginga-NCL~\cite{soares2007ginga} do sistema brasileiro de televisão
digital~\cite{mendes2007sbtvd}. Desenvolvido em Lua, o jogo evoluiu para permitir partidas pela internet, utilizando o
canal de retorno da televisão. Por meio desse trabalho, ingressei no grupo de pesquisa Midiacom e enviei meu primeiro
artigo acadêmico, que foi rejeitado no SBGames 2009.

Em 2010, fui monitor da disciplina de Microprocessadores, minha segunda experiência relacionada à docência. Com mais
habilidades para falar em público, desenvolvidas no movimento estudantil, tive uma experiência enriquecedora ao
ministrar aulas de laboratório e reposição para a turma. Essa experiência fortaleceu minha motivação para a área de
ensino, ao perceber o impacto positivo do meu trabalho no aprendizado dos alunos.

Ainda em 2010, meu artigo~\cite{saad2010damastv} sobre o DamasTV foi aprovado como short paper no SBGames 2010.
Participar desse evento foi muito gratificante. Aprendi sobre jogos educativos que realmente priorizam a experiência
do jogador, além de conhecer empresas do mercado brasileiro de games, como MiniBoss e Vortex.

Nesse mesmo ano, o DamasTV recebeu sua primeira premiação internacional: uma menção honrosa no 1º Concurso
Latino-Americano de Aplicações para Televisão Digital, realizado no Workshop de Televisão Digital no WebMedia 2010.

No segundo semestre de 2010, cursei a disciplina de Programação Gráfica e Jogos Digitais com o professor Esteban Clua,
que foi uma das mais divertidas da graduação. No trabalho final, desenvolvi um protótipo de engine de jogos para
televisão digital, inspirado no trabalho GingaGame de Diego Bastos~\cite{barboza2009ginga}, adaptando-o para o NCLua.

A convivência com o professor Esteban levou-me a participar da Brasil Game Show de 2010 como organizador e colaborador.
Auxiliei na organização e execução da Brasil Game Jam, das palestras e do concurso de jogos.

Ao final de 2010, defendi meu trabalho de conclusão de curso:
\textit{DamasTV: um Jogo Interativo para Televisão Digital}. A banca contou com a presença das professoras Maria Luíza
(coordenadora do meu curso), Débora Saad (minha orientadora) e Esteban Clua. Fui aprovado com nota máxima e decidi
seguir para o mestrado, orientado pela professora Débora, que havia migrado do Departamento de Engenharia de
Telecomunicações para o Departamento de Ciência da Computação, permitindo-me atuar na área que eu desejava desde o
vestibular.




\section{Mestrado em Ciência da Computação}
\label{sec_mes}
\begin{subsummarybox}[frametitle=\faGraduationCap{}\quad Mestrado em Ciência da Computação]
  \begin{fa-ul}
    \faUniversity & Universidade Federal Fluminense \\
    \faCalendar & Fevereiro 2011 -- Junho 2013 \\
    \faUser & Orientadora: Debora Muchaluat-Saade\\
    \faInfoCircle & JNS E NCL4WEB: AUXILIANDO O DESENVOLVIMENTO E DIVULGAÇÃO DE DOCUMENTOS NCL~\cite{silva2013jns}
  \end{fa-ul}
\end{subsummarybox}

Iniciei o mestrado com a mesma orientadora e decidi por um projeto inteiramente novo: uma forma de programar para
televisão digital mais acessível e menos verbosa, na forma de uma linguagem de script chamada
JNS~\cite{silva2013jnsieee}. A ideia surgiu ao constatar que o mercado estava migrando do XML para o JSON como forma
de representação.

No primeiro ano, consegui terminar a especificação da linguagem e o compilador básico para NCL. Nesse mesmo ano,
recebi outro prêmio pelo DamasTV e fui voluntário em mais uma Brasil Game Show.

Apesar de ter algumas dificuldades por estar fazendo o doutorado em uma área diferente da graduação, consegui concluir
as disciplinas. No entanto, isso me fez gastar mais tempo estudando e menos tempo escrevendo.

A implementação sempre foi um ponto forte para mim. No segundo ano do mestrado, entrei para o projeto piloto do Eduroam
na UFF. Testamos a implementação, avaliamos a viabilidade e criamos tutoriais para a RNP utilizar no projeto, que hoje
é usado em todas as instituições federais. Participei de dois cursos sobre o Eduroam em 2012.

No segundo ano do meu mestrado, em 2012, desenvolvi a segunda parte do meu projeto, o NCL4WEB, um
XSLT~\cite{clark1999xsl} para traduzir o NCL para HTML, o que permitiria que qualquer navegador executasse uma
aplicação de televisão digital.

Acabei tendo que pedir uma extensão do mestrado para conseguir escrever os artigos. Sempre tive muita aptidão para
implementar sistemas, mas não tanto para escrever sobre eles. Além disso, eu precisava testar a usabilidade da
linguagem de script JNS.

Com a extensão, tive tempo para escrever e publicar um artigo sobre o JNS para o Workshop de Gerenciamento de
Informação em Multimídia e Indústria (MIS-MEDIA 2013) do ICME 2013~\cite{silva2013jnsieee} (Qualis B! CAPES).
Também consegui publicar um artigo sobre o NCL4WEB no DocEng 2013~\cite{silva2013ncl4web} (Qualis A1 CAPES).

Para a minha banca de mestrado, contei com a presença da minha orientadora, Débora Christina Muchaluat Saad; de Vanessa
Braganholo, também professora da UFF (professora da disciplina de dados semi-estruturados, na qual aprendi XSLT);
e do professor Marcelo Ferreira Moreno, da Universidade Federal de Juiz de Fora (uma das pessoas que ajudou e ainda
ajuda no desenvolvimento da linguagem NCL e no padrão brasileiro de televisão digital). Em agosto de 2013, defendi minha
dissertação de mestrado: "JNS e NCL4WEB: Auxiliando o Desenvolvimento e Divulgação de Documentos NCL"~\cite{silva2013jns}.




%==============================================================================
\chapter{Atuação Profissional apos Mestrado}
\label{cap_atuacao}

\begin{summarybox}[frametitle=\faInfoCircle{}\quad Resumo da Atuação Profissional]
  \begin{datelist}
    2013 & Professor do Curso de Cabeamento Estruturado no IFRN Parnamirim pelo PRONATEC \\
    2014-2015 & Atuação como Desenvolvedor Moodle na SEDIS pela FUNPEC \\
    2015 & Participação da organização do capitulo local da Global Game Jam \\
    2015-atual  & Técnico de TI - Servidor Federal da UFRN \\
    2015-2020 & Participação no Projeto AVASUS \\
    2016 & Autoria no Software Registrado "AVASUS 1.0 - AMBIENTE VIRTUAL DE APRENDIZAGEM DOD SUS" \\
    2018 & Autoria na Publicação de Relatos no ESUD 2018 \\
    2021 & Autoria no Software Registrado "The Manager - Gerenciador de Projetos" \\
    2022 & Autoria no Software Registrado "Paper Maker" \\
    2022 & Participação em 4 bancas de Especilização em Informática na Saúde \\
  \end{datelist}
\end{summarybox}

Meu intuito após o mestrado era fazer o doutorado. No entanto, por pretender me casar antes disso, por razões pessoais,
mudei-me para a cidade de Natal/RN, após ser aprovado em um processo seletivo para desenvolvedor de jogos na FUNPEC.
Ao chegar, passei em um concurso para docente do IFRN, pelo Pronatec, para as disciplinas de Cabeamento Estruturado e
Redes de Acesso, e Planejamento e Projeto de Redes de Computadores. Trabalhei por 1 mês no IFRN no final de 2013.

Em 2014, fui chamado pela UFRN para o cargo de desenvolvedor de jogos na Secretaria de Educação a Distância (SEDIS).
Me envolvi com o coletivo de criadores de jogos de Natal na época, chamado BiteByte, hoje conhecido como PONG.
Auxiliei no capítulo local da Global Game Jam no início de 2015. Tentei vários concursos para professor em 2015, e o
que mais estive próximo da aprovação foi o concurso para Professor de Jogos Digitais no IMD. Também tentei concursos
para técnico administrativo, passando no concurso da UFRN para técnico de TI. Continuei trabalhando na SEDIS com o
sistema Moodle, onde aprendi como o Moodle armazena os dados, para criar relatórios e integrações com outros sistemas,
como o SIGAA.

Em 2015, a SEDIS iniciou uma parceria com o Ministério da Saúde para desenvolver um sistema de ensino para o SUS, o
AVASUS. Em um primeiro momento, o sistema foi voltado apenas para o programa Mais Médicos, mas logo foi aberto a toda
a população. O sistema logo alcançou quase um milhão de usuários, e muitas pesquisas foram feitas com os dados dele,
nas quais pude ajudar extraindo as informações necessárias. Também participei do desenvolvimento do sistema para a
confecção de papers utilizado na especialização em Informática na Saúde, o Paper Maker. Além do registro do AVASUS,
participei do registro de outros dois softwares: o gerenciador de projetos The Manager, feito em PHP, e no framework
JavaScript React.js. Também me envolvi com a comunidade de desenvolvimento Moodle, ajudando a resolver algumas issues
e sendo autor de 2 plugins para o Moodle~\cite{moodle_tool_deletemessage,moodle_tool_sentry}, que foram publicados no
portal oficial do Moodle.

A especialização em Informática na Saúde é feita através do AVASUS, e tive a honra de fazer parte de 4 bancas de projeto
final dela em 2022.

Em 2022, me envolvi com um grupo de conservação de abelhas nativas em Natal, realizando o resgate de colônias e dando
palestras e exposições de conscientização. Ao dar essas palestras, senti o desejo de tentar novamente a carreira
docente, por gostar de ajudar as pessoas a aprender. Por isso, comecei a escrever projetos para o doutorado, me
inscrevendo em 2023 e sendo aceito no doutorado em 2024 no Programa de Pós-Graduação em Sistemas e Computação (PPgSC/UFRN).




%==============================================================================
\chapter{Pesquisa Atual no Doutorado}
\label{cap_pesquisa}
Não esperava ficar tanto tempo distante da academia e da publicação. No entanto, meu envolvimento com projetos, aliado
à falta de foco e maturidade na época, contribuiu para isso.
Após a pandemia de COVID-19, em 2020-2021, realizei uma reavaliação da minha vida e dos meus objetivos profissionais.
Decidi então que a docência seria o caminho que eu gostaria de seguir. Com isso, retomei a escrita e decidi tentar o
doutorado, dez anos após a conclusão do mestrado.

Inicialmente, meu plano era iniciar uma linha de pesquisa em jogos digitais aplicados ao ensino. Com o conhecimento que
adquiri em sistemas de aprendizado e jogos, desenvolvi um plugin para integrar a engine Godot ao SCORM~\cite{godot_scorm}.
Ao ser aceito pela professora Lyrene Fernandes, direcionei meu projeto para a área de Engenharia de Software.

Atualmente, o projeto tem evoluído para o uso de IAs generativas no auxílio à criação de testes em jogos digitais.
Aprendendo com minha trajetória, tenho me dedicado mais à escrita, tendo já produzido três artigos e enviado dois para
publicação. Infelizmente, nenhum dos artigos foi aceito até o momento. Um deles foi crucial para a escolha do tema atual,
pois realizei uma revisão sistemática sobre CodeSmells em projetos de jogos, abordando suas causas e soluções. Percebi
que o uso de testes poderia ser uma resposta para os problemas derivados da baixa qualidade de código nesse campo.

Durante o doutorado, tive a oportunidade de aprofundar meus conhecimentos em Inteligência Artificial. Escrevi um artigo
(não aceito) para o ENIAC, explorando o uso de modelos para prever o número de citações em um artigo com base no Google
Trends.

Esse aprendizado evoluiu rapidamente, levando à criação de um projeto de código aberto para facilitar a elaboração de
revisões sistemáticas, utilizando IAs para classificação de textos. O projeto possui uma versão em
Python~\cite{sistematicreviem} e uma versão web acessível \href{https://esdrascaleb.github.io/websm/}{aqui}.

Atualmente, tenho trabalhado no código do ChatTester~\cite{yuan2023no}, com o objetivo de criar um sistema automatizado
para a geração de testes com correção de código, utilizando autocorreção de prompts.



\chapter{Projeto de Atuação Profissional}
\label{cap_proje}

Além de ministrar aulas, pretendo desenvolver projetos com IA para auxiliar startups e empresas júnior da cidade na
implementação de modelos generativos em seus serviços e produtos. O objetivo é ensinar estratégias para o uso de IAs
quantizadas e de baixo custo operacional, permitindo a implementação diretamente no cliente e, assim, reduzindo os
custos para as startups, especialmente as empresas do Parque Tecnológico do IMD.

Também planejo iniciar projetos piloto para o uso de IA em diversas áreas:

\textbf{Inclusão Digital para Idosos}: Ajudar idosos a navegar no ambiente digital com segurança, por meio de cursos
voltados ao uso de assistentes virtuais. O intuito é reduzir sua vulnerabilidade a informações falsas e promover sua
inclusão digital.

\textbf{Tutoria Virtual para Estudantes}: Implementar pipelines de modelos para criar tutores virtuais que forneçam
suporte personalizado aos alunos em disciplinas específicas, ajudando a melhorar o desempenho acadêmico e a reduzir
a carga de trabalho dos professores.

\textbf{Secretários Virtuais Acadêmicos}: Desenvolver secretários virtuais para simplificar processos acadêmicos,
facilitando a vida dos alunos e aumentando a eficiência dos servidores.

\textbf{Gestão Inteligente de Energia e Resíduos}: Implementar modelos inteligentes para otimizar o uso de energia no
campus, alertando sobre luzes ou ar-condicionado deixados ligados em salas, e para uma coleta eficiente de lixo, com
notificações automáticas para agilizar o recolhimento.

\textbf{Apoio à Acessibilidade}: Auxiliar no desenvolvimento de soluções que promovam a acessibilidade, como cães-guia
virtuais para pessoas com deficiência visual, leitores de Libras mais avançados para deficientes auditivos e assistentes
para pessoas com dificuldade de socialização.

Espero alinhar os ideais do IA360 com esses projetos e estar aberto a contribuir em outras iniciativas, conforme necessário.




%==============================================================================
\chapter{Conclusão}
\label{cap_conclusao}

Minha trajetória acadêmica e profissional é marcada pela busca constante por aprendizado, inovação e contribuição à
sociedade por meio da tecnologia. Cada etapa, desde os meus primeiros projetos na graduação até os desafios atuais no
doutorado, moldou minhas competências e consolidou minha paixão pela docência e pela pesquisa aplicada.

Minha motivação para atuar no IMD está diretamente ligada ao potencial de impactar positivamente a formação de
profissionais, bem como à oportunidade de desenvolver soluções práticas para desafios reais. Acredito que minha
experiência em áreas como sistemas interativos, ensino a distância e inteligência artificial, aliada ao meu
comprometimento com a inclusão digital e acessibilidade, me permitirá agregar valor ao Instituto e contribuir para o
sucesso de iniciativas como o IA360.

Estou determinado a continuar expandindo meu conhecimento e promovendo inovações que transcendam a academia, criando
um impacto duradouro na indústria, na sociedade e no ambiente educacional. Este memorial reflete não apenas o percurso
que me trouxe até aqui, mas também minha visão de futuro, com o desejo de gerar um impacto positivo no ensino e
fortalecer a comunidade acadêmica.


%==============================================================================
\backmatter
\bibliographystyle{apalike-doi}
\bibliography{references}

\end{document}

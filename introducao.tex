\section{Introdução}


Tive meu primeiro contato com a ciência e o ensino superior observando o
trabalho de meus pais, ambos Professores do Instituto de Biociências da
Universidade Estadual Paulista ``Júlio de Mesquita Filho'' (UNESP) de Botucatu,
São Paulo.
Até hoje associo as bancadas de microscópios e o cheiro de formol com o
conceito de ciência, mesmo nunca tendo trabalho em um laboratório com essas
características.
A curiosidade, dedicação e ética de meus pais formou a base da minha posição a
respeito da ciência e o que significa ser um educador.

Ingressei no curso de Bacharelado em Geofísica da Universidade de São
Paulo (USP) em 2004.
De agosto de 2008 a maio de 2009 realizei um intercâmbio internacional na York
University, Canadá, onde busquei aprimorar meus conhecimentos de geodésia e de
gravimetria.
No final de 2009 concluí o Bacharelado defendendo o trabalho de conclusão de
curso intitulado
``Cálculo do tensor gradiente gravimétrico utilizando tesseroides''
(disponível em \url{https://doi.org/10.6084/m9.figshare.963547})
sob orientação da Profa. Dra. Naomi Ussami e em colaboração com a
Profa. Dra. Carla Braitenberg da University of Trieste, Itália.
Em 2010 iniciei o Mestrado em Geofísica no Observatório Nacional (ON) com bolsa
da CAPES e sob orientação da Profa. Dra. Valéria C. F. Barbosa.
A convite da Profa. Braitenberg, passei o mês de fevereiro de 2011 em Trieste
dando continuidade no projeto que havia começado durante a graduação
(o software {\em Tesseroids}).
Defendi minha dissertação de Mestrado intitulada ``Robust 3D gravity gradient
inversion by planting anomalous densities'' em outubro de 2011 (disponível em
\url{http://www.leouieda.com/about/masters.html}).
Em novembro de 2011 dei início ao Doutorado em Geofísica no ON, ainda sob
orientação da Profa. Valéria e novamente com bolsa da CAPES.
Em abril de 2016 defendi minha tese de Doutorado intitulada ``Modelagem direta
e inversão de campos gravitacionais em coordenadas esféricas''
(disponível em \url{http://www.leouieda.com/about/phd.html}).

Em outubro de 2013 prestei e fui aprovado no concurso público para o cargo de
Professor Assistente no Departamento de Geologia Aplicada da Faculdade de
Geologia da Universidade do Estado do Rio de Janeiro (UERJ).
Tomei posse na UERJ em fevereiro de 2014, cerca de 2 anos antes de concluir o
Doutorado.
Sou responsável pelo Laboratório de Geofísica de Exploração (LAGEX) e pelas
disciplinas Geofísica 1 e 2 do curso de Bacharelado em Geologia e Matemática
Especial 1 do curso de Bacharelado em Oceanografia.
Em 2016 tive a imensa honra e felicidade de ser homenageado como Paraninfo da
turma de formandos do curso de Geologia da UERJ.

Em outubro de 2016 fui selecionado para realizar um pós-doutorado na University
of Hawaii, E.U.A., sob supervisão do Prof. Dr. Paul Wessel.
O projeto terá duração de 2 anos e foi financiado pela National Science
Foundation (NSF).
Seu objetivo é o desenvolvimento de uma interface na linguagem de programação
Python para o software Generic Mapping Tools (GMT), criado e desenvolvido pelo
Prof. Wessel e colaboradores.
Em fevereiro de 2017 recebi licença da UERJ para realizar o pós-doutorado e me
mudei para Honolulu, Havaí, E.U.A., onde me encontro atualmente.

Sou autor de 10 artigos publicados nos periódicos científicos internacionais
{\em Nonlinear Processes in Geophysic},
{\em The Leading Edge},
{\em Journal of Applied Geophysics},
{\em Ore Geology Reviews},
{\em Geophysics} e
{\em Geophysical Journal International}.
Publiquei 7 resumos e 11 trabalhos completos em anais do eventos (2 nacionais e
16 internacionais).
Fui o apresentador de 12 desses trabalhos nos eventos
{\em EAGE Conference and Exhibition},
{\em SEG International Exposition and Annual Meeting},
{\em International Congress of the Brazilian Geophysical Society},
{\em International GOCE User Workshop},
{\em International Symposium on Gravity, Geoid and Height Systems},
{\em AGU Meeting of the Americas},
{\em EGU General Assembly} e
{\em Python in Science Conference (Scipy)}.
Sou o criador e principal desenvolvedor dos programas de código aberto
(software livre)
{\em Tesseroids} (\url{http://tesseroids.leouieda.com}),
{\em Fatiando a Terra} (\url{http://www.fatiando.org}) e
{\em GMT/Python} (\url{https://github.com/GenericMappingTools/gmt-python}).

Desde meu primeiro contato com a programação sou encantado e influenciado pelos
ideais do movimento software livre ({\em free software movement}),
principalmente pela transparência, colaboração e reutilização que se encontra
nessa comunidade.
Esses ideais também foram adotados pela chamada ``Ciência Aberta'' e se
transferiram naturalmente para minha carreira de pesquisa e ensino.
Disponibilizo na internet o material didático de minhas disciplinas, as
apresentações que elaboro e o código fonte utilizado em minhas publicações como
primeiro autor.
Faço essa distribuição principalmente através das páginas
{\em Github} (\url{https://github.com/leouieda} e
\url{https://github.com/pinga-lab}) e
{\em figshare} (\url{http://figshare.com/authors/Leonardo%20Uieda/97471})
e de minha página pessoal (\url{http://www.leouieda.com}).
Como um esforço para aumentar a transparência do meu trabalho, tenho publicado
em minha página textos sobre o andamento de meus projetos.

Sem dúvida, os produtos de maior impacto resultantes de minha pesquisa são os
programas {\em Tesseroids} e {\em Fatiando a Terra}.
As citações em artigos científicos que ambos programas
receberam\footnote{Segundo a base Google Scholar
(\url{https://scholar.google.com.br/citations?user=qfmPrUEAAAAJ&hl=en}).}
demonstram que são utilizados internacionalmente.
Um fator importante para seu sucesso foi seu desenvolvimento no formato de
software livre desde sua concepção.
Além disso, tenho me esforçado para recrutar novos desenvolvedores e
colaboradores para os projetos.
Atualmente, o {\em Fatiando a Terra} recebeu contribuições de 13 pessoas de 6
países
diferentes\footnote{\url{http://www.fatiando.org/dev/contributors.html}},
7 das quais eu ainda não conheci em pessoa.
Utilizo o {\em Fatiando} como parte integral de minha pesquisa e das
disciplinas e cursos que ministro.
Atualmente, esse programa é utilizado como a base da maioria dos trabalhos
desenvolvidos pelo Grupo de Pesquisa em Problemas Inversos em Geofísica
(\url{http://www.pinga-lab.org}), do qual faço parte.
\\[0.5cm]

A seguir, apresento uma análise reflexiva sobre os principais temas de minha
vida acadêmica: meus projetos de software livre, minha pesquisa em problemas
inversos e minhas atividades de ensino.

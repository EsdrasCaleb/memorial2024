\section{Curriculum Vitae}


\subsection{Formação acadêmica}
%%%%%%%%%%%%%%%%%%%%%%%%%%%%%%%%%%%%%%%%%%%%%%%%%%%%%%%%%%%%%%%%%%%%%%%%%%%%%%%

\begin{itemize}
    \item \textbf{Pós-doutorado} (02/2017 - Presente),
        University of Hawaii, Honolulu, E.U.A.
        Projeto: Expansion of the Generic Mapping Tools (GMT) to the Python
        programming language.
        Supervisor: Paul Wessel.
        Bolsista da National Science Foundation (NSF), E.U.A.
    \item \textbf{Doutorado em Geofísica} (11/2011 - 04/2016),
        Observatório Nacional, Rio de Janeiro, Brasil.
        Tese: Modelagem direta e inversão de campos gravitacionais em
        coordenadas esféricas.
        Orientadora: Valéria Cristina Ferreira Barbosa.
        Bolsista da Coordenação de Aperfeiçoamento de Pessoal de Nível
        Superior (CAPES).
    \item \textbf{Mestrado em Geofísica} (02/2010 - 10/2011),
        Observatório Nacional, Rio de Janeiro, Brasil.
        Dissertação: Robust 3D gravity gradient inversion by planting anomalous
        densities.
        Orientadora: Valéria Cristina Ferreira Barbosa.
        Bolsista da Coordenação de Aperfeiçoamento de Pessoal de Nível
        Superior (CAPES).
    \item \textbf{Intercâmbio Internacional} (08/2008 - 05/2009),
        York University, Toronto, Canadá.
    \item \textbf{Bacharelado em Geofísica} (02/2004 - 12/2009),
        Universidade de São Paulo, São Paulo, Brasil.
        Trabalho de conclusão: Cálculo do tensor gradiente gravimétrico
        utilizando tesseroides.
        Orientadora: Naomi Ussami.
        Bolsista da Sociedade Brasileira de Geofísica (SBGf).
\end{itemize}


\subsection{Atuação profissional}
%%%%%%%%%%%%%%%%%%%%%%%%%%%%%%%%%%%%%%%%%%%%%%%%%%%%%%%%%%%%%%%%%%%%%%%%%%%%%%%

\begin{itemize}
    \item \textbf{Professor Assistente} (02/2014 - Presente), regime de 40
        horas semanais,
        Universidade do Estado do Rio de Janeiro, Rio de Janeiro, Brasil.
        Coordenador do Laboratório de Geofísica de Exploração. Responsável
        pelas disciplinas Geofísica 1, Geofísica 2 e Matemática Especial I.
\end{itemize}


\subsection{Coordenação de Projetos}
%%%%%%%%%%%%%%%%%%%%%%%%%%%%%%%%%%%%%%%%%%%%%%%%%%%%%%%%%%%%%%%%%%%%%%%%%%%%%%%

\begin{itemize}
    \item \textbf{Projeto Qualitec 2014 para bolsista de Nível Superior}
        (10/2014 - Presente).
        Bolsa para treinamento de um técnico de nível superior para o
        Laboratório de Geofísica de Exploração (LAGEX).
        Financiador: Universidade do Estado do Rio de Janeiro.
\end{itemize}


\subsection{Revisor de periódicos}
%%%%%%%%%%%%%%%%%%%%%%%%%%%%%%%%%%%%%%%%%%%%%%%%%%%%%%%%%%%%%%%%%%%%%%%%%%%%%%%

\begin{itemize}
    \item Computers \& Geosciences - Início em 2011.
    \item Geophysics - Início em 2013.
    \item Central European Journal of Geosciences (Open Geosciences) - Início em 2013.
    \item Pure and Applied Geophysics - Início em 2015.
    \item Journal of Applied Geophysics - Início em 2015.
    \item Geophysical Prospecting - Início em 2015.
    \item Geophysical Journal International - Início em 2015.
\end{itemize}

\subsection{Artigos publicados}
%%%%%%%%%%%%%%%%%%%%%%%%%%%%%%%%%%%%%%%%%%%%%%%%%%%%%%%%%%%%%%%%%%%%%%%%%%%%%%%

\begin{enumerate}
\item \textbf{Uieda, L.}, and V. C. F. Barbosa (2017), Fast nonlinear gravity inversion in spherical coordinates with application to the South American Moho, Geophys. J. Int., 208(1), 162-176, doi:10.1093/gji/ggw390.
\item \textbf{Uieda, L.} (2017), Step-by-step NMO correction, The Leading Edge, 36(2), 179-180, doi:10.1190/tle36020179.1.
\item \textbf{Uieda, L.}, V. Barbosa, and C. Braitenberg (2016), Tesseroids: Forward-modeling gravitational fields in spherical coordinates, GEOPHYSICS, F41-F48, doi:10.1190/geo2015-0204.1.
\item Carlos, D. U., \textbf{L. Uieda}, and V. C. F. Barbosa (2016), How two gravity-gradient inversion methods can be used to reveal different geologic features of ore deposit -- A case study from the Quadrilátero Ferrífero (Brazil), Journal of Applied Geophysics, doi:10.1016/j.jappgeo.2016.04.011.
\item Oliveira Jr., V. C., D. P. Sales, V. C. F. Barbosa, and \textbf{L. Uieda} (2015), Estimation of the total magnetization direction of approximately spherical bodies, Nonlin. Processes Geophys., 22(2), 215-232, doi:10.5194/npg-22-215-2015.
\item \textbf{Uieda, L.}, V. C. Oliveira Jr., and V. C. F. Barbosa (2014), Geophysical tutorial: Euler deconvolution of potential-field data, The Leading Edge, 33(4), 448-450, doi:10.1190/tle33040448.1.
\item Carlos, D. U., \textbf{L. Uieda}, and V. C. F. Barbosa (2014), Imaging iron ore from the Quadrilátero Ferrífero (Brazil) using geophysical inversion and drill hole data, Ore Geology Reviews, 61, 268-285, doi:10.1016/j.oregeorev.2014.02.011.
\item Melo, F. F., V. C. F. Barbosa, \textbf{L. Uieda}, V. C. Oliveira, and J. B. C. Silva (2013), Estimating the nature and the horizontal and vertical positions of 3D magnetic sources using Euler deconvolution, GEOPHYSICS, 78(6), J87-J98, doi:10.1190/geo2012-0515.1.
\item Oliveira Jr., V. C., V. C. F. Barbosa, and \textbf{L. Uieda} (2013), Polynomial equivalent layer, GEOPHYSICS, 78(1), G1-G13, doi:10.1190/geo2012-0196.1.
\item \textbf{Uieda, L.}, and V. C. F. Barbosa (2012), Robust 3D gravity gradient inversion by planting anomalous densities, GEOPHYSICS, 77(4), G55-G66, doi:10.1190/geo2011-0388.1.
\end{enumerate}

\subsection{Trabalhos completos publicados em anais de eventos}
%%%%%%%%%%%%%%%%%%%%%%%%%%%%%%%%%%%%%%%%%%%%%%%%%%%%%%%%%%%%%%%%%%%%%%%%%%%%%%%

\begin{enumerate}
\item Melo, F. F., V. C. F. Barbosa, \textbf{L. Uieda}, V. C. O. Jr, and J. B. C. Silva (2014), A Single Euler Solution Per Anomaly, in 76th EAGE Conference and Exhibition 2014.
\item \textbf{Uieda, L.}, V. C. Oliveira Jr, and V. C. F. Barbosa (2013), Modeling the Earth with Fatiando a Terra, in Proceedings of the 12th Python in Science Conference, edited by S. van der Walt, J. Millman, and K. Huff, pp. 91-98.
\item \textbf{Uieda, L.}, and V. C. F. Barbosa (2012), Use of the ``shape-of-anomaly'' data misfit in 3D inversion by planting anomalous densities, in SEG Annual Meeting, pp. 1-6, Society of Exploration Geophysicists.
\item Carlos, D. U., \textbf{L. Uieda}, Y. Li, V. C. F. Barbosa, M. A. Braga, G. Angeli, and G. Peres (2012), Iron ore interpretation using gravity-gradient inversions in the Carajás, Brazil, in SEG Annual Meeting, pp. 1-5, Society of Exploration Geophysicists.
\item Oliveira Jr., V. C., V. C. F. Barbosa, and \textbf{L. Uieda} (2012), Camada Equivalente Polinomial, in V Simpósio Brasileiro de Geofísica.
\item \textbf{Uieda, L.}, E. P. Bomfim, C. Braitenberg, and E. Molina (2011), Optimal forward calculation method of the Marussi tensor due to a geologic structure at GOCE height, in Proceedings of the 4th International GOCE User Workshop.
\item \textbf{Uieda, L.}, and V. C. F. Barbosa (2011), Robust 3D gravity gradient inversion by planting anomalous densities, in SEG Annual Meeting.
\item \textbf{Uieda, L.}, and V. C. F. Barbosa (2011), 3D gravity gradient inversion by planting density anomalies, in 73th EAGE Conference and Exhibition incorporating SPE EUROPEC.
\item \textbf{Uieda, L.}, and V. C. Barbosa (2011), 3D gravity inversion by planting anomalous densities, in 12th International Congress of the Brazilian Geophysical Society.
\item Carlos, D. U., \textbf{L. Uieda}, V. C. F. Barbosa, M. A. Braga, and A.  A. S. Gomes (2011), In-depth imaging of an iron orebody from Quadrilatero Ferrifero using 3D gravity gradient inversion, in SEG Annual Meeting.
\item Carlos, D. U., V. C. Barbosa, \textbf{L. Uieda}, and M. A. Braga (2011), Inversão de Dados de Aerogradiometria Gravimétrica 3D-Ftg Aplicada a Exploração Mineral na Região do Quadrilátero Ferrífero, in 12th International Congress of the Brazilian Geophysical Society.
\end{enumerate}

\subsection{Programas de computador}
%%%%%%%%%%%%%%%%%%%%%%%%%%%%%%%%%%%%%%%%%%%%%%%%%%%%%%%%%%%%%%%%%%%%%%%%%%%%%%%

\begin{itemize}
    \item \textbf{Tesseroids} (2009 - Presente).
        Página oficial: \url{http://tesseroids.leouieda.com}.
        Código fonte: \url{https://github.com/leouieda/tesseroids}.
        Linguagem de programação: C.
        Licença: BSD 3-clause License.
        DOI da versão mais recente (1.2.1):
        \url{https://doi.org/10.5281/zenodo.582366}.
    \item \textbf{Fatiando a Terra} (2010 - Presente).
        Página oficial: \url{http://fatiando.org}.
        Código fonte: \url{https://github.com/fatiando/fatiando}.
        Linguagem de programação: Python.
        Licença: BSD 3-clause License.
        DOI da versão mais recente (0.5):
        \url{https://doi.org/10.5281/zenodo.157746}.
    \item \textbf{GMT/Python} (2017 - Presente).
        Página oficial: \url{https://genericmappingtools.github.io/gmt-python/}.
        Código fonte: \url{https://github.com/GenericMappingTools/gmt-python}.
        Linguagem de programação: Python.
        Licença: BSD 3-clause License.
\end{itemize}

\subsection{Apresentações de trabalho}
%%%%%%%%%%%%%%%%%%%%%%%%%%%%%%%%%%%%%%%%%%%%%%%%%%%%%%%%%%%%%%%%%%%%%%%%%%%%%%%


\subsection{Prêmios e títulos}
%%%%%%%%%%%%%%%%%%%%%%%%%%%%%%%%%%%%%%%%%%%%%%%%%%%%%%%%%%%%%%%%%%%%%%%%%%%%%%%


\subsection{Participações em bancas}
%%%%%%%%%%%%%%%%%%%%%%%%%%%%%%%%%%%%%%%%%%%%%%%%%%%%%%%%%%%%%%%%%%%%%%%%%%%%%%%


\subsection{Cursos de curta duração ministrados}
%%%%%%%%%%%%%%%%%%%%%%%%%%%%%%%%%%%%%%%%%%%%%%%%%%%%%%%%%%%%%%%%%%%%%%%%%%%%%%%

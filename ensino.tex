\section{Ensino}

Assim como muitos outros professores universitários, eu nunca recebi treinando
formal em técnicas de ensino.
Por isso, busquei ler o máximo que pude a respeito desse assunto para me
familiarizar com o seu estado da arte.
Uma tendência que observei com frequência é a mudança para um estilo de ensino
envolvendo menos palestras e mais atividades práticas.
Alguns educadores levam essa prática ao extremo, dedicando $100\%$ do tempo em
sala de aula a exercícios e atividades práticas.
Outra tendência comum é o uso de técnicas de avaliação formativa, que são
aplicadas de forma contínua durante o curso e têm função diagnóstica.
Por exemplo, um professor pode utilizar questões de múltipla escolha para
rapidamente avaliar quais são as falhas no entendimento dos alunos.
Dessa forma, o professor é capaz de adaptar sua aula para sanar essas dúvidas.
Busco sempre incorporar essas técnicas nas minhas aulas e aprimorar minhas
habilidades didáticas.
Utilizo a programação para criar um material didático interativo para minhas
aulas práticas, muitas vezes utilizando as funções existentes no
\textit{Fatiando a Terra}.
Além disso, coleto a opinião dos alunos de forma anônima no final de cada
curso.
Essas avaliações me permitem determinar o que devo manter como parte do curso e
o que pode ser feito para melhorá-lo.
Assim como todo o resto da minha produção, disponibilizo o material didático
que desenvolvo através da minha página pessoal
\url{http://www.leouieda.com/teaching}.


\subsection{Cursos de curta duração}

Minha primeira experiência como instrutor foi quando ministrei o minicurso
``Tópicos de inversão em geofísica'' junto com o Prof. Dr. Vanderlei C.
Oliveira Jr. na XVI Escola de Verão de Geofísica do
IAG/USP\footnote{\url{https://github.com/pinga-lab/inversao-iag-2012}}.
Na época, éramos alunos de doutorado no Observatório Nacional.
Desde este primeiro curso, decidimos utilizar uma abordagem prática para as
aulas e utilizamos o \textit{Fatiando a Terra} nos exercícios e exemplos.
O curso foi bem recebi pelos alunos, muitos dos quais fizeram comentários
positivos a respeito das atividades práticas.
A apostila que escrevemos para esse curso está disponível livremente em
\url{https://doi.org/10.6084/m9.figshare.1192984.v3} utilizando uma licença
\textit{Creative Commons}.
Atualmente (10 de Julho de 2017), a apostila foi baixada mais de 400 vezes.

Depois dessa experiência eu tive certeza de que ensinar é algo que eu gostaria
de fazer como minha carreira.
Após assumir o cargo de Professor na UERJ, ministrei novamente o minicurso de
inversão na Universidade de Brasília, dessa vez
sozinho\footnote{\url{https://github.com/pinga-lab/inversao-unb-2014}}.
Reutilizei grande parte do material original que desenvolvemos, porém adaptado
para um curso de menor duração.

Em 2016, ministrei dois minicursos sobre programação em Python com ênfase em
ciências da Terra.
O primeiro foi ``Python como uma ferramenta numérica em Ciências da
Terra: Uma nova abordagem de
programação''\footnote{\url{https://github.com/leouieda/verao2016}}, ministrado
durante
a XVIII Escola de Verão de Geofísica do IAG/USP em conjunto com os Profs. Drs.
Marcelo Bianchi e Victor Sacek.
Nesse curso, ensinei os conceitos básicos da utilização do \textit{Fatiando a
Terra} e de outras bibliotecas do Python.
O segundo foi ``Python para
Geologia''\footnote{\url{https://github.com/leouieda/python-geologia-2016}},
ministrado durante a
VII Semana Acadêmica de Geologia da UERJ.
Objetivo de ambos os cursos era fornecer o conhecimento mínimo necessário para
que os alunos pudessem começar a estudar a linguagem Python por conta própria.
Em 2017, ministrei o minicurso ``Introduction to Python
Workshop''\footnote{\url{https://github.com/leouieda/python-hawaii-2017}} na
University of Hawaii adaptando o material desenvolvido para o curso da UERJ.
Escrevi a respeito desse curso, dos métodos de ensino utilizados e das
avaliações que recebi dos alunos em
\url{http://www.leouieda.com/blog/python-hawaii-2017.html}.


\subsection{Disciplinas de graduação na UERJ}

Entrada UERJ

Disciplinas

Homenagem


\subsection{Orientações, coorientações e treinamento}

Orientação do Vinícius e Fernanda.

Santiago

Qualitec e Victor

\section{Ensino}

Assim como muitos outros professores universitários, eu nunca recebi treinando
formal em técnicas de ensino.
Por isso, busquei ler o máximo que pude a respeito desse assunto para me
familiarizar com o seu estado da arte.
Uma tendência que observei com frequência é a mudança para um estilo de ensino
envolvendo menos palestras e mais atividades práticas.
Alguns educadores levam essa prática ao extremo, dedicando $100\%$ do tempo em
sala de aula a exercícios e atividades práticas.
Outra tendência comum é o uso de técnicas de avaliação formativa, que são
aplicadas de forma contínua durante o curso e têm função diagnóstica.
Por exemplo, um professor pode utilizar questões de múltipla escolha para
rapidamente avaliar quais são as falhas no entendimento dos alunos.
Dessa forma, o professor é capaz de adaptar sua aula para sanar essas dúvidas.
Busco sempre incorporar essas técnicas nas minhas aulas e aprimorar minhas
habilidades didáticas.
Utilizo a programação para criar um material didático interativo para minhas
aulas práticas, muitas vezes utilizando as funções existentes no
\textit{Fatiando a Terra}.
Além disso, coleto a opinião dos alunos de forma anônima no final de cada
curso.
Essas avaliações me permitem determinar o que devo manter como parte do curso e
o que pode ser feito para melhorá-lo.
Assim como todo o resto da minha produção, disponibilizo o material didático
que desenvolvo através da minha página pessoal
\url{http://www.leouieda.com/teaching}.


\subsection{Cursos de curta duração}

Minha primeira experiência como instrutor foi quando ministrei o minicurso
``Tópicos de inversão em geofísica'' junto com o Prof. Dr. Vanderlei C.
Oliveira Jr. na XVI Escola de Verão de Geofísica do
IAG/USP\footnote{\url{https://github.com/pinga-lab/inversao-iag-2012}}.
Na época, éramos alunos de doutorado no Observatório Nacional.
Optamos por utilizar uma abordagem prática para as aulas e utilizamos o
\textit{Fatiando a Terra} nos exercícios e exemplos.
O curso foi bem recebido pelos alunos, muitos dos quais fizeram comentários
positivos a respeito das atividades práticas.
A apostila que escrevemos para esse curso está disponível livremente em
\url{https://doi.org/10.6084/m9.figshare.1192984.v3}.
Atualmente (10 de Julho de 2017), a apostila foi baixada mais de 400 vezes.

Depois dessa experiência eu tive certeza de que ensinar é algo que eu gostaria
de fazer como minha carreira.
Após assumir o cargo de Professor na UERJ, ministrei novamente o minicurso de
inversão na Universidade de Brasília, dessa vez
sozinho\footnote{\url{https://github.com/pinga-lab/inversao-unb-2014}}.
Reutilizei grande parte do material original que desenvolvemos, porém adaptado
para um curso de menor duração.

Em 2016, ministrei dois minicursos sobre programação em Python com ênfase em
ciências da Terra.
O primeiro foi ``Python como uma ferramenta numérica em Ciências da
Terra: Uma nova abordagem de
programação''\footnote{\url{https://github.com/leouieda/verao2016}}, ministrado
durante
a XVIII Escola de Verão de Geofísica do IAG/USP em conjunto com os Profs. Drs.
Marcelo Bianchi e Victor Sacek.
Nesse curso, ensinei os conceitos básicos da utilização do \textit{Fatiando a
Terra}.
O segundo foi ``Python para
Geologia''\footnote{\url{https://github.com/leouieda/python-geologia-2016}},
ministrado durante a
VII Semana Acadêmica de Geologia da UERJ.
Objetivo de ambos os cursos era fornecer o conhecimento mínimo necessário para
que os alunos pudessem começar a estudar a linguagem Python por conta própria.
Em 2017, ministrei o minicurso ``Introduction to Python
Workshop''\footnote{\url{https://github.com/leouieda/python-hawaii-2017}} na
University of Hawaii adaptando o material desenvolvido para o curso da UERJ.
Escrevi a respeito desse curso, dos métodos de ensino utilizados e das
avaliações que recebi dos alunos em
\url{http://www.leouieda.com/blog/python-hawaii-2017.html}.


\subsection{Disciplinas de graduação}

Desde minha contratação na UERJ em 2014, ministrei as disciplinas
\textit{Geologia Geral 1} e \textit{Matemática Especial
I}\footnote{\url{https://github.com/mat-esp}} para o curso de
Oceanografia, \textit{Mineralogia e Petrologia} para o curso de Biologia e
\textit{Geofísica 1}\footnote{\url{https://github.com/leouieda/geofisica1}} e
\textit{Geofísica 2}\footnote{\url{https://github.com/leouieda/geofisica2}}
para o curso de Geologia.

A disciplina Matemática Especial I consiste em uma breve introdução à
programação seguida dos conceitos básicos de cálculo numérico.
Os alunos passam cerca de $80\%$ do tempo em sala de aula trabalhando em grupos
para a solução de exercícios.
Para cada tema abordado, os grupos recebem repositórios individuais na página
\url{https://www.github.com} contendo as instruções para os exercícios.
A entrega dos exercícios é feita de forma digital através desses repositórios
que são agrupados em contas de usuário que eu controlo.
Por exemplo, todas as soluções entregues pelos grupos de 2015 estão na página
\url{https://github.com/mat-esp-2015}.
Foi somente através dessa abordagem digital que eu fui capaz de administrar as
duas turmas de aproximadamente 40 alunos cada do segundo semestre de 2015.


As duas disciplinas de geofísica do curso de geologia são divididas em aulas
teóricas e aulas práticas em proporções iguais.
Em ambas disciplinas, utilizo o \textit{Fatiando a Terra} em conjunto com
documentos chamados \textit{Jupyter
notebooks}\footnote{\url{http://jupyter.org}} para criar os exercícios das
aulas práticas.
Os \textit{notebooks} permitem inserir código, texto, imagens, equações e
vídeos em único documento interativo.
Os alunos trabalham na atividade prática em grupos e são guiados pelo texto dos
\textit{notebooks}, que contém explicações e perguntas.
Em 2016, a primeira turma de alunos que cursou as disciplinas de geofísica
comigo me escolheu como Paraninfo em sua formatura.



\subsection{Orientações, coorientações e treinamento}

Atualmente, sou coorientador da tese de doutorado do aluno Santiago Soler, cujo
orientador é o Prof. Dr. Mario Ernesto Gimenez da Universidad Nacional de San
Juan, Argentina.
Conheci o Santigo através das contribuições que ele submeteu para o
\textit{Fatiando a Terra}.
Em 2016, ele e o Prof. Gimenez me convidaram para ser coorientador no projeto
de doutorado ``Modelos de inversão conjunta de dados gravimétricos e de função
do receptor através do uso de tesseroides''.
O trabalho está em estágio inicial e ainda não possui resultados publicados.

Em 2014, fui selecionado pela UERJ para participar do projeto QUALITEC.
Esse projeto financiava bolsas para técnicos de nível superior que trabalhariam
nos laboratórios da UERJ e seriam treinados suas respectivas áreas de atuação.
Recebi uma dessas bolsas para o Laboratório de Geofísica de Exploração, do qual
sou o coordenador.
Selecionei para a vaga o técnico Victor Thadeu Xavier de Almeida que é
extremamente competente e tem formação sólida em física e geofísica.
Como parte do programa QUALITEC, treinei o Victor na linguagem de programação
Python e nos diversas técnicas do desenvolvimento de software.
Ele fez contribuições para o \textit{Fatiando a Terra} e sua ajuda foi
indispensável nas aulas de Matemática Especial I.

Antes de minha viagem para a University of Hawaii, orientei dois alunos de
iniciação científica.
A aluna Fernanda Vianna Gatts trabalhou no projeto ``Variações do volume do
aquífero Guarani determinadas por dados de gravidade do satélite GRACE''.
O aluno Vinícius Vianna Riguête do curso de Geologia continua desenvolvendo seu
trabalho no projeto ``Estudo gravimétrico das intrusões
Eo-cretácicas da Namíbia''.

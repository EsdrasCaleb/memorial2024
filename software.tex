\section{Software}


\subsection{Formação}

Meu primeiro contato com a programação foi através da disciplina
``Introdução à Computação para Ciências Exatas e Tecnologia''
que cursei em 2004 durante meu primeiro semestre na USP.
Antes disso, eu não tinha conhecimento algum de como programas de computador
são feitos ou que qualquer um poderia criar o seu próprio programa.
Aprendi os conceitos básicos da linguagem de programação C.
Porem, não acalcei um nível suficientemente avançado para enxergar aplicações
imediatas da programação nas demais disciplinas do curso de geofísica.

Busquei aprender mais sobre a linguagem C através da disciplina optativa
``Computação para Geofísicos''.
Nela desenvolvi aplicações diretas à geofísica como o cálculo do International
Geomagnetic Reference Field (IGRF) a partir dos coeficientes de harmônicos
esféricos e o método de \citet{talwani1959} para modelagem direta na
gravimetria.
O código que criei para a disciplina foi utilizado anos depois como base
para a implementação do método de \citet{talwani1959} no programa
\textit{Fatiando a
Terra}\footnote{\url{www.fatiando.org/v0.5/api/gravmag.talwani.html}}.
Essas aplicações me mostraram o enorme poder da programação no aprendizado de
conceitos complexos da geofísica e matemática.
Ao criar uma implementação computacional de um método, o aluno é levado a
considerar detalhes e fazer perguntas que pode não ter notado ao estudar a
teoria.
Além disso, também é capaz de explorar as possibilidades e os limites de uma
teoria de forma dinâmica e independente.
Não exagero quando afirmo que ter cursado a disciplina ``Computação para
Geofísicos'' foi crucial para o resto de minha carreira.

Nos anos seguintes comecei a estudar a programação nas horas vagas e a aplicar
à geofísica o que estava aprendendo.
Implementei a Transformada Discreta de Fourier\footnote{Disponível em
\url{https://github.com/leouieda/dft-in-c}} para estudar para a disciplina
``Processamento de Sinais Digitais''.
Utilizei minha implementação do método de otimização Ant Colony Optimization
\citep{socha2008} para realizar uma inversão das velocidades de grupo de ondas
Love\footnote{Disponível em \url{https://github.com/leouieda/love-aco-inv}}
para a disciplina ``Teoria de Ondas Sísmicas e Estrutura da Terra''.
Para as disciplinas de geodésia que cursei durante meu intercâmbio na York
University, implementei ajustes de redes gravimétricas, mudança de sistemas de
coordenadas geográficas, entre outros.
Aprendi as linguagens de programação C++, Java e Python.
Cursei a disciplina optativa ``Princípios de Desenvolvimento de Algoritmos''
onde aprendi os conceitos que possibilitaram os avanços que obtive em
\citet{tesseroids}.

Meu projeto de conclusão de curso foi o tesseroids (ver a seguir).


Conheci o Software Carpentry em 2008 e tudo mudou.
Descobri o quanto não sabia sobre coisas básicas na indústria.
Principalmente testes e controle de versão.
Comecei a prender sobre essas tecnologias e utilizá-las no meu dia a dia.

Na pós continuei com essa abordagem de implementar tudo.
Comecei a construir o Fatiando.
Muitas coisas se tornaram parte do Fatiando.
Na geotermia implementei a variação climática
http://www.fatiando.org/v0.5/api/geothermal.climsig.html
Na gravimetria implementei as modelagens diretas, etc.

Fui aprendendo com a experiência e melhorando meu Python.

Melhores jeitos de testar.

Como tornar o código mais rápido.

Como utilizar paralelismo de forma básica.

Aprendi os desafios da distribuição de software para múltiplos sistemas
operacionais.

Como criar documentação para um projeto.

Principalmente como facilitar e encorajar colaboradores novos.

Esse último ainda estou aprimorando.







\subsection{Tesseroids}

Em 2007 iniciei um estágio de iniciação científica com a Profa. Dra. Naomi
Ussami.
Meu projeto era parte de uma colaboração com a Profa. Dra. Carla Braitenberg da
University of Trieste, Itália.
O objetivo dessa colaboração era preparar a comunidade científica para lidar
com os dados de gradiometria gravimétrica que seriam coletados pelo satélite
GOCE (Gravity field and steady-state Ocean Circulation Explorer).
Minha parte nesse projeto era desenvolver um programa para a modelagem direta
dos dados utilizando tesseroides (prismas esféricos).

Tesseroids e Colaboração com Carla
AGU 2010
GOCE 2011
Utilizações


\subsection{Fatiando a Terra}

Fatiando
Scipy 2013 e 2014
Boletim sbgf
Palestra USP, ON e UH
Utilizações

Na versão 0.3 de 2014, éramos somente 5 pessoas, todos amigos de faculdade.

Em 2017 a futura versão 0.6 conta com 13 pessoas.


\subsection{GMT/Python}

GMT
Scipy  2017

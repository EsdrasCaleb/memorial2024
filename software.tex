\section{Software}

Cada vez mais fazer pesquisa de ponta envolve desenvolver algum tipo de
software.
Isso é verdade para diversos ramos da ciência, da biologia à geofísica e até
mesmo nas ciências sociais.
Aplicações variam de baixar da internet para seu computador dados de fontes
governamentais
a realizar tomografia sísmica de ponta em um \textit{cluster}.
Na ciência moderna, muitas vezes o software assume o papel da metodologia
experimental.
Logo é de se esperar que os programas e seus códigos fonte sejam submetidos
ao mesmo rigor que impomos no resto da ciência experimental.
Infelizmente essa não é a realidade ainda.
Nos próximos 10 anos, questões envolvendo a avaliação da qualidade e
confiabilidade de resultados computacionais serão centrais no cenário
científico internacional.
No entanto, o sistema de educação superior atual fornece pouco treinamento nas
habilidades necessárias para lidar com essa nova realidade.
Algumas iniciativas, como o \textit{Software Carpentry}
(\url{https://software-carpentry.org/}), estão atacando diretamente esses
problemas oferecendo treinamento computacional direcionado a cientistas.

Minha pesquisa sempre teve um viés computacional.
Ao longo dos últimos dez anos, dei início a três projetos de software livre.
A seguir, discorro sobre minha formação em desenvolvimento de software e sobre
a concepção e os impactos de cada projeto.


\subsection{Formação}

Meu primeiro contato com a programação foi através da disciplina
``Introdução à Computação para Ciências Exatas e Tecnologia''
que cursei em 2004 durante meu primeiro semestre na USP.
Antes disso, eu não tinha conhecimento algum de como programas de computador
são feitos ou que qualquer um poderia criar o seu próprio programa.
Aprendi os conceitos básicos da linguagem de programação C.
Porém, não acalcei um nível suficientemente avançado para enxergar aplicações
imediatas da programação nas demais disciplinas do curso de geofísica.

Busquei aprender mais sobre a linguagem C através da disciplina optativa
``Computação para Geofísicos''.
Nela desenvolvi aplicações diretas à geofísica como o cálculo do International
Geomagnetic Reference Field (IGRF) a partir dos coeficientes de harmônicos
esféricos e o método de \citet{talwani1959} para modelagem direta na
gravimetria.
O código que criei para a disciplina foi utilizado anos depois como base
para a implementação do método de \citet{talwani1959} no programa
\textit{Fatiando a
Terra}\footnote{\url{www.fatiando.org/v0.5/api/gravmag.talwani.html}}.
Essas aplicações me mostraram o enorme poder da programação no aprendizado de
conceitos complexos da geofísica e da matemática.
Ao criar uma implementação computacional de um método, o aluno é levado a
considerar detalhes e fazer perguntas que pode não ter notado ao estudar a
teoria.
Além disso, também é capaz de explorar as possibilidades e os limites de uma
teoria de forma dinâmica e independente.
Não exagero quando afirmo que ter cursado a disciplina ``Computação para
Geofísicos'' foi crucial para o resto de minha carreira.

Nos anos seguintes comecei a estudar a programação nas horas vagas e a aplicar
à geofísica o que estava aprendendo.
Implementei a Transformada Discreta de Fourier\footnote{Disponível em
\url{https://github.com/leouieda/dft-in-c}} para estudar para a disciplina
``Processamento de Sinais Digitais''.
Utilizei minha implementação do método de otimização Ant Colony Optimization
\citep{socha2008} para realizar uma inversão das velocidades de grupo de ondas
Love\footnote{Disponível em \url{https://github.com/leouieda/love-aco-inv}}
para a disciplina ``Teoria de Ondas Sísmicas e Estrutura da Terra''.
Para as disciplinas de geodésia que cursei durante meu intercâmbio na York
University, implementei ajustes de redes gravimétricas, mudança de sistemas de
coordenadas geográficas, entre outros.
Aprendi as linguagens de programação C++, Java e Python.
Cursei a disciplina optativa ``Princípios de Desenvolvimento de Algoritmos''
onde aprendi os conceitos que possibilitaram alguns dos avanços que obtive em
\citet{tesseroids}.

Em 2008 descobri a organização sem fins lucrativos \textit{Software Carpentry}.
Segundo \url{https://software-carpentry.org} (acessado em 7 de julho de 2017),
seu objetivo é ``Ensinar habilidades laboratoriais básicas para pesquisa
computacional''.
Através de seu material, disponível gratuitamente na internet, descobri o
quanto eu não sabia sobre o desenvolvimento sustentável de software para
pesquisa científica.
Aprendi sobre ferramentas fundamentais da computação que até então eu
desconhecia, como sistemas de controle de versão, testes unitários e
rastreamento da procedência de resultados.
Atualmente utilizo essas ferramentas no meu dia-a-dia como pesquisador e
professor, seja para escrever
artigos\footnote{\url{https://github.com/pinga-lab/paper-moho-inversion-tesseroids}}
ou manejar a entrega de trabalhos práticos em minha disciplina de computação e
cálculo numérico\footnote{\url{https://github.com/mat-esp-2016}}.

Durante a pós-graduação continuei com a abordagem de criar uma implementação
computacional do que aprendia nas disciplinas.
No entanto, optei por agrupar todo código fonte que desenvolvi em uma única
biblioteca feita na linguagem Python: o projeto \textit{Fatiando a Terra}.
Para a disciplina de ondas sísmicas, por exemplo, implementei uma solução por
diferenças finitas da equação da
onda\footnote{\url{www.fatiando.org/v0.5/api/seismic.wavefd.html}} para
investigar se ondas Love realmente resultam de ondas S
horizontais\footnote{\url{https://youtu.be/YjhSvEpbzps}} e se ondas Rayleigh
possuem movimento elíptico
retrógrado\footnote{\url{https://youtu.be/Mvd8FANLqy4}}.
Concomitantemente, continuei aprimorando minhas habilidades com o
desenvolvimento de software através da experiência pessoal com o
\textit{Fatiando a Terra}.
Muitas das lições que aprendi não são relacionadas à programação em si, mas à
administração de um projeto de software livre.
Exemplos dessas lições são: as diferentes maneiras de se testar um programa,
os desafios envolvidos na distribuição de programas para diferentes sistemas
operacionais e como encorajar o envolvimento de colaboradores externos.


\subsection{Tesseroids}

Em 2007 iniciei um estágio de iniciação científica com a Profa. Dra. Naomi
Ussami.
Meu projeto era parte de uma colaboração com a Profa. Dra. Carla Braitenberg da
University of Trieste, Itália.
O objetivo dessa colaboração era preparar a comunidade científica para lidar
com os dados de gradiometria gravimétrica que seriam coletados pelo satélite
GOCE (Gravity field and steady-state Ocean Circulation Explorer).
Minha participação nesse projeto seria desenvolver um programa para a modelagem
direta dos dados utilizando tesseroides (prismas esféricos).
Ficou determinado que o método que eu utilizaria para isso é a Quadratura
Gauss-Legendre, como proposto por \citet{asgharzadeh2007} e
\citet{wild-pfeiffer2008}.




Tesseroids e Colaboração com Carla
AGU 2010
GOCE 2011
Utilizações


\subsection{Fatiando a Terra}

Fatiando
Scipy 2013 e 2014
Boletim sbgf
Palestra USP, ON e UH
Utilizações

Na versão 0.3 de 2014, éramos somente 5 pessoas, todos amigos de faculdade.

Em 2017 a futura versão 0.6 conta com 13 pessoas.


\subsection{GMT/Python}

Após terminar meu doutorado em abril de 2016, comecei a procurar oportunidades
para fazer um pós-doutorado no exterior.
Havia passado os últimos dois anos me familiarizando com a vida de professor
universitário, ministrando de duas a quatro disciplinas por semestre e
trabalhando na minha tese de doutorado durante os recessos.
Estava exausto e ponderando o futuro ruma de minha carreira em pesquisa.
Logo, percebi que estava na hora de procurar outra experiência internacional.
Alguns meses depois, recebi um comunicado através de uma lista de emails que o
Prof. Paul Wessel da University of Hawaii, E.U.A., estava procurando candidatos
com experiência em programação em Python para um pós-doutorado.
A responsabilidade do candidato selecionado seria construir uma interface
para acessar os comandos do Generic Mapping Tools (GMT,
\url{http://gmt.soest.hawaii.edu}) através da linguagem Python.
Além disso, também utilizaria essa interface para realizar pesquisas na área de
tectônica de placas e métodos potenciais.
Depois de conversar a respeito com minha esposa, resolvi aplicar para a
posição e fui selecionado\footnote{Escrevi a respeito de minhas escolhas e o
processo de seleção em
\url{http://www.leouieda.com/blog/hawaii-gmt-postdoc.html}}.
Confesso que a perspectiva de morar por dois anos no Havaí contribuiu
de forma não insignificante para minha decisão.
Em fevereiro de 2017 me mudei para Honolulu para iniciar o pós-doutorado.

O projeto \textit{GMT/Python} é uma biblioteca escrita em Python que se
comunica com o GMT através de um mecanismo chamado \textit{foreign function
interface} (FFI).
Esse mecanismo possibilita que funções escritas em uma linguagem de programação
sejam utilizadas em programas feitos em outra linguagem.
Por exemplo, programas feitos em Python podem utilizar a FFI para executar
funções de uma biblioteca escrita em C, a linguagem na qual é feito o GMT.
Um vantagem do Python é que seu interpretador oficial é implementado em C,
facilitando a interação entre as duas linguagens.
A maior dificuldade que encontrei até o presente momento é a enorme
complexidade do GMT.
Existem dezenas de módulos, cada um com diversas opções e particularidades,
implementados em milhares de linhas de código.

O primeiro passo desse projeto foi auxiliar o Prof. Wessel no desenvolvimento
de um modo de excussão moderno (chamado \textit{modern mode}) para o
GMT\footnote{Documentação na página do GMT
\url{http://gmt.soest.hawaii.edu/projects/gmt/wiki/Modernization}}.
Esse modo simplifica a utilização do GMT e introduz comandos novos para
facilitar a criação de figuras complexas com múltiplos gráficos.
A interação com o Prof. Wessel se mostrou indispensável para a elaboração do
\textit{GMT/Python}.
Essa está sendo uma oportunidade única de aprender com alguém que possui
décadas de experiência com software livre.
Outro benefício é a proximidade com a fronteira da pesquisa em tectônica de
placas, que é a especialidade do Prof. Wessel.

Atualmente, o \textit{GMT/Python} está em processo de desenvolvimento.
Já desenvolvi a base necessária para acessar a biblioteca do GMT
e estou no processo de adaptar cada um dos mais de 90 módulos.
Apresentarei sobre os resultados que obtivemos até o momento no congresso
Scientific Computing with Python (Scipy)\footnote{Escrevi sobre o trabalho que
submetemos e as revisões que recebemos em
\url{http://www.leouieda.com/blog/scipy2017-proposal-gmt.html}} em julho de 2017 em Austin, E.U.A.
Optamos por manter o código fonte e nossos planos para o projeto aberto ao
público.
Todo o desenvolvimento e planejamento acontece através da página
\url{https://github.com/GenericMappingTools/gmt-python}.

Promissor pr que muita gente usa o GMT.
No Python, não tem alternativa boa para mapas com projeção  ou contas na
esfera.
Atuamente muita gete do python não usa o GMT por que o custo de produtividade
de trocar de
ambiente e passar dados é muito grande.
Além disso, tem gente que usar o GMT e o Python fazendo gambiarra.
O GPlates depende do GMT e tem uma nova interface Python.
Beneficiaria muito do GMT/Python.

\section{Pesquisa}

Minha pesquisa se concentra na área de problemas inversos em métodos
potenciais.
Geralmente, meus trabalhos são avanços metodológicos e são acompanhados por
um código fonte que os implementa.
Como mencionei no capítulo \ref{software}, acredito que o código fonte que
acompanha uma publicação é tão importante quanto a descrição de sua
metodologia.
Muitas vezes é impossível reproduzir os resultados de um trabalho
sem ter acesso ao software que os gerou.
Logo, é crucial que o código esteja disponível para ser revisado pela
comunidade científica.
Para tanto, disponibilizo o código e dados (à medida do possível) necessários
para reproduzir os resultados de meus trabalhos como primeiro autor.
Cada trabalho é acompanhado de um repositório na página do Grupo de Pesquisa em
Problemas Inversos em Geofísica (\url{https://github.com/pinga-lab}).
Os repositórios contém o código fonte que implementa a metodologia, realiza os
testes com dados sintéticos e produz as figuras para o artigo.
Ultimamente, torno público um repositório no momento da submissão do respectivo
artigo para publicação.
Dessa forma, os revisores tem acesso ao conteúdo total de meus trabalhos e
podem se certificarem de que meus resultados estão corretos.
Cada repositório também é arquivado permanentemente e recebe um Digital Object
Identifier (DOI) através de serviços como Zenodo (\url{http://zenodo.org}) e
figshare (\url{https://figshare.com}).

A seguir, apresento reflexões sobre os aspectos de minha formação que me
levaram a essa área de pesquisa e sobre os diferentes trabalhos que formam
minha produção acadêmica.


\subsection{Formação}

Desde o início da graduação me senti intrigado pelos métodos de inversão.
Sempre ouvia de alunos veteranos, ou até mesmo de professores, que esse era um
assunto extremamente complexo.
Conhecendo minha personalidade, creio que meu interesse inicial sobre o assunto
era puramente devido ao desafio.
Por sorte, a disciplina que forma a base do conhecimento de problemas inversos,
a álgebra linear, também foi um tema que despertou meu interesse.
Por outro lado, devo confessar que, a princípio, os métodos potenciais não me
interessaram tanto quando a inversão.
Minha primeira impressão da gravimetria e magnetometria foi que eram métodos
simples.
Porém, terminei minha primeira disciplina sobre o assunto completamente confuso
e com mais dúvidas do que tinha antes de cursá-la.
Hoje percebo que essas são características de um assunto complexo e
interessante e de uma aula de qualidade.

Fiz minha primeira iniciação científica no laboratório de paleomagnetismo com
bolsa da FAPESP e orientação do Prof. Manoel S. D'Agrella Filho.
Ao final do período de um ano da bolsa, decidi não continuar nessa área.
Em seguida, busquei outro projeto que unisse a geofísica
com meu interesse pela programação.
Em 2007, iniciei o projeto sob orientação da
Profa. Naomi Ussami que resultou no software \textit{Tesseroids}.

Em 2008, realizei um programa de intercâmbio de 10 meses na York University,
Canadá.
Parte do motivo para essa escolha é a excelência do país nas áreas de
gravimetria e geodésia.
Lá, cursei disciplinas sobre geodésia física, gravimetria e ajuste de redes
através do método dos mínimos quadrados.

Em 2010, ingressei no Mestrado em Geofísica do Observatório Nacional sob
orientação da Profa. Valéria C. F. Barbosa.
Meu projeto era criar um método de inversão 3D de dados de gradiometria
gravimétrica, tema que contava com atenção internacional e era pioneiro no
cenário nacional.
Nesse ponto, percebi que as disciplinas cursadas na York me forneceram a base
necessária para compreender a inversão e a gravimetria com muito mais
facilidade.
Aprendi com a Profa. Valéria
as diferentes vertentes e sutilezas da inversão,
a arte da elaboração de artigo científico
e sua ética profissional impecável.
Também devo muito do meu aprendizado durante a pós-graduação às longas
conversas e debates com meu amigo Vanderlei C. Oliveira Jr. (atualmente
pesquisador do Observatório Nacional).



\subsection{Modelagem direta de campos gravitacionais com tesseroides}

Comecei a desenvolver esse tema durante minha iniciação científica com a
Profa. Naomi de 2007 a 2009.
A princípio, meu trabalho simplesmente aprender a metodologia de
\citet{wild-pfeiffer2008} e transformá-la em um programa de computador.
Apresentei meu primeiro trabalho em um evento internacional sobre os resultados
de meu trabalho de conclusão de curso:

\begin{displayquote}
    UIEDA, L.; USSAMI, N.; BRAITENBERG, C. Computation of the gravity
    gradient tensor due to topographic masses using tesseroids. In: AGU Meeting
    of the Americas, 2010.\footnote{Apresentação disponível em
    \url{http://www.leouieda.com/talks/agu2010.html}}
\end{displayquote}

Retomei esse projeto em 2011 durante minha estadia em Trieste com a Profa.
Braitenberg para atualizar o software e a metodologia para o cálculo dos campos
gravitacionais de um tesseroide (prisma esférico).
Utilizei as equações otimizadas de \citet{grombein2013} para eliminar
as singularidades presentes na formulação de \citet{wild-pfeiffer2008}.
Também desenvolvi um método de discretização adaptativa dos tesseroides para
garantir a acurácia da integração numérica com a Quadratura Gauss-Legendre.
Sem meu conhecimento, um algoritmo similar \citep{li2011} havia sido publicado
no mesmo período em que estava em Trieste, inviabilizando a nossa publicação do
método.
Ainda em Trieste, busquei caracterizar o erro numérico envolvido
nos cálculos para melhorar controlar a discretização adaptativa.
Publiquei meus esforços com o software \textit{Tesseroids} e os resultados da
análise do erro da quadratura nos anais do 4th International GOCE User
Workshop:

\begin{displayquote}
    UIEDA, L.; BOMFIM, E. P.; BRAITENBERG, C.; MOLINA, E. C. Optimal
    forward calculation method of the Marussi tensor due to a geologic
    structure at GOCE height. In: 4th International GOCE User Workshop,
    2011.\footnote{Pôster e texto disponíveis em
    \url{http://www.leouieda.com/posters/goce2011.html}}
\end{displayquote}

Para minha tese de doutorado, iria utilizar a modelagem direta com tesseroides
para desenvolver um método de inversão em coordenadas esféricas.
Porém, para que a inversão seja correta é necessário que a modelagem direta
seja o mais precisa possível.
Logo, continuei com o desenvolvimento do algoritmo de discretização adaptativa
e com os experimentos para caracterizar o erro da quadratura.
Após diversas tentativas frustadas, cheguei aos resultados apresentados no
artigo:

\begin{displayquote}
    UIEDA, L; BARBOSA, V. C. F.; BRAITENBERG, C. Tesseroids:
    Forward-modeling gravitational fields in spherical coordinates. Geophysics,
    v. 81, p. F41-F48, 2016.\footnote{Código fonte disponível em
    \url{https://github.com/pinga-lab/paper-tesseroids}}
\end{displayquote}

Neste trabalho, aprimoramos o algoritmo de discretização adaptativa proposto
por \citet{li2011}.
Também determinamos empiricamente valores para o parâmetro
\textit{distance-size ratio}, que controla a discretização adaptativa,
para manter o erro de integração abaixo de $0.1\%$.

Os trabalhos relacionados à essa metodologia estão entre meus trabalhos mais
citados.
Acredito que isso é devido em parte à disponibilização do software
\textit{Tesseroids} que implementa essa metodologia.
Outro fruto dessa pesquisa é minha coorientação da tese de doutorado do aluno
Santiago Soler com o Prof. Dr. Mario Ernesto Gimenez da Universidad Nacional de
San Juan, Argentina.
A tese de Santiago dará continuidade à modelagem direta com tesseroides,
expandindo a formulação para incluir tesseroides com distribuições internas de
densidades variáveis.


\subsection{Inversão 3D utilizando o método de plantação}

Meu projeto de mestrado era desenvolver um método de inversão 3D para dados de
gradiometria gravimétrica.
Um dos desafios enfrentados nessa área é o aumento significativo do número de
dados em uma aquisição gradiométrica comparados com uma aquisição
aerogravimétrica.
Esse aumento tornava a inversão impossível de ser executada nos computadores
que tínhamos disponíveis no Observatório Nacional.

Minha ideia inicial para o método de inversão surgiu durante uma conversa com o
Prof. Dr. João B. C. Silva da Universidade Federal do Pará.
No momento, ele se encontrava no Rio de Janeiro para participar de uma banca de
mestrado.
Durante a conversa, o Prof. João mencionou o trabalho de \citet{rene} como um
exemplo de uma metodologia diferente e pouco reconhecida pela comunidade
científica.
\citet{rene} propôs um método de inversão 2D de dados gravimétricos na qual a
solução cresce em torno de alguns elementos nucleares chamados de ``sementes''.
O que despertou meu interesse nesse trabalho foi o viés computacional do método
proposto, ao invés da abordagem matemática clássica.
Ao estudá-lo, percebi que poderia ser adaptado e aprimorado para o
caso 3D e dados de gradiometria gravimétrica.
Minhas modificações incluem a introdução da função de regularização de
compacidade de \citet{silvadias2009},
um termo de normalização para as funções do ajuste
e a avaliação parcial da matriz de sensibilidade.
Esta última modificação é a que possibilita a inversão de conjuntos grandes de
dados com poucos recursos computacionais.

Apresentei este novo método, denominado ``método de plantação'',  nos
congressos internacionais
73rd EAGE Conference and Exhibition incorporating SPE EUROPEC,
SEG International Exposition and Eighty-First Annual Meeting,
12th International Congress of the Brazilian Geophysical Society,
e
International Symposium on Gravity, Geoid and Height Systems.
Fui premiado com auxílio financeiro da Near Surface Geophysics Section (NSGS)
da SEG para participar do SEG Annual Meeting.
Também obtive o auxílio PACE Student Travel Grant para participar do 73rd EAGE
Conference and Exhibition.

Publiquei trabalhos completos nos anais de 3 desses eventos:

\begin{displayquote}
    UIEDA, L.; BARBOSA, V. C. F. 3D gravity Gradient Inversion by Planting
    Density Anomalies. In: 73rd EAGE Conference and Exhibition incorporating
    SPE EUROPEC, 2011, Vienna. v. 1.\footnote{Pôster e código fonte:
    \url{http://www.leouieda.com/posters/eage2011.html}}
\end{displayquote}

\begin{displayquote}
    UIEDA, L.; BARBOSA, V. C. F. Robust 3D gravity gradient
    inversion by planting anomalous densities. In: SEG International Exposition
    and Eighty-First Annual Meeting, 2011, San Antonio. p.
    820-824.\footnote{Apresentação e código fonte:
    \url{http://www.leouieda.com/talks/seg2011.html}}
\end{displayquote}

\begin{displayquote}
    UIEDA, L.; BARBOSA, V. C. F. 3D gravity inversion by planting
    anomalous densities. In: 12th International Congress of the Brazilian
    Geophysical Society, Rio de Janeiro, Brazil,
    2011.\footnote{Apresentação e código fonte:
    \url{http://www.leouieda.com/talks/sbgf2011.html}}
\end{displayquote}

Subsequentemente, publiquei meu primeiro artigo em periódico:

\begin{displayquote}
    UIEDA, L.; BARBOSA, V. C. F. Robust 3D gravity gradient inversion by
    planting anomalous densities. Geophysics, v. 77, p. G55-G66,
    2012.\footnote{Código fonte disponível em
    \url{https://github.com/pinga-lab/paper-planting-densities}}
\end{displayquote}

O próximo passo no desenvolvimento desse método veio quando, ao ler novamente o
trabalho original de \citet{rene}, percebi as enormes vantagens da utilização
da função ``shape-of-anomaly''.
Essa função é definida e utilizada no trabalho de 1986.
Porém, acredito que René não explorou todas suas implicações para a inversão.
Ao utilizá-la em meu método de plantação, percebi que era capaz de obter
resultados melhores utilizando menos elementos nucleares (as ``sementes'' da
inversão).
Apresentei estes resultados e uma análise do motivo de seu sucesso no congresso
SEG International Exposition and Eighty-Second Annual Meeting,
acompanhado da publicação nos anais do evento:

\begin{displayquote}
    UIEDA, L.; BARBOSA, V. C. F. Use of the shape-of-anomaly data
    misfit in 3D inversion by planting anomalous densities. In: SEG Technical
    Program Expanded Abstracts 2012.\footnote{Apresentação e código fonte:
    \url{http://www.leouieda.com/talks/seg2012.html}}
\end{displayquote}


Durante meu doutorado, adaptei este método para a inversão de dados de anomalia
magnética de campo total com resultados insatisfatórios.
Os resultados da inversão se mostraram extremamente sensíveis à direção de
magnetização total assumida para o alvo.
Não dei continuidade com esse trabalho pois o foco de minha tese havia mudado
para a inversões em escala regional.
Apresentei meus resultados no evento AGU Meeting of the Americas de 2013:

\begin{displayquote}
    UIEDA, L.; BARBOSA, V. C. F. 3D magnetic inversion by planting anomalous
    densities. In: AGU Meeting of the Americas, 2013,
    Cancun.\footnote{Apresentação e código fonte:
    \url{http://www.leouieda.com/talks/agu-cancun2013.html}}
\end{displayquote}

Em seguida, adaptei o método de plantação para utilizar tesseroides ao invés de
prismas retangulares retos.
Dessa forma, poderia realizar a inversão em coordenadas esféricas e em escala
regional.
No entanto, o método de plantação assume que os alvos da inversão são corpos
contínuos e com contraste de densidade abrupto em relação às estruturas
encaixantes.
Embora o método funcione em testes com dados sintéticos,
tive dificuldade de encontrar situações reais onde o método se aplicaria.
Apresentei esses resultados no EGU General Assembly de 2014:

\begin{displayquote}
    UIEDA, L.; BARBOSA, V. C. F. Gravity inversion in spherical
    coordinates using tesseroids. In: EGU General Assembly 2014.
    EGU2014-10898-1.\footnote{Apresentação e código fonte:
    \url{http://www.leouieda.com/talks/egu2014.html}}
\end{displayquote}


O método de plantação foi utilizado na tese de doutorado de Dionísio U. Carlos,
que na época era aluno da Profa. Valéria, para modelar dados de gradiometria do
Quadrilátero Ferrífero.
Dessa colaboração com o Dionísio foram publicados 3 trabalhos completos em
anais de eventos e 2 artigos em periódicos internacionais:

\begin{displayquote}
    CARLOS, D. U.; BARBOSA, V. C. F.; UIEDA, L.; BRAGA, M. A. Inversão de
    dados de aerogradiometria gravimétrica 3D-FTG aplicada a exploração mineral
    na região do Quadrilátero Ferrífero. In: 12th International Congress of the
    Brazilian Geophysical Society, 2011.
\end{displayquote}

\begin{displayquote}
    CARLOS, D. U.; UIEDA, L.; BARBOSA, V. C. F.; BRAGA,
    M. A.; GOMES, A. A. S. In-depth imaging of an iron
    orebody from Quadrilatero Ferrifero using 3D gravity gradient inversion.
    In: SEG International Exposition and Eighty-First Annual Meeting, 2011.
\end{displayquote}

\begin{displayquote}
    CARLOS, D. U,; UIEDA, L.; LI, Y.; BARBOSA, V.
    C. F.; BRAGA, M. A. ; ANGELI, G.; PERES, G. Iron ore
    interpretation using gravity-gradient inversions in the Carajás, Brazil.
    In: SEG Technical Program Expanded Abstracts 2012.
\end{displayquote}

\begin{displayquote}
    CARLOS, D. U.; UIEDA, L.; BARBOSA, V. C. F. Imaging
    iron ore from the Quadrilátero Ferrífero (Brazil) using geophysical
    inversion and drill hole data. Ore Geology Reviews, v. 61, p. 268-285,
    2014.
\end{displayquote}

\begin{displayquote}
    CARLOS, D. U.; UIEDA, L.; BARBOSA, V. C. F. How two
    gravity-gradient inversion methods can be used to reveal different geologic
    features of ore deposit - A case study from the Quadrilátero Ferrífero
    (Brazil). Journal of Applied Geophysics, v. 130, p. 153-168, 2016.
\end{displayquote}


O método de plantação \citep{seed} é meu trabalho mais citado\footnote{Segundo
a base Google Scholar:
\url{https://scholar.google.com.br/citations?user=qfmPrUEAAAAJ&hl=en}} e o que
me rendeu o maior número de publicações.
Este trabalho marcou minha iniciação
à apresentação de trabalhos em congressos no exterior e
à escrita e publicação de artigos científicos.
Também serviu como um primeiro experimento das possibilidades de tornar
pesquisa mais transparente, acessível e reprodutível.


\subsection{Inversão 3D em coordenadas esféricas}


Começou com o último trabalho da sementes (EGU 2014).
Inversão da Moho


\subsection{Camada Equivalente}

A técnica da camada equivalente é utilizada para o processamento de dados de
métodos potenciais.
Suas aplicações incluem a interpolação de dados, continuação para cima, redução
ao polo, remoção de ruídos aleatórios e cálculo de derivadas.
A camada equivalente é aplicada em dois passos:
primeiro, estimamos os coeficientes de uma série de funções harmônicas que
melhor ajustam os dados observados;
segundo, utilizamos os coeficientes estimados para realizar a transformação
desejada através da modelagem direta.
É comum utilizar como as funções harmônicas o efeito de uma malha regular de
fontes pontuais.
Dessa forma, os coeficientes estimados possuem significado físico.
Por exemplo, no caso da gravimetria os coeficientes representam as densidades
de massas pontuais.
A camada equivalente é capaz de operar em dados distribuídos de forma irregular
e é geralmente mais estável que métodos que utilizam a Transformada de Fourier.
Porém, o primeiro passo para sua aplicação é uma inversão, o que a torna
computacionalmente custosa.

Das muitas conversas que tive com o Vanderlei durante a pós-graduação,
surgiu a ideia de parametrizar a camada equivalente utilizando polinômios
bidimensionais no lugar de fontes pontuais.
A camada seria dividida em janelas e a distribuição de propriedades físicas
dentro de cada janela seria representada por um polinômio.
Essa parametrização nos possibilitaria estimar os coeficientes desses
polinômios ao invés de estimar os valores de propriedade física de cada fonte
pontual.
Esta mudança reduz drasticamente o número de parâmetros a serem estimados na
inversão.

Essa ideia foi executada pelo Vanderlei e se tornou parte de sua tese de
doutorado.
Publicamos o método desenvolvido na revista \textit{Geophysics} em 2013:

\begin{displayquote}
    OLIVEIRA Jr., V. C.; BARBOSA,  V. C. F.; UIEDA, L. Polynomial equivalent
    layer. GEOPHYSICS, 78(1), G1-G13, 2013.
\end{displayquote}


\subsection{Estimação da direção de magnetização}

magdir


\subsection{Deconvolução de Euler}

Colaboração com Figura.


\subsection{Tutoriais sobre geofísica}

Tutorial Euler
Tutorial NMO

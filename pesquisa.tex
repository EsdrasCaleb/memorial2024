\section{Pesquisa}

\citet{uieda_fast_2017}

Posição sobre pesquisa e ciência aberta
Iniciação na paleo
Iniciação com tesseroides

Durante a graduação, realizei meu primeiro estágio de iniciação científica no
laboratório de paleomagnetismo com bolsa da FAPESP e orientação do Prof. Manoel
S. D'Agrella Filho. Em seguida, busquei outro projeto que unisse a geofísica
com meu interesse em programação. Em 2008, iniciei um projeto sob orientação da
Profa. Naomi Ussami em colaboração com a Profa. Carla Braitenberg da
Universidade de Trieste, Itália. O objetivo era desenvolver um software para
modelagem de campos gravitacionais em coordenadas esféricas. Apresentei o
resultado desse projeto, o software Tesseroids, como meu trabalho de conclusão
de curso com bolsa da Sociedade Brasileira de Geofísica. Paralelamente,
realizei um programa de intercâmbio de 10 meses na York University, Canadá.

Em 2010, ingressei no Mestrado sob orientação da Profa. Valéria C. F. Barbosa.
Meu projeto era criar um método de inversão 3D de dados de gradiometria
gravimétrica, tema que contava com atenção internacional e era pioneiro no
cenário nacional. Ao mesmo tempo, comecei a desenvolver o software Fatiando a
Terra para uso em minha dissertação e nas disciplinas de pós-graduação.
Publiquei meu primeiro artigo com os resultados da dissertação na revista
Geophysics e em 3 anais de congressos internacionais. Defendi meu mestrado em
outubro de 2011. O método que desenvolvi foi utilizado na tese de Doutorado de
Dionísio U. Carlos, aluno da Profa. Valéria. Dessa colaboração surgiram 2
artigos publicados em revistas internacionais e 3 trabalhos completos em anais
de eventos internacionais.

Iniciei meu Doutorado em novembro de 2011, ainda sob orientação da Profa.
Valéria. Meu objetivo para a tese era retornar à ciência básica ao invés de
continuar as aplicações à mineração, como fiz no Mestrado. Para tanto, iria
aplicar a metodologia criada para o software Tesseroids na inversão de dados
gravimétricos para estudar a crosta terrestre. O desenvolvimento desse software
ainda estava em andamento. A convite da Profa. Carla Braitenberg, passei o mês
de fevereiro de 2011 na Universidade de Trieste trabalhando na versão 1.0. Nos
anos seguintes, aprimorei a metodologia implementada no software. Em 2016,
publiquei meus resultados e a versão 1.2 na revista Geophysics. Este trabalho
se tornou a primeira parte da minha tese. Simultaneamente, continuei a
construção do Fatiando a Terra, com apresentações no congresso internacional
SciPy em 2013 e 2014. O trabalho publicado nos anais do evento de 2013 se
tornou a segunda parte da minha tese. Em 2015, iniciei o trabalho em um método
de inversão para determinação da interface crosta-manto em escala regional.
Este trabalho tem como base a metodologia desenvolvida para o Tesseroids e a
infraestrutura do Fatiando a Terra. Essa base permitiu que esse trabalho fosse
desenvolvido durante os breves períodos de recesso acadêmico da UERJ em 2015 e
parte de 2016. O artigo referente a esse trabalho foi publicado na revista
Geophysical Journal International e compõe a terceira e última parte de minha
tese.



\subsection{Modelagem direta com tesseroides}

Começa com Naomi. Carla. Doutorado.
Paper.
Viagem para Trieste.

\subsection{Inversão 3D utilizando o método de plantação}

Sementes
Colaboração com Dio
Congressos
SEG e EAGE 2011 travel grant.


\subsection{Inversão 3D em coordenadas esféricas}


Começou com o último trabalho da sementes (EGU 2014).
Inversão da Moho


\subsection{Tutoriais sobre geofísica}

Tutorial Euler
Tutorial NMO


\subsection{Colaborações}


\subsubsection{Aplicações do método de plantação}

Colaborações com Dio.


\subsubsection{Camada Equivalente}

Colaborações com Biroca: PEL,


\subsubsection{Estimação da direção de magnetização}

magdir


\subsubsection{Deconvolução de Euler}

Colaboração com Figura.

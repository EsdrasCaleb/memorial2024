\section{Pesquisa}

Minha pesquisa se concentra na área de problemas inversos em métodos
potenciais.
Geralmente, meus trabalhos são avanços metodológicos e são acompanhados por
um código fonte que os implementa.
Como mencionei no capítulo \ref{software}, acredito que o código fonte que
acompanha uma publicação é tão importante quanto a descrição de sua
metodologia.
Muitas vezes é impossível reproduzir os resultados de um trabalho
sem ter acesso ao software que os gerou.
Logo, é crucial que o código esteja disponível para ser revisado pela
comunidade científica.
Para tanto, disponibilizo o código e dados (à medida do possível) necessários
para reproduzir os resultados de meus trabalhos como primeiro autor.
Cada trabalho é acompanhado de um repositório na página do Grupo de Pesquisa em
Problemas Inversos em Geofísica (\url{https://github.com/pinga-lab}).
Cada repositório também é arquivado permanentemente e recebe um Digital Object
Identifier (DOI) através de serviços como Zenodo (\url{http://zenodo.org}) e
figshare (\url{https://figshare.com}).

A seguir, apresento uma reflexões sobre os aspectos de minha formação que me
levaram a essa área de pesquisa e sobre os diferentes trabalhos que formam
minha produção acadêmica.


\subsection{Formação}

Desde o início da graduação me senti intrigado pelos métodos de inversão.
Sempre ouvia de alunos veteranos, ou até mesmo de professores, que esse era um
assunto extremamente complexo.
Conhecendo minha personalidade, creio que meu interesse inicial sobre o assunto
era puramente devido ao desafio.
Por sorte, a disciplina que forma a base do conhecimento de problemas inversos,
a álgebra linear, também foi um tema que despertou meu interesse.
Por outro lado, devo confessar que, a princípio, os métodos potenciais não me
interessaram tanto quando a inversão.
Minha primeira impressão da gravimetria e magnetometria foi que eram métodos
simples.
Porém, terminei minha primeira disciplina sobre o assunto completamente confuso
e com mais dúvidas do que tinha antes de cursá-la.
Hoje percebo que essas são características de um assunto complexo e
interessante e de uma aula de qualidade.

Fiz minha primeira iniciação científica no laboratório de paleomagnetismo com
bolsa da FAPESP e orientação do Prof. Manoel S. D'Agrella Filho.
Ao final do período de um ano da bolsa, decidi não continuar nessa área.
Em seguida, busquei outro projeto que unisse a geofísica
com meu interesse pela programação.
Em 2007, iniciei o projeto sob orientação da
Profa. Naomi Ussami que resultou no software \textit{Tesseroids}.

Em 2008, realizei um programa de intercâmbio de 10 meses na York University,
Canadá.
Parte do motivo para essa escolha é a excelência do país nas áreas de
gravimetria e geodésia.
Lá, cursei disciplinas sobre geodésia física, gravimetria e ajuste de redes
através do método dos mínimos quadrados.

Em 2010, ingressei no Mestrado em Geofísica do Observatório Nacional sob
orientação da Profa. Valéria C. F. Barbosa.
Meu projeto era criar um método de inversão 3D de dados de gradiometria
gravimétrica, tema que contava com atenção internacional e era pioneiro no
cenário nacional.
Nesse ponto, percebi que as disciplinas cursadas na York me forneceram a base
necessária para compreender a inversão e a gravimetria com muito mais
facilidade.
Aprendi com a Profa. Valéria
as diferentes vertentes e sutilezas da inversão,
a arte da elaboração de artigo científico
e sua ética profissional impecável.
Também devo muito do meu aprendizado durante a pós-graduação às longas
conversas e debates com meu amigo Vanderlei C. Oliveira Jr. (atualmente
pesquisador do Observatório Nacional).



\subsection{Modelagem direta com tesseroides}

Sobre o algoritmo em si, não o software. Tanto.

Comecei com \citet{wild-pfeiffer2008} e fórmula 2D.
Tem singularizadade.
Mistura com 3D quando precisa.

Em trieste, atualiza pra Grombein e implementei a primeira versão da
discretização adaptativa usando recursivo.

Tentei caracterizar o erro pra saber o quanto discretizar.

Dificil.

Publicamos resumo e poster no GOCE.

2 meses antes da 1.0 (ver no github) saiu o paper do Li.


Várias tentativas de analisar o efeito da curvatura.

O grande desafio foi achar um jeito de achar o D.


Duas metodologias diferentes para realizar esses cálculos haviam sido propostas
neste mesmo ano: a Quadratura Gauss-Legendre \citep{asgharzadeh2007}
e a expansão em série de Taylor \citep{heck2007}.
\citet{wild-pfeiffer2008} mostrou que a Quadratura Gauss-Legendre (QGL) é



\subsection{Inversão 3D utilizando o método de plantação}

Sementes
Colaboração com Dio
Congressos
SEG e EAGE 2011 travel grant.


\subsection{Inversão 3D em coordenadas esféricas}


Começou com o último trabalho da sementes (EGU 2014).
Inversão da Moho


\subsection{Tutoriais sobre geofísica}

Tutorial Euler
Tutorial NMO


\subsection{Colaborações}


\subsubsection{Aplicações do método de plantação}

Colaborações com Dio.


\subsubsection{Camada Equivalente}

Colaborações com Biroca: PEL,


\subsubsection{Estimação da direção de magnetização}

magdir


\subsubsection{Deconvolução de Euler}

Colaboração com Figura.
